\chapter{Компактні метричні простори}

\section{Зв'язки між видами компактності}

\begin{definition}
Нехай $A$ --- деяка множина в метричному
просторі $(X, \rho)$ і $\epsilon$ --- деяке додатне число. Множина $B$
із цього простору називається \vocab{$\epsilon$-сіткою для множини $A$},
якщо $\forall x \in A$ $\exists y \in B$: $\rho(x, y) < \epsilon$.
\end{definition}

\begin{definition}
Множина $A$ називається \vocab{цілком обмеженою},
якщо для неї при довільному $\epsilon > 0$ існує скінченна $\epsilon$-сітка.
\end{definition}

\begin{theorem}
[Хаусдорф] Нехай $(X, \rho)$--- метричний
простір. Наступні твердження є еквівалентними.
\begin{enumerate}
\item $(X, \rho)$ --- компактний;
\item $(X, \rho)$ --- повний і цілком обмежений;
\item із довільної післідовності точок простору $(X, \rho)$
можна вибрати збіжну підпослідовність
(\vocab{секвенціальна компактність});
\item довільна нескінченна підмножина в $X$ має хоча б одну
граничну точку (\vocab{зліченна компактність}).
\end{enumerate}
\end{theorem}

\begin{proof} $1 \implies 2 \implies 3 \implies 4 \implies 1$.

Покажемо, що $1 \implies 2$. Нехай $(X, \rho)$ --- компактний
простір. Покажемо його повноту. Нехай $\{x_n\}$ ---
фундаментальна послідовність в $X$. Покладемо
$A_n = \{x_n, x_{n + 1}, \dots\}$ і $B_n = \closure A_n$. Оскільки система $\{B_n\}$ є
центрованою системою замкнених підмножин, то
$\Bigcap_{i = 1}^\infty B_i$ ---
непорожня множина. Нехай $x_0 \in \Bigcap_{i = 1}^\infty B_i$. Тоді
\begin{equation*}
    \forall \epsilon > 0 \forall N > 0 \exists n > N: \rho(x_0, x_n) < \epsilon
\end{equation*}
\begin{equation*}
    \forall \epsilon > 0 \exists N > 0 \forall n, m > N: \rho(x_n, x_m) < \epsilon
\end{equation*}
\begin{equation*}
    \forall \epsilon > 0 \exists N > 0 \forall n, m > N:
    \rho(x_0, x_m) \le \rho(x_0, x_n) + \rho(x_n, x_m) < 2 \epsilon.
\end{equation*}

З цього випливає, що
\begin{equation*}
    x_0 = \lim_{n \to \infty} x_n \in X.
\end{equation*}

Отже, $(X, \rho)$ --- повний простір.
Припустимо тепер, що простір $(X, \rho)$ не є цілком
обмеженим. Інакше кажучи, припустимо, що існує таке
число $\epsilon_0$ таке, що в $X$ немає скінченної $\epsilon_0$-сітки.
Візьмемо довільну точку $x_1 \in X$.
\begin{enumerate}
    \item $\exists x_2 \in  X$: $\rho(x_1, x_2) > \epsilon_0$.
    Інакше точка $x_1$ утворювала б $\epsilon_0$-сітку в $X$.

    \item $\exists x_3 \in  X$: $\rho(x_1, x_3), \rho(x_2, x_3) > \epsilon_0$.
    Інакше точки $x_1, x_2$ утворювали б $\epsilon_0$-сітку в $X$.

    \dots

    \item[$n$.] $\exists x_{n + 1} \in  X$: $\rho(x_i, x_{n + 1}) > \epsilon_0$,
    $i = 1, 2, \dots, n$.
    Інакше точки $x_1, x_2, \dots, x_n$ утворювали б $\epsilon_0$-сітку в $X$.
    
    \dots
\end{enumerate}

Таким чином, ми побудували послідовність $\{x_n\}$, яка не є
фундаментальною, а, отже, не має границі. З цього
випливає, що кожна із множин $A_n = \{x_n, x_{n +1}, \dots\}$, які
утворюють центровану систему, є замкненою. Їх перетин є
порожнім. Це протирічить компактності простору $(X, \rho)$ .

Покажемо, що $2 \implies 3$. Нехай $\{x_n\}$ --- послідовність
точок $X$.
\begin{enumerate}
\item Виберемо в $X$ скінченну $1$-сітку і побудуємо навколо
кожної з точок, що її утворюють, кулю радіуса $1$:
$S_i(a_i, 1)$, $i = 1, \dots, N_1$. Оскільки $X$ є цілком обмеженою,
\begin{equation*}
    \Bigcup_{i = 1}^{N_1} S_i(a_i, 1) = X.
\end{equation*}

З цього випливає, що принаймні одна куля, скажімо,
$S_1$, містить нескінченну підпослідовність $\{x_n^{(1)}\}_{n = 1}^\infty$
послідовності $\{x_n\}$.

\item Виберемо в $X$ скінченну $\frac{1}{2}$-сітку і побудуємо навколо
кожної з цих точок, що її утворюють кулю радіуса $\frac{1}{2}$:
$S_i(b_i, \frac{1}{2})$, $i = 1, 2, \dots, N_2$. Оскільки множина $X$ є цілком
обмеженою,
\begin{equation*}
    \Bigcup_{i = 1}^{N_2} S_i(b_i, \tfrac{1}{2}) = X.
\end{equation*}

З цього випливає, що принаймні одна куля, скажімо,
$S_2$, містить нескінченну підпослідовність $\{x_n^{(2)}\}_{n = 1}^\infty$
послідовності $\{x_n^{(1)}\}_{n = 1}^\infty$.

\dots

\item[$m$.] Виберемо в $X$ скінченну $\frac{1}{m}$-сітку і побудуємо
навколо кожної з цих точок, що її утворюють кулю
радіуса $\frac{1}{m}$: $S_i(c_i, \frac{1}{m})$, $i = 1, 2, \dots, N_m$. Оскільки
множина $X$ є цілком обмеженою,
\begin{equation*}
    \Bigcup_{i = 1}^{N_m} S_i(c_i, \tfrac{1}{m}) = X.
\end{equation*}

З цього випливає, що принаймні одна куля, скажімо,
$S_m$, містить нескінченну підпослідовність $\{x_n^{(m)}\}_{n = 1}^\infty$
послідовності $\{x_n^{(m - 1)}\}_{n = 1}^\infty$.

\dots
\end{enumerate}

Продовжимо цей процес до нескінченності.
Розглянемо діагональну послідовність $\{x_n^(n)\}_{n = 1}^\infty$. Вона є
підпослідовністю послідовності $\{x_n\}_{n = 1}^\infty$. Крім того, при
$m \ge n_0$: $x_m^{(m)} \in \{x_n^{(n_0)}\} \subset S_{n_0}$.
Це означає, що $\{x_n^{(n)}\}$ є
фундаментальною і внаслідок повноти $(X, \rho)$ має границю.

Твердження $3 \implies 4$ є тривіальним, оскільки із довільної
нескінченної множини можна виділити зліченну множину
$\{x_n\}_{n = 1}^\infty$, яка внаслідок секвенціальної компактності містить
збіжну підпослідовність: $\{x_{n_k}\}_{k = 1}^\infty \to x_0 \in X$.

Покажемо тепер, що $4 \implies 1$. Для цього спочатку
доведемо, що множина $X$ є цілком обмеженою, тобто в ній
для довільного числа $\epsilon > 0$ існує $\epsilon$-сітка. Якщо б це було не
так, то застосувавши той же прийом, що і на етапі $1 \implies 2$, ми
побудували б послідовність $\{x_n\}_{n = 1}^\infty$, яка не має граничних
точок, оскільки вона не є фундаментальною. Для кожного $n$
побудуємо скінченну $\frac{1}{n}$-сітку і розглянемо об’єднання всіх
таких сіток. Воно є щільним і не більше ніж зліченним.
Таким чином, простір $(X, \rho)$ є сепарабельним, отже, має
зліченну базу.

Для того щоб довести компактність простору, що має
зліченну базу, достатньо перевірити, що із будь-якого
зліченного (а не довільного нескінченного) відкритого
покриття можна виділити скінченне підпокриття.
Припустимо, що $\{U_\alpha\}$ --- довільне покриття простору
$(X, \rho)$ , а $\{V_n\}$ --- його зліченна база. Кожна точка $x \in X$
міститься в деякому $U_\alpha$. За означенням бази знайдеться
деяке $V_i \in \{V_n\}$ таке, що $x \in V_i \subset U_\alpha$. Якщо кожній точці
$x \in X$ поставити у відповідність окіл $V_i \in \{V_n\}$, то
сукупність цих околів утворить зліченне покриття
множини $X$.

Залишилося довести, що із довільного зліченного
відкритого покриття множини $X$ можна вибрати скінченне
підпокриття. Для цього достатньо довести еквівалентне
твердження для замкнених підмножин, що утворюють
зліченну центровану систему.

Нехай $\{F_n\}_{n = 1}^\infty$ --- центрована система замкнених
підмножин $X$. Покажемо, що
\begin{equation*}
    \Bigcap_{n = 1}^\infty F_n \ne \emptyset.
\end{equation*}

Нехай $\Phi_n = \Bigcap_{k = 1}^n F_k$. Ясно, що множини $\Phi_n$ є замкненими і
непорожніми, оскільки система $\{F_n\}_{n = 1}^\infty$ є центрованою, і
\begin{equation*}
    \Phi_1 \supset \Phi_2 \supset \dots, \quad
    \Bigcap_{n = 1}^\infty \Phi_n = \Bigcap_{n = 1}^\infty F_n.    
\end{equation*}

Можливі два випадки.
\begin{enumerate}
\item Починаючи з деякого номера
\begin{equation*}
    \Phi_{n_0} = \Phi_{n_0 + 1} = \dots = \Phi_{n_0 + k} = \dots
\end{equation*}

Тоді
\begin{equation*}
    \Bigcap_{n = 1}^\infty F_n = \Bigcap_{n = 1}^\infty \Phi_n = \Phi_{n_0} \ne \emptyset.
\end{equation*}

\item Серед $\Phi_n$ є нескінченно багато попарно різних.
Достатньо розглянути випадок, коли всі вони
відрізняються одна від одної. Нехай $x_n \in \Phi_n \setminus \Phi_{n + 1}$.
Тоді послідовність $\{x_n\}$ є нескінченною множиною
різних точок із $X$ і, внаслідок уже доведеного факту
(зліченна компактність), має хоча б одну граничну
точку $x_0$. Оскільки $\Phi_n$ містить всі точки $x_n, x_{n + 1}, \dots$ то
$x_0$ --- гранична точка для кожної множини $\Phi_n$ і
внаслідок замкненості $\Phi_n$
\begin{equation*}
    \forall n \in \NN: x_0 \in \Phi_n.
\end{equation*}

Отже,
\begin{equation*}
    x_0 \in \Bigcap_{n = 1}^\infty \Phi_n = \Bigcap_{n = 1}^\infty F_n,
\end{equation*}
тобто $\Bigcap_{n = 1}^\infty F_n$ є непорожнім. \qedhere
\end{enumerate}
\end{proof}

\section{Література}

\begin{enumerate}[label={[\arabic*]}]
\item \textbf{Садовничий~В.~А.}
Теория операторов /
В.~А.~Садовничий ---
М.: Изд-во Моск. ун-та, 1986 (стр.~49--51).
\end{enumerate}