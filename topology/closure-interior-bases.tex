\chapter{Методи введення топології}

\section{Замикання і внутрішність}

Система аксіом, наведена в означенні топології належить
радянському математику П.С.~Александрову (1925). Проте
першу систему аксіом, що визначає топологічну структуру,
запропонував польський математик К.~Куратовський (1922).

\begin{definition}
Нехай $X$ --- довільна множина. Відображення
$\Closure: 2^X \to 2^X$ називається \vocab{оператором замикання
Куратовського на $X$}, якщо воно задовольняє наступні
умови (\vocab{аксіоми Куратовського}):
\begin{enumerate}
\item[К1.] $\Closure(M \cup N) = \Closure(M) \cup \Closure(N)$ (адитивність);
\item[К2.] $M \subset \Closure(M)$;
\item[К3.] $\Closure(\Closure(M)) = \Closure(M)$ (ідемпотентність);
\item[K4.] $\Closure(\emptyset) = \emptyset$.
\end{enumerate}
\end{definition}

\begin{theorem}
Якщо в деякій множині $X$ введено топологію в
розумінні Александрова, то відображення $\Closure$, що
задовольняє умові $\Closure (M) = \closure{M}$ є оператором Куратовського
на $X$.
\end{theorem}

\begin{proof}
Неважко помітити, що аксіоми К1--К4 просто
збігаються із властивостями замикання, доведеними в
теоремі про властивості замикання.
\end{proof}

\begin{theorem}[про завдання топології оператором Куратовського]
Кожний оператор Куратовського $\Closure$ на
довільній множині $X$ задає в $X$ топологію
$\tau = \{ U \subset X: \Closure(X \setminus U) = X \setminus U\}$ в розумінні Александрова, до
того ж замикання $\closure{M}$ довільної підмножини $M$ із $X$ в цій
топології $\tau$ збігається з $\Closure(M)$, тобто $\Closure (M) = \closure{M}$.
\end{theorem}

\begin{proof}
Побудуємо сімейство \[\sigma = \{ M \subset X: M = X \setminus U, U \in \tau\}, \]
що складається із всіх можливих доповнень множин із
системи $\tau$, тобто таких множин, для яких $\Closure(M) = \closure{M}$. Інакше
кажучи, система $\sigma$ складається з нерухомих точок оператора
замикання Куратовського. За принципом двоїстості де
Моргана, для сімейства $\sigma$ виконуються аксіоми замкненої
топології

\begin{enumerate}
\item[F1.] $X, \emptyset \in \sigma$.
\item[F2.] $F_\alpha \in \sigma, \alpha \in A \implies \bigcap_{\alpha \in A} F_\alpha \in \sigma$, де $A$ --- довільна множина.
\item[F3.] $F_\alpha \in \sigma, \alpha = 1, 2, \ldots, n \implies \bigcup_{\alpha = 1}^n G_\alpha \in \sigma$.
\end{enumerate}

Отже, щоб перевірити аксіоми Александрова для
сімейства множин $\tau$, достатньо перевірити виконання аксіом
F1--F3 для сімейства множин $\sigma$.

\begin{enumerate}
\item Перевіримо аксіому F1: $X \in \sigma$? $\emptyset \in \sigma$?

Аксіома K2 стверджує, що $M \subset \Closure(M)$. Покладемо $M = X$.
Отже, $X \subset \Closure(X) \subset X \implies \Closure(X) = X \implies X \in \sigma$.
Аксіома К4 стверджує, що $\Closure(\emptyset) = \emptyset \implies \emptyset \in \sigma$.

\item Перевіримо виконання аксіоми F2.

Спочатку покажемо, що оператор $\Closure$ є \vocab{монотонним}:
\[\forall A, B \in \sigma: A \subset B \implies \Closure(A) \subset \Closure(B).\]

Нехай $A, B \in \sigma$ і $A \subset B$. Тоді за аксіомою К1:
\[ \Closure(B) = \Closure(B \cup A) = \Closure(B) \cup \Closure(A).\]

Отже, \[ \Closure(A) \subset \Closure(A) \cup \Closure(B) = \Closure(B \cup A) = \Closure(B). \]

Використаємо це допоміжне твердження для перевірки
аксіоми F3. З одного боку,

\begin{align*}
& \forall F_\alpha \in \sigma: \quad \bigcap_{\alpha \in A} F_\alpha \subset F_\alpha \quad \forall \alpha \in A \implies \\
& \implies \Closure \left(\bigcap_{\alpha \in A} F_\alpha \right) \in \Closure (F_\alpha) = F_\alpha \quad \forall \alpha \in A \implies \\
& \implies \Closure \left(\bigcap_{\alpha \in A} F_\alpha \right) \subset \bigcap_{\alpha \in A} F_\alpha.
\end{align*}

З іншого боку, за аксіомою К2 \[ \bigcap_{\alpha \in A} F_\alpha \subset \Closure \left(\bigcap_{\alpha \in A} F_\alpha \right). \]

Отже, \[ \Closure \left(\bigcap_{\alpha \in A} F_\alpha \right) = \bigcap_{\alpha \in A} F_\alpha \in \sigma. \]

\item Перевіримо виконання аксіоми F3. \[A, B \in \sigma \implies \Closure(A\cup B) = \Closure(A)\cup \Closure(B) = A\cup B \implies A\cup B\in\sigma.\]
\end{enumerate}

Таким чином, $\sigma$ --- замкнена топологія, а сімейство $\tau$, що
складається із доповнень до множин із сімейства $\sigma$ ---
відкрита топологія.

Залишилося показати, що в просторі $(X, \tau)$, побудованому
за допомогою оператора $\Closure$, замикання $\closure{M}$ довільної
множини $M$ збігається з $\Closure(M)$.

Дійсно, за критерієм замкненості, множина $M$ є замкненою, якщо
$\closure{M} = M$. З аксіом К2 і К3 випливає, що множина $\Closure(M)$ є
замкненою і містить $M$. Покажемо, що ця множина ---
найменша замкнена множина, що містить множину $M$,
тобто є її замиканням.

Нехай $F$ --- довільна замкнена в $(X, \tau)$ множина, що
містить $M$: \[ M \subset F, \quad \Closure(F) = F. \]

Внаслідок монотонності оператора $\Closure$ отримуємо
наступне: \[ M \subset F, \Closure(F) = F \implies \Closure(M) \subset \Closure(F) = F. \qedhere \]
\end{proof}

\begin{definition}
Нехай $X$ --- довільна множина. Відображення $\Interior: 2^X \to 2^X$ називається \vocab{оператором взяття
внутрішності множини $X$}, якщо воно задовольняє
наступні умови:
\begin{enumerate}
\item[К1.] $\Interior(M \cap N) = \Interior(M)\cap \Interior(N)$ (адитивність);
\item[К2.] $\Interior(M) \subset M$;
\item[К3.] $\Interior(\Interior(M)) = \Interior(M)$ (ідемпотентність);
\item[K4.] $\Interior(\emptyset) = \emptyset$.
\end{enumerate}
\end{definition}

\begin{corollary}
Оскільки \[\Interior A = X \setminus \closure{X \setminus A}, \]
оператор взяття внутрішності є двоїстим для оператора
замикання Куратовського. Отже, система множин
$\tau = \{A \subseteq X: \Interior A = A\}$ утворює в $X$ топологію, а множина
$\Interior A$ в цій топології є внутрішністю множини $A$.
\end{corollary}

\section{База топології}

Для завдання в множині $X$ певної топології немає
потреби безпосередньо указувати всі відкриті підмножини
цієї топології. Існує деяка сукупність відкритих підмножин,
яка повністю визначає топологію. Така сукупність
називається базою цієї топології.

\begin{definition}
Сукупність $\beta$ відкритих множин простору
$(X, \tau)$ називається \vocab{базою топології} $\tau$ або \vocab{базою простору}
$(X, \tau)$, якщо довільна непорожня відкрита множина цього
простору є об'єднанням деякої сукупності множин, що
належать $\beta$: \[ \forall G \in \tau, G \ne \emptyset \quad \exists B_\alpha \in \beta, \alpha \in A: \quad G = \bigcup_{\alpha \in A} B_\alpha. \]
\end{definition}

\begin{remark}
Будь-який простір $(X, \tau)$ має базу,
оскільки система всіх відкритих підмножин цього простору
утворює базу його топології.
\end{remark}

\begin{remark}
Якщо в просторі $(X, \tau)$ існують
ізольовані точки, вони повинні входити в склад будь-якої
бази цього простору.
\end{remark}

\begin{theorem}
Для того щоб сукупність $\beta$ множин із
топології $\tau$ була базою цієї топології, необхідно і
достатньо, щоб для кожної точки $x \in X$ і довільної
відкритої множини $U$, що містить точку $x$, існувала
множина $V \in \beta$, така щоб $x \in V \subset U$.
\end{theorem}

\begin{proof}
Необхідність. Нехай $\beta$ --- база простору
$(X, \tau)$, $x_0 \in X$, а $U_0 \in \tau$, таке що $x_0 \in U_0$. Тоді за означенням бази $U_0 = \bigcup_{\alpha \in A} V_\alpha$, де $V_\alpha \in \beta$. З цього випливає, що $x_0 \in V_{\alpha_0} \subset U_0$.
\begin{equation*}
\beta = \mathcal{B}(\tau), x_0 \in X, U_0 \in \tau, x_0 \in U_0 \implies U_0 = \bigcup_{\alpha \in A} V_\alpha, V_\alpha \in \beta \implies x_0 \in V_{\alpha_0} \subset U_0.
\end{equation*}

Достатність. Нехай для кожної точки $x \in X$ і довільної
відкритої множини $U \in \tau$, що містить точку $x$, існує множина
$V_x \in \beta$, така що $x \in V_x \subset U$. Легко перевірити, що $U = \bigcup_{x \in U} V_x$.

Дійсно, якщо точка $x \in U$, то за умовою теореми, вона
належить множині $V_x \subset U$, а отже й об'єднанню таких
множин $\bigcup_{x \in U} V_x$: \[ x \in U \implies \exists V_x \subset U: x \in V_x \implies x \in \bigcup_{x \in U} V_x. \]

І навпаки, якщо точка належить об'єднанню $\bigcup_{x \in U} V_x$, то
вона належить принаймні одній із цих множин $V_x \subset U$, а
отже --- вона належить множині $U$: \[ x \in \bigcup_{x \in U} V_x \implies \exists V_x \subset U: x \in V_x \implies x \in U. \]

Таким чином, довільну відкриту множину $U \in \tau$ можна
подати у вигляді об'єд\-нання множин із $\beta$.
\end{proof}

\begin{example}
Оскільки $\forall x \in \RR^1$ і $\forall (a, b) \ni x$ $\exists(a_0, b_0) \subset (a, b)$,
то за попередньою теоремою сукупність всіх відкритих інтервалів
утворює базу топології в $\RR^1$.
\end{example}

\begin{example}
Оскільки $\forall x \in \RR^1$ і $\forall (a, b) \ni x$ $\exists (r_1, r_2) \subset (a, b)$, $r_1, r_2 \in \QQ$, то за попередньою теоремою сукупність всіх відкритих
інтервалів із раціональними кінцями також утворює базу
топології в $\RR^1$.
\end{example}

З цієї теореми випливають два наслідки.

\begin{corollary}
Об'єднання всіх множин, які належать
базі $\beta$ топології $\tau$, утворює всю множину $X$.
\end{corollary}

\begin{proof}
Оскільки $X \in \tau$, то за означенням бази
$X = \bigcup_{\alpha \in A} V_\alpha$, де $V_\alpha \in \beta$.
\end{proof}

Надалі будемо також називати цей наслідок першою властивістю бази топології.

\begin{corollary}
Для довільних двох множин $U$ і $V$ із бази
$\beta$ і для кожної точки $x \in U \cap V$ існує множина $W$ із $\beta$ така,
що $x \in W \subset U \cap V$.
\end{corollary}

\begin{proof}
Оскільки $U \cap V \in \tau$, то за попередньою теоремою в
множині $U \cap V$ міститься відкрита множина $W$ із бази, така
що $x \in W$.
\end{proof}

Надалі будемо також називати цей наслідок другою властивістю бази топології.

\begin{theorem}[про завдання топології за допомогою бази]
Нехай в довільній множині $X$ задана деяка сукупність
відкритих множин $\beta$, що має властивості бази топології. Тоді в
множині $X$ існує єдина топологія $\tau$, однією з баз якої є
сукупність $\beta$.
\end{theorem}

\begin{proof}
Припустимо, що $\tau$ --- сімейство, що містить
лише порожню множину і всі підмножини множини $X$,
кожна з якиx є об'єднанням підмножин із сукупності $\beta$: \[ \tau = \Bigg\{ \emptyset, G_\alpha \subset X, \alpha \in A, G_\alpha = \bigcup_{i \in I} B_i^\alpha, B_i^\alpha \in \beta \Bigg\} . \]

Перевіримо, що це сімейство множин є топологією.
Виконання аксіом топології 1 і 2 є очевидним: $\emptyset \in \tau$, $X \in \tau$ і
\[ G_\alpha \in \tau, \alpha \in A \implies \cup_{\alpha \in A} G_\alpha \in \tau. \]

Аксіома 3 є наслідком властивостей. Не обмежуючи загальності, можна
перевірити її для випадку перетину двох множин.

Нехай $U, U' \in \tau$. За означенням, $U = \bigcup_{i \in I} V_i$ i $U' = \bigcup_{j \in J} V_j'$, де $V_i, V_j' \in \beta$. Розглянемо перетин \[ U \cap U' = \Bigg( \bigcup_{i \in I} V_i \Bigg) \cap \Bigg( \bigcup_{j \in J} V_i' \Bigg) = \bigcup_{i \in I, j \in J} (V_i \cap V_j'). \]

Доведемо, що $V_i \cap V_j' \in \tau$. Нехай $x \in V_i \cap V_j'$. Тоді, за
другою властивістю, існує множина $W_x \in \beta$, така що $x \in W_x \subset V_i \cap V_j'$. Оскільки точка $x \in V_i \cap V_j'$ є довільною, то $V_i \cap V_j' = \bigcup_{x \in V_i \cap V_j'} W_x \in \tau$. Отже, $U \cap U' \in \tau$.

Таким чином, сімейство $\tau$ дійсно утворює топологію на
$X$, а система $\beta$ є її базою.
\end{proof}

\section{Література}

\begin{enumerate}[label={[\arabic*]}]
\item \textbf{Александрян~Р.А.}
Общая топология /
Р.~А.~Александрян, Э.~А.~Мирзаханян ---
М.: Высшая школа, 1979 (стр.~14--22).
\item \textbf{Энгелькинг~Р.}
Общая топология /
Р.~Энгелькинг ---
М.: Мир, 1986 (стр.~46--50).
\end{enumerate}