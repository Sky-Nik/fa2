\chapter{Компактність в топологічних просторах}

Велику роль в топології відіграє клас компактних
просторів, які мають дуже важливі властивості. Введемо
основні поняття.

\section{Покриття і підпокриття}

\begin{definition}
Система множин $S = \{A_i \subset X, i \in I\}$ називається
\vocab{покриттям} простору $X$, якщо $\Bigcup_{i \in I} A_i = X$.
\end{definition}

\begin{definition}
Покриття $S$ називається \vocab{відкритим}
(\vocab{замкненим}), якщо кожна із множин $A_i$ є відкритою
(замкненою).
\end{definition}

\begin{definition}
Підсистема $P$ покриття $S$ простору $X$
називається \vocab{підпокриттям} покриття $S$, якщо сама $P$
утворює покриття $X$.
\end{definition}

\begin{theorem}
[Ліндельоф] Якщо простір $X$ має злічену
базу, то із його довільного відкритого покриття можна
виділити не більш ніж злічене підпокриття.
\end{theorem}

\begin{proof}
Нехай $\beta = \{U_n\}$ --- деяка злічена база
простору $X$, а $S = \{G_i, i \in I\}$ --- довільне відкрите покриття
простору $X$. Для кожного $x \in X$ позначимо через $G_n(x)$ один
з елементів покриття $S$, що містить точку $x$, а через $U_n(x)$ ---
один з елементів бази $\beta$, що містить точку $x$ і цілком
міститься у відкритій множині $G_n$ (теорема 2.3).
\begin{equation*}
x \in U_n(x) \subset G_n(x).
\end{equation*}

Відібрані нами множини $U_n(x) \in \beta$ утворюють злічену
множину. Крім того, кожна точка $x$ простору $X$ міститься в
деякій множині $U_n(x)$, отже
\begin{equation*}
\Bigcup_{x \in X} U_n(x) = X.
\end{equation*}

Вибираючи для кожного $U_n(x)$ відкриту множину $G_n(x)$,
ми отримаємо не більш ніж злічену систему, яка є
підпокриттям покриття $S$.
\end{proof}

\begin{definition}
Топологічний простір $(X, \tau)$, в якому із
довільного відкритого покриття можна виділити не більш
ніж злічене підпокриття, називається \vocab{ліндельофовим}, або
\vocab{фінально компактним}.
\end{definition}

\section{Компактні простори}

Звузимо клас ліндельофових просторів і введемо
наступне поняття.

\begin{definition}
Топологічний простір $(X, \tau)$ називається
\vocab{компактним (бікомпактним)}, якщо будь-яке його
відкрите покриття містить скінченне підпокриття (умова
Бореля---Лебега).
\end{definition}

\begin{example}
Простір з тривіальною топологією є компактним.
\end{example}

\begin{example}
Простір з дискретною топологією є
компактним тоді й лише тоді, коли він складається зі
скінченної кількості точок.
\end{example}

\begin{example}
Простір Зариського є компактним.
\end{example}

\begin{example}
Простір $\RR^n$, $n \ge 1$ не є компактним.
\end{example}

\begin{theorem}
[перший критерій компактності] Для
компактності топологічного простору $(X, \tau)$ необхідно і
достатньо, щоб будь-яка сукупність його замкнених
підмножин з порожнім перетином містила скінченну
підмножину таких множин із порожнім перетином.
\begin{multline*}
(X, \tau) \text{ --- компактний } \iff \\
\forall \Bigg\{ \closure F_\alpha, \alpha \in A:
\Bigcap_{\alpha \in A} \closure F_\alpha = \emptyset \Bigg\} \quad
\exists \Bigg\{ \closure F_{\alpha_1}, \closure F_{\alpha_2}, \dots, \closure F_{\alpha_n} \Bigg\}: \quad
\Bigcap_{i = 1}^n \closure F_{\alpha_i} = \emptyset.
\end{multline*}
\end{theorem}

\begin{proof}
Необхідність. Нехай $(X, \tau)$ --- компактний,
а $\{\closure F_\alpha, \alpha \in A\}$ --- довільна сукупність замкнених множин, що
задовольняє умові $\Bigcap_{\alpha \in A} \closure F_\alpha = \emptyset$. Розглянемо множини
$U_\alpha = X \setminus F_\alpha$. За правилами де Моргана (принцип двоїстості)
сукупність множин $\{U_\alpha, \alpha \in A\}$ задовольняє умові
$\Bigcup_{\alpha \in A} U_\alpha = X$, тобто утворює покриття простору $(X, \tau)$.
Оскільки, за припущенням, $(X, \tau)$ --- компактний простір,
то існує скінченна підмножина множин $\{U_{\alpha_1}, U_{\alpha_2}, \dots, U_{\alpha_n}\}$,
які також утворюють покриття: $\Bigcup_{i = 1}^n U_{\alpha_i} = X$. Отже, за
правилами де Моргана
\begin{equation*}
X \setminus\Bigcap_{i = 1}^n \closure F_{\alpha_i} =
\Bigcup_{i = 1}^n (X \setminus \closure F_{\alpha_i}) =
\Bigcup_{i = 1}^n U_{\alpha_i} = X \implies
\Bigcap_{i = 1}^n \closure F_{\alpha_i} = \emptyset.
\end{equation*}

Достатність. Нехай $\{U_\alpha, \alpha \in A\}$ --- довільне відкрите
покриття простору $(X, \tau)$. Очевидно, що множини
$\closure F_\alpha = X \setminus U_\alpha$, $\alpha \in A$ є замкненими,
а їх сукупність має порожній перетин:
$\Bigcap_{\alpha \in A} \closure F_\alpha = \emptyset$.
За умовою, ця сукупність містить скінченну підмножину множин
$\{ \closure F_{\alpha_1}, \closure F_{\alpha_2}, \dots, \closure F_{\alpha_n} \}$,
таку що $\Bigcap_{i = 1}^n \closure F_{\alpha_i} = \emptyset$.
Звідси випливає, що множини $U_{\alpha_n}$, які є
доповненнями множин $\closure F_{\alpha_n}$, утворюють покриття простору
$(X, \tau)$, тобто простір $(X, \tau)$ є компактним.
\end{proof}

\begin{definition}
Система підмножин $\{M_\alpha \subset X, \alpha \in A\}$
називається \vocab{центрованою}, якщо перетин довільної
скінченної кількості цих підмножин є непорожнім.
\begin{equation*}
\forall \{\alpha_1, \alpha_2, \dots, \alpha_n\} \in A
\Bigcap_{i = 1}^n M_{\alpha_i} \ne \emptyset \implies
\{M_\alpha \subset X, \alpha \in A\} \text{ --- центрована система}.
\end{equation*}
\end{definition}

\begin{theorem}
[другий критерій компактності] Для
компактності топологічного простору $(X, \tau)$ необхідно і
достатньо, щоб будь-яка центрована система його
замкнених підмножин мала непорожній перетин
\end{theorem}

\begin{proof}
Необхідність. Нехай простір $(X, \tau)$ ---
компактний, а $\{F_\alpha\}$ --- довільна центрована система
замкнених підмножин. Множини $G_\alpha = X \setminus F_\alpha$ відкриті.
Жодна скінченна система цих множин $G_{\alpha_n}$, $n \in \NN$ не
покриває $X$, оскільки
\begin{equation*}
\forall n \in \NN \Bigcap_{i = 1}^n F_{\alpha_i} \ne \emptyset \implies
X \setminus \Bigcap_{i = 1}^n F_{\alpha_i} =
\Bigcup_{i = 1}^n G_{\alpha_i} \ne X \setminus \emptyset = X.
\end{equation*}

Отже, оскільки $(X, \tau)$ --- компактний простір, система
$\{G_\alpha\}$ не може бути покриттям компактного простору.
Інакше ми могли б вибрати із системи $\{G_\alpha\}$ скінченне
підпокриття $\{G_{\alpha_1}, G_{\alpha_2}, \dots, G_{\alpha_n}\}$,
а це означало б, що $\Bigcap_{i = 1}^n F_{\alpha_i} = \emptyset$.
Але, якщо $\{G_\alpha\}$ --- не покриття, то $\Bigcap_\alpha F_\alpha \ne \emptyset$.

Достатність. Припустимо, що довільна центрована
система замкнених множин із $X$ має непорожній перетин.
Нехай $\{G_\alpha\}$ --- відкрите покриття $(X, \tau)$. Розглянемо
множини $F_\alpha = X \setminus G_\alpha$. Тоді
\begin{equation*}
\Bigcup_\alpha G_\alpha = X \implies
X \setminus \Bigcup_\alpha G_\alpha = X \setminus X = \emptyset \implies
\Bigcap_\alpha (X \setminus G_\alpha) =
\Bigcap_\alpha F_\alpha = \emptyset.
\end{equation*}

Це означає, що система $\{F_\alpha\}$ не є центрованою, тобто
існують такі множини $F_1, F_2, \dots, F_N$, що
\begin{equation*}
\Bigcap_{i = 1}^N F_i = \emptyset \implies
X \setminus \Bigcap_{i = 1}^N F_i = X \setminus \emptyset \implies
\Bigcup_{i = 1}^N G_i = X.
\end{equation*}

Отже, із покриття $\{G_\alpha\}$ ми виділили скінчену підсистему
\begin{equation*}
\{G_1, \dots, G_N\} = \{X \setminus F_1, \dots, X \setminus F_N\}
\end{equation*}
таку що $\Bigcup_{i = 1}^N G_i = X$. Це означає, що простір $(X, \tau)$ є
компактним.
\end{proof}

\section{Види компактності}

\begin{definition}
Множина $M \subset X$ називається \vocab{компактною
(бікомпактною)}, якщо топологічний підпростір $(M, \tau_M)$,
що породжується індукованою топологією, є компактним.
\end{definition}

\begin{definition}
Множина $M \subset X$ називається \vocab{відносно
компактною (відносно бікомпактною)}, якщо її замикання
$\closure M$ є компактною множиною.
\end{definition}

\begin{definition}
Компактний і хаусдорфів простір називається
\vocab{компактом (бікомпактом)}.
\end{definition}

\begin{definition}
Топологічний простір називається
% Топологічний простір $(X, \tau)$ називається
\vocab{зліченно компактним}, якщо із його довільного зліченного
відкритого покриття можна виділити скінченне
підпокриття (умова Бореля).
\end{definition}

\begin{definition}
Топологічний простір називається
% Топологічний простір $(X, \tau)$ називається
\vocab{секвенційно компактним}, якщо довільна нескінченна
послідовність його елементів містить збіжну
підпослідовність (умова Больцано-Вейєрштрасса).
\end{definition}

\section{Зв'язки між видами компактності}

\begin{theorem}
[перший критерій зліченної компактності]
Для того щоб простір $(X, \tau)$ був зліченно
компактним необхідно і достатньо, щоб кожна його
нескінченна підмножина мала принаймні одну строгу
граничну точку, тобто точку, в довільному околі якої
міститься нескінченна кількість точок підмножини.
\end{theorem}

\begin{proof}
Необхідність. Нехай $(X, \tau)$ --- зліченно
компактний простір, а $M$ --- довільна нескінченна множина
в $X$. Припустимо, усупереч твердженню, що $M$ не має
жодної строгої граничної точки. Розглянемо послідовність
замкнених множин $\Phi_n \subset M$, таку що $\Phi_n \subset \Phi_{n + 1}$.
Візьмемо $x_n \in \Phi_n$.
За припущенням нескінченна послідовність точок
$x_1, x_2, \dots, x_n, \dots$ не має строгих граничних точок. Побудуємо
скінченну систему підмножин $\{F_n, n \in \NN\}$, поклавши
$F_n = \{x_n, x_{n+1}, \dots\}$. Зі структури цих множин випливає, що
будь-яка скінченна сукупність точок $F_n$ має непорожній
перетин, всі множини $F_n$ є замкненими,
але $\Bigcap_{n \in \NN} F_n = \emptyset$.
Отже, ми побудували зліченну центровану систему
замкнених множин, перетин яких порожній, що суперечить
припущенню, що простір $(X, \tau)$ зліченно компактним.

Достатність. Нехай в просторі $(X, \tau)$ кожна
нескінченна множина $M$ має строгу граничну точку.
Доведемо, що простір $(X, \tau)$ є зліченно компактним. Для
цього достатньо перевірити, що будь-яка зліченна
центрована система $\{F_n\}$ замкнених множин має
непорожній перетин. Побудуємо множини
$\hat F_n = \Bigcap_{i = 1}^n F_i$.
Оскільки система $\{F_n\}$ є центрованою, то замкнені
непорожні множини $\hat F_n$ утворюють послідовність
$\hat F_1, \hat F_2, \dots, \hat F_n, \dots$, що не зростає.
Очевидно, що $\Bigcap_{n \in \NN} F_n = \Bigcap_{n \in \NN} \hat F_n$.
Можливі два варіанти: серед множин $\hat F_n$ є лише скінченна
кількість попарно різних множин, або нескінченна кількість
таких множин. Розглянемо ці варіанти окремо.
\begin{enumerate}
\item Якщо серед множин $\hat F_n$ є лише скінченна кількість
попарно різних множин, то починаючи з деякого номера $n_0$
виконується умова $\hat F_{n_0} = \hat F_{n_0 + 1} = \dots$.
Тоді твердження доведено, оскільки $\Bigcap_{n \in \NN} \hat F_n = \hat F_{n_0} \ne \emptyset$.

\item Якщо серед множин $\hat F_n$
є лише нескінченна кількість
попарно різних множин, то можна вважати, що
$\hat F_n \setminus \hat F_{n + 1} \ne \emptyset$.
Оберемо по одній точці з кожної множини
$\hat F_n \setminus \hat F_{n + 1}$.
Отже, ми побудували нескінченну множину різних
точок, яка, за умовою, має граничну точку $x^\star$. Всі точки
$x_n, x_{n + 1}, \dots$ належать множинам $\hat F_n$. Отже,
$x^\star \in \hat F_n' \forall n \in \NN$,
до того ж $\closure{\hat{F_n}} = \hat F_n$.
З цього випливає, що $\Bigcap_{n \in \NN} \hat F_n \ne \emptyset$. \qedhere
\end{enumerate}
\end{proof}

\begin{remark}
Вимогу наявності строгої граничної
точки можна замінити аксіомою $T_1$. Інакше кажучи, в
досяжних просторах будь-яка гранична точка є строгою.
Припустимо, що $X$ --- досяжний простір, а гранична точка
$x$ множини $A$ не є строгою, і тому існує деякий окіл $U$, що
містить лише скінчену кількість точок множини $A$, що
відрізняються від $x$. Розглянемо множину
$V = U \setminus ((A \cap U) \setminus \{x\})$,
тобто різницю між множиною $U$ і
цим скінченним перетином. Оскільки простір $X$ є
досяжним, то в ньому будь-яка скінченна множина є
замкненою. Отже, множина V є відкритою
$(V = X \cap (U \setminus \{A \cap U \setminus \{x\}\}) =
U \cap (X \setminus (U \cap A \setminus \{x\}))$, містить
точку $x$, а перетин множин дорівнює $A \cap V = \{x\}$ або $\emptyset$. Це
суперечить тому, що $x$ --- гранична точка множини $A$.
\end{remark}

\begin{remark}
Чому не можна взагалі зняти умову
наявності строгої граничної точки? Розглянемо як
контрприклад топологію, що складається з натуральних
чисел на відрізку $[1, n]$, тобто
$\tau = \{\emptyset, \NN, [1, n] \cap \NN \forall n \in \NN\}$.
Цей простір не є зліченно компактним (порушується другий
критерій компактності). Розглянемо нескінченну множину
$A \subset \NN$ і покладемо $n = \min A$.
Тоді будь-який $m \in A \setminus \{n\}$ є
граничною точкою множини $A$, тобто $\NN$ є слабко зліченно
компактним простором.
\end{remark}

\begin{theorem}
[другий критерій зліченної компактності]
Для того щоб досяжний простір $(X, \tau)$ був
зліченно компактним необхідно і достатньо, щоб кожна
нескінченна множина точок із $X$ мала принаймні одну
граничну точку (такі простори називаються слабко
зліченно компактними). Інакше кажучи, в досяжних
просторах слабка зліченна компактність еквівалентна
зліченній компактності.
\end{theorem}

\begin{proof}
Необхідність. Припустимо, що $A$ ---
злічена підмножина $X$, що не має граничних точок (це не
обмежує загальності, оскільки в будь-якій нескінченій
підмножині ми можемо вибрати злічену підмножину).
Множина $A$ є замкненою в $X$ (оскільки будь-яка точка
множини $\closure A \setminus A$ є граничною точкою множини $A$, яка за
припущенням не має граничних точок, тому $\closure A = A$). Нехай
$A = \{a_1, s_2, \dots\}$ і $A_n = \{a_n, a_{n + 1}, \dots\}$.
Зі сказаного вище випливає,
що $A_n = \closure A_n$, інакше $A' = \emptyset$.
Покладемо $G_n = X \setminus A_n$. Ця
множина є доповненням замкненої множини $A_n$, тому вона
є відкритою. Розглянемо послідовність множин $G_n$. Вона
зростає і покриває $X$, тому що кожна точка $x$ із множини
$X \setminus A$ належить $G_1$, а значить, усім множинам $G_n$, а якщо
$x \in A$, то вона дорівнює якомусь $a_N$, отже, належить $G_{N + 1}$.
Таким чином, послідовність множин $G_n$ є покриттям, але
вона не може містити скінченне підпокриття
$\{G_{i_1}, G_{i_2}, \dots, G_{i_n}\}$,
оскільки об’єднання елементів цього
скінченного підпокриття було б найбільшим серед усіх
множин $G_n$ (які утворюють зростаючу послідовність).
\begin{equation*}
G_1 \subset G_2 \subset \dots \subset \Bigcup_{k = 1}^n G_{i_k} = G_N = X.
\end{equation*}
У цьому випадку об’єднання $G_N = \Bigcup_{k = 1}^n G_{i_k}$
не може містити
усі елементи $a_i$, номер яких перевищує $N$ (за
конструкцією), отже, воно не покриває $X$. У такому
випадку простір $X$ не є зліченно компактним. Отримана
суперечність доводить бажане.

Достатність. Припустимо, що простір $X$ не є зліченно
компактним. Значить, існує зліченне відкрите покриття
$\{G_n\}_{n \in \NN}$, що не містить скінченного підпокриття.
Жодна сукупність множин $\{G_1, G_2, \dots, G_n\}$ не є покриттям,
тому можемо вибрати з множин $X \setminus \Bigcup_{k = 1}^n G_i$
по одній точці $x_i$ і утворити із них множину $A$.

Розглянемо довільну точку $x \in X$. Оскільки $\{G_n\}_{n \in \NN}$ ---
покриття простору $X$, точка $x$ належить якійсь множині
$G_N$, яка своєю чергою може містити лише такі точки $x_i$ із
множини $A$, номер яких задовольняє умові $i < N$ (оскільки
за означенням точка $x_i$ не належить жодному $G_j$, якщо
$j \le i$). Отже, множина $G_N$ є околом точки $x$, перетин якої
із множиною $A$ є лише скінченним. Водночас, оскільки
простір є досяжним, в околі граничної точки будь-якої
множини повинно міститись нескінченна кількість точок
цієї множини. Отже, точка $x$ не є граничною точкою
множини $A$. Це твердження є слушним для будь-якої точки
$x$, отже, множина $A$ не має жодної граничної точки.
Отримана суперечність доводить бажане.
\end{proof}

\begin{theorem}
[про еквівалентність компактності та зліченої компактності]
Для топологічного простору
$(X, \tau)$ зі зліченною базою компактність еквівалентна
зліченній компактності.
\end{theorem}

\begin{proof}
Необхідність. Нехай $(X, \tau)$ --- компактний
простір. Тоді із довільного відкритого покриття можна
виділити скінченне покриття. Значить, скінченне покриття
можна виділити зі зліченного відкритого покриття.

Достатність. Нехай $(X, \tau)$ є зліченно компактним
простором, а $S = \{U_\alpha, \alpha \in A\}$ --- його довільне відкрите
покриття. Оскільки простори зі зліченою базою мають
властивість Ліндельофа (теорема 5.1), то покриття $S$
містить підпокриття $S'$, яке, внаслідок, зліченної
компактності простору $(X, \tau)$ містить скінченне
підпокриття $S''$. Отже, простір $(X, \tau)$ є зліченно
компактним.
\end{proof}

\begin{theorem}
[про еквівалентність компактності,
секвенційної компактності та зліченної
компактності]
Для досяжних просторів зі зліченою базою
компактність, секвенційна компактність і зліченна
компактність є еквівалентними.
\end{theorem}

\begin{proof}
З огляду на теорему 5.6, достатньо показати,
що злічена компактність в досяжному просторі зі зліченною
базою еквівалентна секвенційній компактності.

Необхідність. Розглянемо зліченно компактний простір
$(X, \tau)$. Нехай $A = \{x_n\}_{n \in \NN}$ --- довільна нескінченна
послідовність (тобто послідовність, що містить нескінченну
кількість різних точок), а простір є зліченно компактним.
Отже, за теоремою 5.5, множина $A$ має граничну точку $x^\star$.
Розглянувши зліченну локальну базу околів $\{G_k\}_{k \in \NN}$ точки
$x^\star$, так що $G_{k + 1} \subset G_k$, можна виділити послідовність
$x_{n_k}$, що
збігається до $x^\star$. Отже, простір $(X, \tau)$ є секвенційно
компактним.

Достатність. Нехай простір $(X, \tau)$ є секвенційно
компактним. З теореми 5.4 випливає, що будь-яка зліченна
нескінченна підмножина простору X має строгу граничну
точку. Це означає, що будь-яка нескінченна зліченна
послідовність має граничну точку, тобто із неї можна
виділити збіжну підпослідовність.
\end{proof}

\section{Література}

\begin{enumerate}[label={[\arabic*]}]
\item \textbf{Александрян~Р.А.}
Общая топология /
Р.~А.~Александрян, Э.~А.~Мирзаханян ---
М.: Высшая школа, 1979 (стр.~225--238).
\item \textbf{Колмогоров~А.Н.}
Элементы теории функций и функционального анализа. 5-е изд. /
Колмогоров~А.Н., С.~В.~Фомин ---
М.: Наука, 1981 (стр.~98--105).
\item \textbf{Энгелькинг~Р.}
Общая топология /
Р.~Энгелькинг ---
М.: Мир, 1986 (стр.~195--215).
\end{enumerate}