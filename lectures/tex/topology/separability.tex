\chapter{Аксіоми віддільності}

Аналізуючи властивості різних топологічних просторів
ми бачили, що їх структура може бути настільки
''неприродною'', що будь-яка послідовність збігається до
будь-яких точок (тривіальний простір), існують точки
дотику множин, які не є границями послідовностей їх
елементів (простір Зариського) тощо. В математичному
аналізі ми не зустрічаємо таких ''патологій'': там всі
послідовності мають лише одну границю, кожна точка
дотику є границею тощо. Отже, виникає потреба в
інструментах, які дозволили б виділити серед
топологічних просторів ''природні'' простори. Такими
інструментами є аксіоми віддільності, які разом з аксіомами
зліченності дають можливість повністю описати
властивості топологічних просторів.

\section{Власне аксіоми}

Аксіоми віддільності в топологічному просторі $(X, \tau)$
формулюються наступним чином.

\begin{itemize}
\item $T_0$ (Колмогоров, 1935). Для двох довільних різних точок
$x$ і $y$, що належать множині $X$, існує множина із
топологічної структури $\tau$, яка містить рівно одну з цих
точок.
\[ (\forall x \ne y \in X): ((\exists V_x \in \tau: x \in V_x, y \notin V_x) \lor (\exists V_y \in \tau: y \in V_y, x \notin V_y)). \]

\item $T_1$ (Рісс, 1907). Для двох довільних різних точок $x$ і $y$,
що належать множині $X$, існують множина $V_x$ із
топологічної структури $\tau$, яка містить точку $x$ і не
містить точки $y$, і множина $V_y$ із топологічної
структури $\tau$, яка містить точку $y$ і не містить точки $x$.
\[ (\forall x \ne y \in X): ((\exists V_x \in \tau: x \in V_x, y \notin V_x) \land (\exists V_y \in \tau: y \in V_y, x \notin V_y)). \]

\item $T_2$ (Хаусдорф, 1914). Для двох довільних різних точок $x$ і
$y$, що належать множині $X$, існують множина $V_x$ із
топологічної структури $\tau$, яка містить точку $x$, і
множина $V_y$ із топологічної структури $\tau$, яка містить
точку $y$, такі що не перетинаються.
\[ (\forall x \ne y \in X): (\exists V_x \sqcup V_y \in \tau): (x \in V_x \land y \in V_y). \]

\item $T_3$ (В'єторіс, 1921). Для довільної точки $x$ і довільної
замкненої множини $F$, що не містить цієї точки, існують
дві відкриті множини $V_x$ і $V$, що не перетинаються, такі
що $x \in V_x$, а $F \subset V$.
\[ (\forall x \in X, \closure{F} \subset X): (\exists V_x \sqcup V \in \tau): (x \in V_x \land \closure{F} \subset V). \]

\item $T_{3\frac{1}{2}}$ (Урисон, 1925). Для довільної точки $x$ і довільної
замкненої множини $\closure{F}$, що не містить цієї точки, існує
неперервна числова функція $f$, задана на просторі $X$,
така що $0 \le f(t) \le 1$, до того ж $f(x) = 0$ і $f(t) = 1$, якщо
$x \in \closure{F}$.
\begin{equation*}
(\forall x \in X, \closure{F} \subset X: x \notin \closure{F}):
(\exists f: X \to \RR^1: 0 \le f(t) \le 1, f(x) = 0, f(\closure{F}) = 1).
\end{equation*}

\item $T_4$ (В'єторіс, 1921). Для двох довільних замкнених
множин $\closure{F_1}$ і $\closure{F_2}$ що не перетинаються, існують відкриті
множини $G_1$ і $G_2$, що не перетинаються, такі що $\closure{F_1} \subset G_1$, $\closure{F_2} \subset G_2$.
\begin{equation*}
(\forall \closure{F_1}, \closure{F_2} \subset X: \closure{F_1} \cap \closure{F_2} = \emptyset):
(\exists G_1, G_2 \in \tau: \closure{F_1} \subset G_1, \closure{F_2} \subset G_2, G_1 \cap G_2 = \emptyset).
\end{equation*}
\end{itemize}

\begin{definition}[Колмогоров, 1935]
Топологічні простори, що
задовольняють аксіому $T_0$, називаються \vocab{$T_0$-просторами},
або \vocab{колмогоровськими}.
\end{definition}

\begin{definition}[Рісс, 1907]
Топологічні простори, що
задовольняють аксіому $T_1$, називаються \vocab{$T_1$-просторами},
або \vocab{досяжними}.
\end{definition}

\begin{definition}[Хаусдорф, 1914]
Топологічні простори, що
задовольняють аксіому $T_2$, називаються \vocab{хаусдорфовими},
або \vocab{віддільними}.
\end{definition}

\begin{definition}[В'єторіс, 1921]
Топологічні простори, що
задовольняють аксіоми $T_1$ і $T_3$, називаються \vocab{регулярними}.
\end{definition}

\begin{definition}[Тихонов, 1930]
Топологічні простори, що
задовольняють аксіоми $T_1$ і $T_{3\frac{1}{2}}$, називаються \vocab{цілком
регулярними}, або \vocab{тихоновськими}.
\end{definition}

\begin{definition}[Тітце (1923), Александров і Урисон (1929)]
Топологічні простори, що задовольняють аксіоми
$T_1$ і $T_4$, називаються \vocab{нормальними}.
\end{definition}

\section{Наслідки з аксіом}

Розглянемо наслідки, які випливають з аксіом
віддільності.

\begin{theorem}[критерій досяжності]
Для того щоб
топологічний простір $(X, \tau)$ був $T_1$-простором необхідно і
достатньо, щоб будь-яка одноточкова множина $\{x\} \subset X$
була замкненою.
\end{theorem}

\begin{proof}
Необхідність. Припустимо, що виконується
перша аксіома віддільності: якщо $x \ne y$, то існує окіл
$V_x \in \tau: x \notin V_y$. Тоді $\forall y \ne x$, $y \notin \closure{\{x\}}$,
тобто $\closure{\{x\}} = \{x\}$.

Достатність. Припустимо, що $\closure{\{x\}} = \{x\}$. Тоді
$\forall y \ne x$: $\exists V_y \in \tau$: $x \notin V_y$. Отже, виконується перша аксіома
віддільності.
\end{proof}

\begin{corollary}
В просторі $T_1$ будь-яка скінченна множина є
замкненою.
\end{corollary}

\begin{theorem}
Для того щоб точка $x$ була граничною
точкою множини $M$ в $T_1$-просторі необхідно і достатньо,
щоб довільний окіл $U$ цієї точки містив нескінченну
кількість точок множини $M$.
\end{theorem}

\begin{proof}
Необхідність. Якщо точка $x$ є граничною
точкою множини $M$, то \[ \forall O(x) \in \tau: O(x) \cap M \setminus \{x\} \ne \emptyset. \]

Припустимо, що існує такий окіл $U$ точки $x$, що містить
лише скінченну кількість точок $x_1, x_2, \ldots, x_n \in M$. Оскільки
простір $(X, \tau)$ є $T_1$-простором, то існує окіл i $U$ точки $x$, що
не містить точку $x_i$.

Введемо в розгляд множину $V = \bigcap_{i = 1}^n U_i$.
Ця множина є околом точки $x$, що не містить точок
множини $M$, за винятком, можливо, самої точки $x$. Отже,
точка $x$ не є граничною точкою множини $M$, що
суперечить припущенню.

Достатність. Якщо довільний окіл $U$ точки $x$ містить
нескінченну кількість точок множини $M$, то вона є
граничною за означенням.
\end{proof}

\begin{example}
Зв'язна двокрапка є колмогоровским, але
недосяжним простором.
\end{example}

\begin{example}
Простір Зариського є досяжним, але не
хаусдорфовим.
\end{example}

\begin{theorem}[критерій хаусдорфовості]
Для того щоб
простір $(X, \tau)$ був хаусдорфовим необхідно і достатньо,
щоб для кожної пари різних точок $x_1$ і $x_2$ в $X$ існувало
неперервне ін'єктивне відображення $f$ простору $X$ в
хаусдорфів простір $Y$.
\end{theorem}

\begin{proof}
Необхідність. Нехай простір $(X, \tau)$ є
хаусдорфовим. Тоді можна покласти $Y = X$ і $f = I$ ---
тотожне відображення.

Достатність. Нехай $(X, \tau)$ --- топологічний простір і
\[ (\exists O(f(x_1)) \in \tau_Y, O(f(x_2)) \in \tau_Y): (O(f(x_1)) \cap O(f(x_2)) = \emptyset), \]
де $Y$ --- хаусдорфів, а
$f$ --- неперервне відображення. Оскільки простір $Y$ є
хаусдорфовим, то
\[ (\exists O(f(x_1)), O(f(x_2)) \in \tau_Y): (O(f(x_1)) \cap O(f(x_2)) = \emptyset). \]

Оскільки відображення $f$ є неперервним, то
\[ (\exists O(x_1) \in \tau_X, O(x_2) \in \tau_Y): (f(O(x_1)) \subset O(f(x_1)) \land f(O(x_2)) \subset O(f(x_2))). \]

Тоді околи $V(x_1) = f\inv(f(O(x_1)))$ і $V(x_2) = f\inv(f(O(x_2)))$
не перетинаються.
\end{proof}

\begin{definition}
Замкнена множина, що містить точку $x$
разом з деяким її околом, називається \vocab{замкненим околом}
точки $x$.
\end{definition}

\begin{theorem}[критерій регулярності]
Для того щоб $T_1$-простір $(X, \tau)$ був регулярним необхідно і достатньо,
щоб довільний окіл $U$ довільної точки $x$ містив її
замкнений окіл.
\end{theorem}

\begin{proof}
Необхідність. Нехай простір $(X, \tau)$ є
регулярним, $x$ --- його довільна точка, а $U$ --- її довільний
окіл. Покладемо $F = X \setminus U$. Тоді внаслідок регулярності
простору $(X, \tau)$ існує окіл $V$ точки $x$ і окіл $W$ множини
$F$, такі що $V \cap W = \emptyset$. Звідси випливає, що $V \subset X \setminus W$,
отже, $\closure{V} = \closure{X \setminus W} = X \setminus W \subset X \setminus F = U$.

Достатність. Нехай довільний окіл довільної точки $x$
містить замкнений окіл цієї точки, а $F$ --- довільна замкнена
множина, що не містить точку $x$. Покладемо $G = X \setminus F \in \tau$.
Нехай $V$ --- замкнений окіл точки $x$, що міститься в
множині $G$. Тоді $W = X \setminus V$ є околом множини $F$, який не
перетинається з множиною $V$.
\end{proof}

\begin{example}
Розглянемо множину $X = \RR$ і введемо
топологію так: замкненими будемо вважати всі множини,
що є замкненими у природній топології числової прямої, а
також множину $A = \left\{ \frac1n, n = 1, 2, \ldots \right\}$.
Точка нуль їй не належить, але будь-які околи точки нуль і довільні околи
множини $A$ перетинаються. Це означає, що побудований
простір не є регулярним, але є хаусдорфовим.
\end{example}

\section{Замкнені бази та функціональна віддільність}

\begin{definition}
Система $\gamma = \{ A_i, i \in I\}$ замкнених підмножин
простору $X$ називається його \vocab{замкненою базою}, якщо будь-яку замкнену
в $X$ множину можна подати у вигляді перетину множин із системи $\gamma$.
\end{definition}

\begin{definition}
Система $\delta = \{B_j\}$ замкнених підмножин $B_j$ називається \vocab{замкненою
передбазою}, якщо будь-яку замкнену в $X$ множину можна
подати у вигляді перетину скінченних об'єднань множин із
системи $\delta$.
\end{definition}

\begin{definition}
Підмножини $A$ і $B$ простору $X$ називаються
\vocab{функціонально віддільними}, якщо існує дійсна неперервна
функція $f: X \to [0, 1]$ така, що \[ f(x) = \begin{cases} 0, & x \in A \\ 1, & x \in B. \end{cases} \]
\end{definition}

Оскільки замкнені бази і передбази є двоїстими до
відкритих, мають місце наступні твердження.

\begin{lemma}
Для того щоб система $\gamma = \{A_i, i \in I\}$
замкнених множин із $X$ була замкненою базою в $X$,
необхідно і достатньо, щоб для кожної точки $x_0 \in X$ і для
кожної замкненої множини $F_0$, що не містить точку $x_0$,
існувала множина $A_{j_0}$ така, що $x_0 \notin A_{j_0} \supset F_0$.
\end{lemma}

\begin{exercise}
Доведіть лему.
\end{exercise}

\begin{lemma}
Для того щоб система $\delta = \{B_j, j \in J\}$
замкнених множин із $X$ була замкненою передбазою в $X$,
необхідно і достатньо, щоб для кожної точки $x_0 \in X$ і для
кожної замкненої множини $F_0$, що не містить точку $x_0$,
існував скінченний набір елементів $B_{j_1}, B_{j_2}, \ldots, B_{j_n}$
такий, що $x_0 \notin \bigcup{k = 1}^n B_{j_k} \supset F_0$.
\end{lemma}

\begin{exercise}
Доведіть лему.
\end{exercise}

\begin{theorem}[критерій цілковитої регулярності]
Для того щоб $(X, \tau)$ був цілком регулярним (тихоновським)
необхідно і достатньо, щоб кожна його точка $x_0$ була
функціонально віддільною від усіх множин із деякої
замкненої передбази $\delta = \{F_i, i \in I\}$, що її не містять.
\end{theorem}

\begin{proof}
Необхідність. Якщо простір $(X, \tau)$ є
цілком регулярним (тихоновським), то точка $x_0$ є
функціонально віддільною від усіх замкнених множин, що її
не містять, а значить, і від усіх множин із деякої замкненої
передбази $\delta = \{F_i, i \in I\}$, що її не містять.

Достатність. Нехай $F_0$ --- довільна замкнена в $X$
множина, що не містить точку $x_0$, і нехай $F_{i_1}, F_{i_2}, \ldots, F_{i_n}$ ---
скінченний набір елементів із $\delta$ такий, що
$x_0 \notin \bigcup_{k = 1}^n F_{j_k} \supset F_0$ (за другою лемою).
За припущенням, існує неперервна функція $f_k: X \to [0, 1]$, яка здійснює
функціональну віддільність точки $x_0$ і замкненої множини
$F_{i_k}$.

Покладемо $f(x) = \sup_k f_k(x)$ і покажемо, що функція $f$
здійснює функціональну віддільність точки $x_0$ і множини
$F$, а тим більше, точки $x_0$ і множини $F_0 \subset F$.

Дійсно, $f(x_0) = \sup_k f_k(x_0) = 0$. Далі, оскільки
$\forall k = 1, 2, \ldots, n$: $f_k(x) \le 1$, із $x \in F$ випливає, що
$f(x) = \sup_k f_k(x) = 1$. Крім того, із того що
$x \in F = \bigcup_{k = 1}^n F_{i_k}$
випливає, що, $x \in F_{i_m}$, $1 \le m \le n$, тобто $f_m(x) = 1$.

Залишилося показати неперервність побудованої функції.
Для цього треба довести, що \[ (\forall x' \in X, \epsilon > 0): (\exists U \in \tau: X' \in U): (\forall x \in U): |f(x) - f(x')| < \epsilon. \]

Оскільки $f_k$ --- неперервна функція, то існує окіл $U_k$ точки $x'$, такий що
$\forall x \in U_k: |f_k(x) - f_k(x')| < \epsilon$.

Покладемо
$U = \bigcap_{k = 1}^n U_k$. Тоді для кожного $x \in U$ і
$\forall k = 1, 2, \ldots, n$ виконуються нерівності
\begin{align*}
f_k(x') - \epsilon < f_k(x) &\le \sup_k f_k(x) = f(x), \\
f_k(x) < f_k(x') + \epsilon &\le f_k(x') + \epsilon = f(x') + \epsilon.
\end{align*}

Звідси випливає, що $f(x' - \epsilon < f(x) < f(x') + \epsilon$.
\end{proof}

\begin{remark}
Побудова регулярних просторів, які не є
тихоновськими є нетривіальною задачею.
\end{remark}

\begin{theorem}[Мала лема Урисона (критерій нормальності)]
Досяжний простір $X$ є нормальним тоді й лише тоді, коли
для кожної замкненої підмножини $F \subset X$ і відкритої
множини $U$, що її містить, існує такий відкритий окіл $V$
множини $F$, що $\closure{V} \subset U$, тобто коли кожна замкнена
підмножина має замкнену локальну базу.
\end{theorem}

\begin{proof}
Необхідність. Нехай простір $X$ нормальний.
Розглянемо замкнену множину $F$ та її окіл $U$.
Покладемо $F' = X \setminus U$. Оскільки $F \cap F' = \emptyset$, то існує
відкритий окіл $V$ множини $F$ і відкритий окіл $V'$ множини
$F'$, такі що $V \cap V' = \emptyset$. Отже, $V \subset X \setminus V'$. З цього
випливає, що $\closure{V} \subset \closure{X \setminus V'} = X \setminus V' \subset X \setminus F' = U$.

Достатність. Нехай умови леми виконані, а $F$ і $F'$ ---
довільні диз'юнктні замкнені підмножини простору $X$.
Покладемо $U = X \setminus F'$. Тоді, оскільки множина $U$ є
відкритим околом множини $F$, то за умовою леми, існує
окіл $V$ множини $F$, такий що $\closure{V} \subset U$. Покладаючи
$V' = X \setminus \closure{V}$ безпосередньо переконуємося, що множини $V$ і
$V'$ не перетинаються і є околами множини $F$ і $F'$.
\end{proof}

\begin{theorem}[Велика лема Урисона]
Будь-які непорожні диз'юнктні
замкнені підмножини нормального простору є
функціонально віддільними.
\end{theorem}

\begin{remark}
Ця лема --- критерій нормальності.
\end{remark}

\section{Література}

\begin{enumerate}[label={[\arabic*]}]
\item \textbf{Александрян~Р.А.}
Общая топология /
Р.~А.~Александрян, Э.~А.~Мирзаханян ---
М.: Высшая школа, 1979 (стр.~191--206).
\item \textbf{Колмогоров~А.Н.}
Элементы теории функций и функционального анализа. 5-е изд. /
Колмогоров~А.Н., С.~В.~Фомин ---
М.: Наука, 1981 (стр.~94--97).
\item \textbf{Энгелькинг~Р.}
Общая топология /
Р.~Энгелькинг ---
М.: Мир, 1986 (стр.~69--85).
\end{enumerate}