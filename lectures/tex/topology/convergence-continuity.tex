\chapter{Збіжність і неперервність}

\section{Аксіоми зліченності}

В основі поняття збіжності послідовностей в топологічних просторах лежать аксіоми зліченності, які своєю чергою використовують поняття локальної бази в точці.

\begin{definition}
    Система~$\beta_{x_0}$ відкритих околів точки~$x_0$ називається \vocab{локальною базою в точці~$x_0$}, якщо кожнийокіл~$U$ точки~$x_0$ містить її деякий окіл~$V$ із системи~$\beta_{x_0}$.
\end{definition}

\begin{definition}
    Топологічний простір~$X$ називається таким, що \vocab{задовольняє першій аксіомі зліченності}, якщо в кожній його точці існує локальна база, що складається із не більш ніж зліченої кількості околів цієї точки.
\end{definition}

\begin{definition}
    Топологічний простір~$X$ називається таким, що \vocab{задовольняє другій аксіомі зліченності}, або \vocab{простором зі зліченною базою}, якщо він має базу, що складається із не більш ніж зліченої кількості відкритих множин.
\end{definition}

\begin{lemma}
    Якщо простір~$X$ задовольняє другій аксіомі зліченності, то він задовольняє і першій аксіомі зліченності.
\end{lemma}

\begin{proof}
    Нехай~$U_1, U_2, \ldots, U_n, \ldots$~--- зліченна база в просторі~$X$, тоді~$\beta_{x_0} = \{ U_k \in \beta: x_0 \in U_k \}$~--- зліченна локальна база в точці~$x_0$.
\end{proof}

\begin{lemma}
    Існують простори, що задовольняють першій аксіомі зліченності, але не задовольняють другій аксіомі зліченності.
\end{lemma}

\begin{proof}
    В якості контрприкладу розглянемо довільну \vocab{незліченну} множину~$X$, в якій введено дискретну топологію~$\tau = 2^X$.
\end{proof}

\begin{exercise}
    Переконайтеся що ви розумієте, чому цей простір задовольняє першій аксіомі зліченності, але не задовольняє другій аксіомі зліченності перед тим як читати далі.
\end{exercise}

\begin{example}
    Простір~$\RR^n$, топологія якого утворена відкритими кулями, задовольняє першій аксіомі зліченності, оскільки в кожній точці~$x_0 \in X$ існує зліченна локальна база~$S(x_0, 1 / n)$.

    Очевидно, що цей простір задовольняє і другій аксіомі зліченності, оскільки має зліченну базу, що складається з куль~$S(x_n, r)$, де центри куль~$x_n$ належать зліченній скрізь щільній множині (наприклад, мають раціональні координати), а~$r$~--- раціональне число.
\end{example}

Поняття точки дотику і замикання множини відіграють основну роль в топології, оскільки будь-яка топологічна структура повністю описується в цих термінах.

\section{Збіжність}

Проте поняття точки дотику занадто абстрактне. Набагато більше змістовних результатів можна отримати, якщо виділити широкий клас просторів, топологічну структуру яких можна описати виключно в термінах границь збіжних послідовностей.

\begin{definition}
    Послідовність точок~$\{x_n\}$ топологічного простору~$X$ називається збіжною до точки~$x_0 \in X$, якщо кожний окіл~$U_0$ точки~$x_0$ містить всі точки цієї послідовності, починаючи з деякої. Точку~$x_0$ називають границею цієї послідовності: $\lim\limits_{n \to \infty} x_n = x_0$.
\end{definition}

\begin{example}
    В довільному тривіальному просторі послідовність збігається до будь-якої точки цього простору.
\end{example}

Довільна гранична точка множини~$A$ довільного топологічного простору~$X$ є точкою дотику. Проте в загальних топологічних просторах не для всякої точки дотику~$x \in A$ існує послідовність~$\{x_n\} \subset A$, що до неї збігається.

\begin{example}
    Нехай~$X$~--- довільна незліченна множина. Задамо в просторі~$X$ топологію, оголосивши відкритими порожню множину і всі підмножини, які утворені із~$X$ викиданням не більш ніж зліченної кількості точок.
    \[ \tau = \{ \emptyset, X \setminus \{ x_1, x_2, \ldots, x_n, \ldots\} \}. \]
\end{example}

\begin{proof}
    Спочатку покажемо, що в цьому просторі збіжними є лише стаціонарні послідовності.

    Припустимо, що в просторі існує нестаціонарна послідовність~$\{x_n\} \to x_0$. Тоді, взявши за окіл точки~$x_0$ множину~$U$, яка утворюється викиданням із~$X$ всіх членів послідовності~$\{x_n\}$, які відрізняються від точки~$x_0$, ми дійдемо до суперечності з тим, що окіл~$U$ мусить містити всі точки послідовності~$\{x_n\}$, починаючи з деякої.

    Тепер розглянемо підмножину~$A = X \setminus \{x_0\}$. Точка~$x_0$ є точкою дотику множини~$A$. Справді, якщо~$U$~--- довільний відкритий окіл точки~$x_0$, то за означенням відкритих в~$X$ множин, доповнення~$X \setminus U$ є не більш ніж зліченим.
    \begin{align*}
        & U \in \tau \implies U = X \setminus \{ x_1, x_2, \ldots, x_n, \ldots \} \implies \\
        & \implies X \setminus U = X \setminus (X \setminus \{ x_1, x_2, \ldots, x_n, \ldots \}) = \{ x_1, x_2, \ldots, x_n, \ldots \} \implies \\
        & \implies A \cap U \ne \emptyset,
    \end{align*}
    оскільки~$|A| = c$, а доповнення~$X \setminus U$ і тому не може містити в собі незліченну множину~$A$.

    З іншого боку, оскільки в просторі~$X$ збіжними є лише стаціонарні послідовності, то із~$x_0 \notin A$ випливає, що жодна послідовність точок із множини~$A$ не може збігатися до точки дотику~$x_0 \notin A$.
\end{proof}

\begin{theorem}
    Якщо простір~$X$ задовольняє першій аксіомі зліченності, то~$x_0 \in \closure{A}$ тоді й лише тоді, коли~$x_0$ є границею деякої послідовності~$\{x_n\}$ точок із~$A$.
\end{theorem}

\begin{proof}
    \textbf{Достатність.} Якщо % в довільному топологічному просторі
    послідовність~$\{x_n\} \in A$, $\lim_{n \to \infty} x_n = x_0$, то~$x_0 \in \closure{A}$.

    \textbf{Необхідність.} Нехай~$x_0 \in \closure{A}$. Якщо~$x_0 \in A$, достатньо в якості~$\{x_n\} \in A$ взяти стаціонарну послідовність.

    Припустимо, що~$x_0 \in \closure{A} \setminus A$ і~$U_1, U_2, \ldots, U_n, \ldots$~--- зліченна локальна база в точці~$x_0$, до того ж~$\forall n \in \NN$: $U_{n + 1} \subset U_n$. (Якби ця умова не виконувалася, ми взяли б іншу базу~$\{V_n\}$, де~$V_n = \bigcap_{k = 1}^n U_k$). Оскільки~$A \cap U_n \ne \emptyset$, взявши за~$x_n$ довільну точку із~$A \cap U_n$, ми отримаємо послідовність~$\{x_n\} \in A$, $\lim_{n \to \infty} x_n = x_0$.

    Дійсно, нехай~$V$~--- довільний окіл точки~$x_0$. Оскільки~$U_1, U_2, \ldots, U_n, \ldots$ база в точці~$x_0$, існує такий елемент~$U_{n_0}$, який належить цій базі, що~$U_{n_0} \subset V$. З іншого боку, для всіх~$n \ge n_0$: $U_{n + 1} \subset U_n$. Це означає, що~$\forall n \ge n_0$: $x_n \in A \cap U_n \subset U_{n_0} \subset U$. Отже, $x_0 = \lim_{n \to \infty} x_n$.
\end{proof}

\section{Неперервність}

Поняття неперервного відображення належить до фундаментальних основ топології.

\begin{definition}
    Відображення~$f: X \to Y$ називається \vocab{сюр'єктивним}, якщо $f(X) = Y$, тобто множина~$X$ відображається на весь простір~$Y$.
\end{definition}

\begin{definition}
    Відображення~$f: X \to Y$ називається \vocab{ін'єктивним}, якщо з того, що~$f(x_1) \ne f(x_2)$ випливає, що~$x_1 \ne x_2$.
\end{definition}

\begin{definition}
    Відображення~$f: X \to Y$, яке одночасно є сюр'єктивним та ін'єктивним, називається \vocab{бієктивним}, або взаємно однозначною відповідністю між $X$ і~$Y$.
\end{definition}

Тепер нагадаємо основні співвідношення для образів та прообразів множин відносно функції~$f: X \to Y$.

Якщо~$A, B \subset X$, то
\begin{enumerate}
    \item $A \subset B \implies f(A) \subset f(B) \not\implies A \subset B$;
    \item $A \ne \emptyset \implies f(A) \ne \emptyset$;
    \item $f(A \cap B) \subset f(A) \cap f(B)$;
    \item $f(A \cup B) \subset f(A) \cup f(B)$.
\end{enumerate}
Якщо~$A', B' \subset Y$, то
\begin{enumerate}
    \item $A' \subset B' \implies f\inv(A') \subset f\inv(B')$;
    \item $f\inv(A' \cap B') = f\inv(A') \cap f\inv(B')$;
    \item $f\inv(A' \cup B') = f\inv(A') \cup f\inv(B')$.
\end{enumerate}
Якщо~$B' \subset A' \subset Y$, то
\begin{enumerate}
    \item $f\inv(A' \setminus B') = f\inv(A') \setminus f\inv(B')$;
    \item $f\inv(Y \setminus B') = X \setminus f\inv(B')$;
\end{enumerate}
Для довільних множин~$A \subset X$ і~$B' \subset Y$
\begin{enumerate}
    \item $A \subset f\inv(f(A))$;
    \item $f(f\inv(B')) \subset B'$.
\end{enumerate}

Введемо поняття неперервного відображення.

\begin{definition}
    Нехай~$X$ і~$Y$~--- два топологічних простора. Відображення~$f: X \to Y$ називається \vocab{неперервним в точці~$x_0$}, якщо для довільного околу~$V$ точки~$y_0 = f(x_0)$ існує такий окіл~$U$ точки~$x_0$, що~$f(U) \subset V$.
\end{definition}

\begin{definition}
    Відображення~$f: X \to T$ називається \vocab{неперервним}, якщо воно є неперервним в кожній точці~$x \in X$.
\end{definition}

Інакше кажучи, неперервне відображення зберігає граничні властивості: якщо точка~$x \in X$ є близькою до деякої множини~$A \subset X$, то точка~$y = f(x) \in Y$ є близькою до образу множини~$A$.

\begin{theorem}
    Для того щоб відображення~$f: X \to Y$ було неперервним, необхідно і достатньо, щоб прообраз~$f\inv(V)$ будь-якої відкритої множини~$V \subset Y$ був відкритою множиною в~$X$.
\end{theorem}

\begin{proof}
    \textbf{Необхідність.} Нехай~$f: X \to Y$~--- неперервне відображення, а~$V$~--- довільна відкрита множина в~$Y$. Доведемо, що множина~$U = f\inv(V)$ є відкритою в~$X$.

    Для цього візьмемо довільну точку~$x_0 \in U$ і позначимо~$y_0 = f(x_0)$. Оскільки множина~$V$ є відкритим околом точки~$y_0$ в просторі~$Y$, а відображення~$f$ є неперервним в точці~$x_0$, в просторі~$X$ існує відкритий окіл~$U_0$ точки~$x_0$, такий що~$f(U_0) \subset V$. Звідси випливає, що~$U_0 \subset U$. Отже, множина~$U$ є відкритою в~$X$.
    \begin{multline*}
        f \in C(X, Y) \implies \exists U_0 \in \tau_X: x_0 \in U_0, f(U_0) \subset V \implies \\
        \implies f\inv(f(U_0)) \subset f\inv(V) = U \implies U_0 \subset f\inv(f(U_0)) \subset U \implies U \in \tau_X.
    \end{multline*}

    \textbf{Достатність.} Нехай прообраз~$f\inv(V)$ довільної відкритої в~$Y$ множини~$V$ є відкритим в~$X$, а~$x_0 \in X$~--- довільна точка. Доведемо, що відображення~$f$ є неперервним в точці~$x_0$.

    Дійсно, нехай~$y_0 = f(x_0)$, а~$V$~--- її довільний відкритий окіл. Тоді~$U = f\inv(V)$ за умовою теореми є відкритим околом точки~$x_0$, до того ж~$f(U) \subset V$. Отже, відображення~$f$ є неперервним в кожній точці~$x_0 \in X$. Таким чином, $f$ є неперервним в~$X$.
    \begin{equation*}
        V \in \tau_X, U := f\inv(V) \in \tau_X \implies f(U) = f(f\inv(V)) \subset V \implies f \in C(X, Y). \qedhere
    \end{equation*}
\end{proof}

\begin{theorem}
    Для того щоб відображення~$f: X \to Y$ було неперервним, необхідно і достатньо, щоб прообраз~$f\inv(V)$ будь-якої замкненої множини~$V \subset Y$ був замкненою множиною в~$X$.
\end{theorem}

Доведення випливає з того, що доповнення відкритих множин є замкненими, а прообрази множин, що взаємно доповнюють одна одну, самі взаємно доповнюють одна одну.

\begin{theorem}
    Для того щоб відображення~$f: X \to Y$ було неперервним, необхідно і достатньо, щоб~$\forall A \subset X: f(\closure{A}) \subset \closure{f(A)}$.
\end{theorem}

\begin{proof}
    \textbf{Необхідність.} Нехай відображення~$f: X \to Y$ є неперервним, а~$x_0 \in \closure{A}$. Покажемо, що~$y_0 = f(x_0) \in \closure{f(A)}$.

    Справді, нехай~$V$~--- довільний окіл точки~$y_0$. Тоді внаслідок неперервності~$f$ існує окіл~$U$, який містить точку~$x_0$ такий, що~$f(U) \subset V$. Оскільки~$x_0 \in \closure{A}$, то в околі~$U$ повинна міститись точка~$x' \in A$ (можливо, вона збігається з точкою~$x_0$). Разом з тим, очевидно, що~$y' = f(x')$ належить одночасно множині~$f(A)$ і околу~$V$, тобто~$y_0 \in \closure{f(A)}$.
    \begin{multline*}
        f \in C(X, Y) \implies \forall V \in \tau_Y: f(x_0) \in V: \exists U \in \tau_X: x \in U, f(U) \subset V. \\
        x_0 \in \closure{A} \implies U \cap A \ne \emptyset \implies \exists x' \in U \cap A \implies \\
        \implies f(x') \in f(U \cap A) \subset f(U) \cap f(A) \implies y_0 = f(x_0) \in \closure{f(A)}.
    \end{multline*}

    \textbf{Достатність.} Нехай~$\forall A \subset X: f(\closure{A}) \subset \closure{f(A)}$ і~$B$~--- довільна замкнена в~$Y$ множина. Покажемо, що множина~$A = f\inv(B)$ є замкненою в~$X$.

    Нехай~$x_0$~--- довільна точка із~$\closure{A}$. Тоді~$f(x_0) \in f(\closure{A}) \subset \closure{f(A)}$. Разом з тим
    \[ A = f\inv(B) \implies f(A) = f(f\inv(B)) \subset B \implies \closure{f(A)} \subset \closure{B} = B. \]

    Тому~$f(x_0) \in B$, отже, $x_0 \in A$. Таким чином, $\closure{A} \subset A$, тобто~$A$~--- замкнена множина. Звідси випливає, що відображення~$f$ є неперервним.
\end{proof}

\section{Гомеоморфізми}

\begin{definition}
    Бієктивне відображення~$f: X \to Y$ називається \vocab{гомеоморфним}, або \vocab{гомеоморфізмом}, якщо і само відображення~$f$ і обернене відображення~$f\inv~$ є неперервними.
\end{definition}

\begin{definition}
    Топологічні простори~$X$ і~$Y$ називаються \vocab{гомеоморфними}, або \vocab{топологічно еквівалентними}, якщо існує хоча б одне гомеоморфне відображення~$f: X \to Y$.
\end{definition}

Цей факт записується так: $X \overset{f}{\equiv} Y$.

\begin{example}
    Тривіальний приклад гомеоморфізму~--- тотожне перетворення.
\end{example}

\begin{example}
    Відображення, що задається строго монотонними неперервними дійсними функціями дійсної змінної є гомеоморфізмами. Гомеоморфним образом довільного інтервалу є інтервал.
\end{example}

\begin{definition}
    Неперервне відображення~$f: X \to Y$ називається \vocab{відкритим}, якщо образ будь-якої відкритої множини простору~$X$ є відкритим в~$Y$.
\end{definition}

\begin{definition}
    Неперервне відображення~$f: X \to Y$ називається \vocab{замкненим}, якщо образ будь-якої замкненої множини простору~$X$ є замкненим в~$Y$.
\end{definition}

\begin{remark}
    Поняття відкритого і замкненого відображення не є взаємовиключними. Тотожне відображення одночасно є і відкритим, і замкненим.
\end{remark}

\begin{example}
    Відображення вкладення (ін'єктивне відображення)~$i: A \subset X \to X$ є відкритим, якщо підмножина~$A$ є відкритою, і замкненим, якщо підмножина~$A$ є замкненою.
\end{example}

\begin{theorem}
    Відображення~$f: X \to Y$ є замкненим тоді й лише тоді, коли~$\forall A \subset X: f(\closure{A}) = \closure{f(A)}$.
\end{theorem}

\begin{proof}
    \textbf{Необхідність.} Оскільки замкнене відображення є неперервним (за означенням), то внаслідок \error теореми 3.4~$\forall A \subset X: f(\closure{A}) \subset \closure{f(A)}$.

    Разом з тим, очевидно, що~$f(A) \subset f(\closure{A})$, тому внаслідок монотонності замикання~$\closure{f(A)} \subset \closure{f(\closure{A})}$.

    Оскільки відображення~$f$ є замкненим, то~$\closure{f(\closure{A})} = f(\closure{A})$. Таким чином, $\closure{f(A)} = f(\closure{A})$.

    \textbf{Достатність.} Функція~$f$ є неперервною внаслідок \error теореми 3.4. З умови~$\closure{f(A)} = f(\closure{A})$ для замкненої множини~$A \subset X$ отримуємо, що~$f(A) = \closure{f(A)}$, тобто образ будь-якої замкненої множини є замкненим.
\end{proof}

\begin{theorem}
    Відкрите бієктивне відображення~$f: X \to Y$ є гомеоморфізмом.
\end{theorem}

\begin{proof}
    Оскільки~$f: X \to Y$~--- бієктивне відображення, існує обернене відображення~$f\inv : Y \to X$. Оскільки~$\forall A \subset X: (f\inv )\inv(A) = f(A)$ і, за умовою теореми, $f$~--- відкрите відображення, то прообрази відкритих підмножин із~$X$ є відкритими.

    З \error теореми 3.2 випливає, що відображення~$f\inv~$ є неперервним. Оскільки бієктивне відкрите відображення завжди є неперервним, доходимо висновку, що~$f$~--- гомеоморфізм.
\end{proof}

\begin{theorem}
    Замкнене бієктивне відображення~$f: X \to Y$ є гомеоморфізмом.
\end{theorem}

Доведення цілком аналогічне попередній теоремі.

\begin{theorem}
    Гомеоморфне відображення~$f: X \equiv Y$ одночасно є і відкритим, і замкненим.
\end{theorem}

\begin{proof}
    Нехай~$f\inv : Y \to X$~--- обернене відображення. Тоді~$\forall A \subset X: f(A) = (f\inv )\inv(A)$. Оскільки відображення~$f$ є гомеоморфізмом, відображення~$f$ і~$f\inv~$ є неперервними.

    Оскільки образ множини~$A$ при відображенні~$f$ є прообразом множини~$A$ при відображенні~$(f\inv )\inv~$ і обидва ці відображення є неперервними, то відображення~$f$ є відкритим і замкненим одночасно, тобто відкриті множини переводить у відкриті, а замкнені~--- у замкнені.
\end{proof}

\begin{theorem}
    Бієктивне відображення~$f: X \to Y$ є гомеоморфізмом тоді й лише тоді, коли воно зберігає операцію замикання, тобто~$\forall A \subset X: f(\closure{A}) = \closure{f(A)}$.
\end{theorem}

Необхідність випливає з \error теорем 3.5 і 3.8, а достатність~--- з \error теорем 3.5 і 3.7.

\section{Література}

\begin{enumerate}[label={[\arabic*]}]
\item \textbf{Александрян~Р.А.}
Общая топология /
Р.~А.~Александрян, Э.~А.~Мирзаханян ---
М.: Высшая школа, 1979 (стр.~24--28).
\item \textbf{Энгелькинг~Р.}
Общая топология /
Р.~Энгелькинг ---
М.: Мир, 1986 (стр.~57--68).
\item \textbf{Колмогоров~А.Н.}
Элементы теории функций и функционального анализа. 5-е изд. /
Колмогоров~А.Н., С.~В.~Фомин ---
М.: Наука, 1981 (стр.~89--91).
\end{enumerate}