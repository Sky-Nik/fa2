\chapter{Топологічні простори}

\section{Нагадування: метрична топологія}

В курсі математичного аналізу [1, c. 26] уже розглядалися поняття околу точки, відкритої та замкненої множин, точки дотику, граничної точки, границі послідовності в просторі~$\RR$ тощо. Всі ці поняття визначалися за допомогою метрики простору~$\RR$ і відбивали певні властивості, притаманні множинам, за допомогою яких ми могли описувати основну концепцію цієї теорії~--- близькість між точками. Адже саме поняття близькості між точками (в розумінні малої відстані) є базовим для таких головних понять математичного аналізу як збіжність послідовностей і неперервність функцій.

Відносним недоліком цього підходу є очевидна залежність від метрики, уведеної в просторі. Тому постало питання, чи не можна побудувати більш абстрактну конструкцію, за допомогою якої можна було б описати ідеї, згадані вище. Серед дослідників цієї проблеми слід відзначити французьких математиків М.~Фреше (1906), М.~Рісса (1907--1908), німецького математика Ф.~Хаусдорфа (1914), польського математика К.~Куратовського (1922) і радянського математика П.~Александрова (1924). В результаті досліджень цих та багатьох інших математиків виникла нова математична дисципліна~--- загальна топологія, предметом якої є вивчення ідеї про неперервність на максимально абстрактному рівні.

В цій та наступній лекціях ми введемо в розгляд ряд важливих топологічних понять. Це дозволить нам вийти на вищий рівень абстракції та опанувати ідеї, що пронизують майже всі розділи математики. Не буде великим перебільшенням сказати, що в певному розумінні топологія разом з алгеброю є скелетом сучасної математики, а \vocab{функціональний аналіз}~--- це розділ математики, головною задачею якого є дослідження нескінченновимірних просторів та їх відображень.

\section{Основні означення}

\begin{definition}
    Нехай~$X$~--- множина елементів, яку ми будемо називати \vocab{носієм}. \vocab{Топологією} в~$X$ називається довільна система~$\tau$ його підмножин, яка задовольняє таким умовам (\vocab{аксіомам Александрова}):
    \begin{enumerate}
        \item[А1.]~$\emptyset, X \in \tau$.
        \item[A2.]~$G_\alpha \in \tau$,~$\alpha \in A \implies \Bigcup_{\alpha \in A} G_\alpha \in \tau$, де~$A$~--- довільна множина.
        \item[A3.]~$G_\alpha \in \tau$,~$\alpha = 1, 2, \dots, n \implies \Bigcap_{\alpha = 1}^n G_\alpha \in \tau$.
    \end{enumerate}
\end{definition}

Інакше кажучи, топологічною структурою називається система множин, замкнена відносно довільного об'єднання і скінченого перетину.

\begin{definition}
    Пара~$T = (X, \tau)$ називається \vocab{топологічним простором}.
\end{definition}

\begin{example}[топологічного простору]
    Нехай~$X$~--- довільна множина,~$\tau = 2^X$~--- множина всіх підмножин~$X$. Пара~$(X, 2^X)$ називається простором з \vocab{дискретною} (\emph{максимальною}) топологією.
\end{example}

\begin{example}[топологічного простору]
    Нехай~$X$~--- довільна множина,~$\tau = \{\emptyset, X\}$. Пара~$(X, \tau)$ називається простором з \vocab{тривіальною} (\emph{мінімальною}, або \emph{антидискретною}) топологією.
\end{example}

Зрозуміло, що на одній і тій же множині~$X$ можна ввести різні топології, утворюючи різні топологічні простори. Припустимо, що на носії~$X$ введено дві топології~---~$\tau_1$ і~$\tau_2$. Вони визначають два топологічні простори: $T_1 = (X, \tau_1)$, і~$T_2 = (X, \tau_2)$.

\begin{definition}
    Говорять, що топологія~$\tau_1$ є \vocab{сильнішою}, або \emph{тонкішою}, ніж топологія~$\tau_2$, якщо~$\tau_2 \subset \tau_1$. Відповідно, топологія~$\tau_2$ є \vocab{слабкішою}, або \emph{грубішою}, ніж топологія~$\tau_1$. Легко бачити, що найслабкішою є тривіальна топологія, а найсильнішою~--- дискретна.
\end{definition}

\begin{remark}
    Множина всіх топологій не є цілком упорядкованою, тобто не всі топології можна порівнювати одну з одною. Наприклад, наступні топології (зв'язні двокрапки) порівнювати не можна: $X = \{a, b\}$,~$\tau_1 = \{\emptyset, X, \{a\}\}$,~$\tau_2 = \{\emptyset, X, \{b\}\}$.
\end{remark}

\begin{definition}
    Множини, що належать топології~$\tau$, називаються \vocab{відкритими}. Множини, які є доповненням до відкритих множин, називаються \vocab{замкненими}.
\end{definition}

Наприклад, множина всіх цілих чисел~$\ZZ$ замкнена в~$\RR$.

\begin{remark}
    Топологія містить всі відкриті множини. Водночас, треба зауважити, що поняття відкритих і замкнених множин не є взаємовиключними. Одна і та ж множина може бути одночасно і відкритою і замкненою (наприклад,~$\emptyset$ або~$X$), або ані відкритою, ані замкненою (множини раціональних та ірраціональних чисел в~$\RR$). Отже, топологія може містити й замкнені множини, якщо вони одночасно є відкритими.
\end{remark}

Як бачимо, поняття відкритої множини в топологічному просторі постулюється~--- для того щоб довести, що деяка множина~$M$ в топологічному просторі~$T$ є відкритою, треба довести, що вона належить його топології.

\begin{definition}
    Нехай~$(X, \tau)$~--- топологічний простір,~$M \subset X$. Топологія~$(M, \tau_M)$, де~$\tau_M = \{ U_M^{(\alpha)} = U_\alpha \cap M, U_\alpha \in \tau\}$, називається \vocab{індукованою}.
\end{definition}

\begin{definition}
    Топологічний простір~$(X, \tau)$ називається \vocab{зв'язним}, якщо лише множини~$X$ і~$\emptyset$ є замкненими й відкритими одночасно.
\end{definition}

\begin{definition}
    Множина~$M$ топологічного простору~$(X, \tau)$ називається \vocab{зв'яз\-ною}, якщо топологічний простір~$(M, \tau_M)$ є зв'язним.
\end{definition}

\begin{example}[зв'язних просторів]
    Тривіальний (антидискретний) простір і зв'язна двокрапка є зв'яз\-ними просторами.
\end{example}

\begin{abuse}
    Надалі ми будемо часто скорочувати~$(X, \tau)$ просто як~$X$ або~$T$.
\end{abuse}

\begin{definition}
    Довільна відкрита множина~$G \in T$, що містить точку~$x \in T$, називається її \vocab{околом}.
\end{definition}

\begin{definition}
    Точка~$x \in T$ називається \vocab{точкою дотику} множини~$M \subset T$, якщо кожний окіл~$O(x)$ точки~$x$ містить хоча б одну точку із~$M$: $\forall O(x) \in \tau: O(x) \cap M \ne \emptyset$.
\end{definition}

\begin{definition}
    Точка~$x \in T$ називається \vocab{граничною точкою} множини~$M \subset T$, якщо кожний окіл точки~$x$ містить хоча б одну точку із~$M$, що не збігається з~$x$: $\forall O(x) \in \tau: O(x) \cap M \setminus \{x\} \ne \emptyset$.
\end{definition}

\section{Замикання}

\begin{definition}
    Сукупність точок дотику множини~$M \subset T$ називається \vocab{замиканням} множини~$M$ і позначається~$\closure M$.
\end{definition}

\begin{definition}
    Сукупність граничних точок множини~$M \subset T$ називається \vocab{похідною} множини~$M$ і позначається~$M'$.
\end{definition}

\begin{theorem}[про властивості замикання]
    Замикання задовольняє наступним умовам:
    \begin{enumerate}
        \item $M \subset \closure M$;
        \item $\closure{\closure{M}} = \closure M$ (ідемпотентність);
        \item $M \subset N \implies \closure M \subset \closure N$ (монотонність);
        \item $\closure {M \cup N} = \closure M \cup \closure N$ (адитивність).
        \item $\closure \emptyset = \emptyset$.
    \end{enumerate}
\end{theorem}

\begin{proof}
    \listhack
    \begin{enumerate}
        \item $M \subset \closure M$.

        Нехай~$x \in M$. Тоді~$x$~--- точка дотику множини~$M$. Отже,~$x \in \closure M$.
        
        \item $\closure{\closure{M}} = \closure M$.
        
        Внаслідок твердження 1)~$\closure M \subset \closure{\closure{M}}$. Отже, достатньо довести, що~$\closure{\closure{M}} \subset \closure M$. Нехай~$x_0 \in \closure{\closure{M}}$ і~$U_0$~--- довільний окіл точки~$x_0$. Оскільки~$U_0 \cap \closure M \ne \emptyset$ (за означенням точки дотику), то існує точка~$y_0 \in U_0 \cap \closure M$. Отже, множину~$U_0$ можна вважати околом точки~$y_0$. Оскільки~$y_0 \in \closure M$, то~$U_0 \cap M \ne \emptyset$. Значить, точка~$x_0$ є точкою дотику множини~$M$, тобто~$x_0 \in \closure M$.
        
        \item $M \subset N \implies \closure M \subset \closure N$.
        
        Нехай~$x_0 \in \closure M$ і~$U_0$~--- довільний окіл точки~$x_0$. Оскільки~$U_0 \cap M \ne \emptyset$ (за означенням точки дотику) і~$M \subset N$ (за умовою), то~$U_0 \cap N \ne \emptyset$. Отже,~$x_0$~--- точка дотику множини~$N$, тобто~$x_0 \in \closure N$. Таким чином,~$\closure M \subset \closure N$.
        
        \item $\closure{M \cup N} = \closure M \cup \closure N$.
        
        З очевидних включень~$M \subset M \cup N$ і~$N \subset M \cup N$ внаслідок монотонності операції замикання випливає, що~$\closure M \subset \closure{M \cup N}$ і~$\closure N \subset \closure{M \cup N}$. Отже,~$\closure M \cup \closure N \subset \closure{M \cup N}$. З іншого боку, припустимо, що~$x \not\in \closure M \cup \closure N$, тоді~$x \not\in \closure M$ і~$x \not\in \closure N$. Отже, існує такий окіл точки~$x$, у якому немає точок з множини~$M \cup N$, тобто~$x \not\in \closure{M \cup N}$. Таким чином, за законом заперечення,~$x \in \closure{M \cup N} \implies x \in \closure M \cup \closure N$, тобто~$\closure{M \cup N} \subset \closure M \cup \closure N$.
        
        \item $\closure \emptyset = \emptyset$.
        
        Припустимо, що замикання порожньої множини не є порожньою множиною: $x \in \closure \emptyset \implies \forall O(x): O(x) \cap \emptyset \ne \emptyset$. Але~$\forall N \subset X: N \cap \emptyset = \emptyset$. Отже,~$\closure \emptyset = \emptyset$. \qedhere
    \end{enumerate}
\end{proof}

\begin{theorem}[критерій замкненості]
    Множина~$M$ топологічного простору~$X$ є замкненою тоді й лише тоді, коли~$M = \closure M$, тобто коли вона містить всі свої точки дотику.
\end{theorem}

\begin{proof}
    \textbf{Необхідність.} Припустимо, що~$M$~--- замкнена множина, тобто~$G = X \setminus M$~--- відкрита множина. Оскільки,~$M \subset \closure M$, достатньо довести, що~$\closure M \subset M$. Дійсно, оскільки~$G$~--- відкрита множина, вона є околом кожної своєї точки. До того ж~$G \cap M = \emptyset$. Звідси випливає, то жодна точка~$x \in G$ не може бути точкою дотику для множини~$M$, отже всі точки дотику належать множині~$M$, тобто~$\closure M \subset M$.
    \begin{equation*}
        G = X \setminus M \in \tau \implies G \cap M = \emptyset \implies \closure M \subset M.
    \end{equation*}

    \textbf{Достатність.} Припустимо, що~$\closure M = M$. Доведемо, що~$G = X \setminus M$~--- відкрита множина (звідси випливатиме замкненість множини~$M$). Нехай~$x_0 \in G$. З цього випливає, що~$x_0 \not\in M$, а значить~$x_0 \not\in \closure M$. Тоді за означенням точки дотику існує окіл~$U_{x_0}$ такий, що~$U_{x_0} \cap M = \emptyset$. Значить,~$U_{x_0} \subset X \setminus M = G$, тобто~$G = \Bigcup_{x \in G} U_x \in \tau$.
\end{proof}

\begin{corollary}
    Замикання~$\closure M$ довільної множини~$M$ із простору~$X$ є замкненою множиною в~$X$.
\end{corollary}

\begin{theorem}
    Замикання довільної множини~$M$ простору~$(X, \tau)$ збігається із перетином всіх замкнених множин, що містять множину~$M$.
    \begin{equation*}
        \forall M \subset X \quad \closure M = \Bigcap_\alpha F_\alpha, \quad F_\alpha = \closure F_\alpha, M \subset F_\alpha.
    \end{equation*}
\end{theorem}

\begin{proof}
    Нехай~$M$~--- довільна множина із~$(X, \tau)$ і~$N = \Bigcap_\alpha F_\alpha$, де~$F_\alpha = \closure F_\alpha$,~$M \subset F_\alpha$.

    Покажемо включення~$\Bigcap_\alpha F_\alpha \subset \closure M$.
    \begin{equation*}
        N = \Bigcap_\alpha F_\alpha \implies N \subset F_\alpha \forall \alpha \implies N \subset \closure F_\alpha \forall \alpha.
    \end{equation*}
    
    Оскільки~$\{F_\alpha\}$~--- множина усіх замкнених множин, серед них є множина~$\closure M$: $\exists \alpha_0: F_{\alpha_0} = \closure M$. Отже,
    \begin{equation*}
        N \in \closure F_\alpha \forall \alpha \implies N \in F_{\alpha_0} = \closure M \implies \Bigcap_\alpha F_\alpha \subset \closure M.
    \end{equation*}
    
    Тепер покажемо включення~$\closure M \subset \Bigcap_\alpha F_\alpha$. Розглянемо довільну замкнену множину~$F$, що містить~$M$: $F = \closure F$,~$M \subset F$. Внаслідок монотонності замикання маємо:
    \begin{equation*}
        \closure F = F, M \subset F \implies
        \closure M \subset \closure F = F \implies
        \closure M \subset F_\alpha, F_\alpha = \closure \forall \alpha \implies
        \closure M \subset \Bigcap_\alpha F_\alpha.
    \end{equation*}
    
    Порівнюючи обидва включення, маємо
    \begin{equation*}
        \closure M = \Bigcap_\alpha F_\alpha. \qedhere
    \end{equation*}
\end{proof}

\begin{corollary}
    Замикання довільної множини~$M$ простору~$X$ є найменшою замкненою множиною, що містить множину~$M$.
\end{corollary}

\section{Щільність}

\begin{definition}
    Нехай~$A$ і~$B$~--- дві множини в топологічному просторі~$T$. Множина~$A$ називається \vocab{щільною в~$B$}, якщо~$\closure A \supset B$.
\end{definition}

\begin{example}[щільних множин]
    В топології числової прямої множина всіх раціональних чисел~$\QQ$ є щільною в множині всіх ірраціональних чисел~$\RR \setminus \QQ$, і навпаки.
\end{example}

\begin{remark}
    Множина~$A$ не обов'язково міститься в~$B$: множина раціональних чисел є щільною в множині ірраціональних чисел і навпаки.
\end{remark}

\begin{definition}
    Якщо~$\closure A = X$, множина~$A$ називається \vocab{скрізь щільною}.
\end{definition}

\begin{definition}
    Множина~$A$ називається \vocab{ніде не щільною}, якщо вона не є щільною в жодній непорожній відкритій підмножині множини~$X$.
\end{definition}

\begin{example}[ніде не щільних множин]
    Найпростішими прикладами ніде не щільних множин є цілі числа просторі~$\RR$ і пряма в просторі~$\RR^2$.
\end{example}

Множина~$A$ є щільною в кожній непорожній відкритій множині, якщо~$\forall U \in \tau$,~$U \ne \emptyset$: $\closure A \supset U$, тобто кожна точка множини~$U$ є точкою дотику множини~$A$. Отже,~$\forall x \in U \forall O(x) \in \tau O(x) \cap A \ne \emptyset$. Заперечення цього твердження збігається з означенням ніде не щільної множини. Формальний запис означення має такий вигляд:
\begin{equation*}
    \exists U_0 \in \tau, U_0 \ne \emptyset: \closure A \not\supset U_0 \implies \exists x_0 \in U_0 \exists O(x_0) \in \tau: O(x_0) \cap A = \emptyset
\end{equation*}

\begin{definition}
    Простір~$T$, що містить скрізь щільну зліченну множину, називається \vocab{сепарабельним}.
\end{definition}

\begin{example}[сепарабельного простору]
    Зліченна множина всіх раціональних чисел~$\QQ$ є скрізь щільною у просторі~$\RR$, отже простір~$\RR$ є сепарабельним.
\end{example}

З того, що~$\closure \QQ = \RR$ і~$\closure{\RR \setminus \QQ} = \RR$, зокрема, випливає, що~$\QQ$ і~$\RR \setminus \QQ$ є ані відкритими, ані замкненими множинами.

\begin{example}[сепарабельного простору]
    Зліченна множина всіх поліномів з раціональними коефіцієнтами за теоремою Вєйєрштрасса є скрізь щільною в просторі неперервних функцій~$C[a, b]$. Отже, простір~$C[a, b]$ є сепарабельним.
\end{example}

\section{Література}

\begin{enumerate}[label={[\arabic*]}]
\item \textbf{Ляшко~И.~И.}
Основы классического и современного математического анализа /
И.~И.~Ляшко, В.~Ф~Емельянов, А.~К.~Боярчук. ---
К.: Вища школа, 1988 (стр.~26--27).
\item \textbf{Александрян~Р.А.}
Общая топология /
Р.~А.~Александрян, Э.~А.~Мирзаханян ---
М.: Высшая школа, 1979 (стр.~10--20).
\item \textbf{Энгелькинг~Р.}
Общая топология /
Р.~Энгелькинг. ---
М.: Мир, 1986 (стр.~32--50).
\end{enumerate}
