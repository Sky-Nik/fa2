\chapter{Напрямленості}

Як добре відомо, в основі усіх основних понять і конструкцій математичного аналізу (неперервності, диференційовністі, інтегрованісті, сумування рядів тощо) лежить концепція збіжності. В основному курсі функціонального аналізу ми показали, що за допомогою цієї концепції в топологічних просторах, що задовольняють першу аксіому зліченості, можна навіть задавати топологію. 

Концепція збіжності містить в собі два поняття: послідовність і границю. Спочатку в математиці розглядалися лише послідовності дійсних чисел. Згодом теорію розповсюдили на послідовності точок в метричному просторі, і, нарешті, узагальнили для послідовності точок в довільному топологічному просторі. 

Прагнення вийти за межі просторів, що задовольняють першу аксіому зліченості, в 1920-х роках привело до узагальнення поняття границі звичайних послідовностей на узагальнену послідовність (збіжність за Муром--Смітом) і появи теорії напрямленостей. В 1930-х роках французський математик А.~Картан розробив загальну теорію збіжності, яка заснована на поняттях фільтра, ультрафільтра та їх границь. Ця теорія є універсальною. Вона заміняє теорію Мура--Сміта і суттєво спрощує загальну теорію збіжності. 

Для того щоб глибше зрозуміти зміст цих теорій, доцільно детально їх розглянути та порівняти. 

\section{Частково упорядковані множини (нагадування)}

Нагадаємо деякі означення із теорії множин.

\begin{definition}
    Нехай $A$~--- довільна множина. Позначимо як $A \times A$ сукупність усіх упорядкованих пар $(a, b)$, де $a, b \in A$. Говорять, що в множині A задано \vocab{бінарне відношення} $\phi$, якщо в $A \times A$ виділено довільну підмножину $R_\phi$. Елемент $a$ перебуває у відношенні $\phi$ з елементом $b$, якщо пара $(a, b)$, належить $R_\phi$.
\end{definition}

\begin{example}
    Бінарним відношенням є, наприклад, тотожність. Множиною $R_\phi$ у цьому випадку є діагональ $(a, a) \in A \times A$.
\end{example}

\begin{definition}
    Бінарне відношення, задане в множині $A$, називається \vocab{відношенням часткового передупорядкування}, якщо воно є рефлексивним і транзитивним, тобто
    \begin{enumerate}
        \item $(a, a) \in R_\phi$~--- рефлексивність;
        \item $(a, b), (b, c) \in R_\phi \implies (a, c) \in R_\phi$~--- транзитивність.
    \end{enumerate}
\end{definition}

\begin{definition}
    Бінарне відношення, задане в множині $A$, називається \vocab{відношенням часткового упорядкування}, якщо воно є рефлексивним, транзитивним і антисиметричним, тобто
    \begin{enumerate}
        \item $(a, a) \in R_\phi$~--- рефлексивність;
        \item $(a, b), (b, c) \in R_\phi \implies (a, c) \in R_\phi$~--- транзитивність.
        \item $(a, b), (b, a) \in R_\phi \implies a = b$~--- антисиметричність.
    \end{enumerate}
\end{definition}

\begin{definition}
    Множина $A$ із заданим на ній відношенням часткового упорядкування (передупорядкування) називається \vocab{частково упорядкованою (передупорядкованою) множиною}.
\end{definition}

\begin{remark}
    У частково упорядкованих множинах за традицією відношення $x R y$ позначають як $x \le y$ або $y \ge x$.
\end{remark}

\section{Напрямленості}

\begin{definition}
    Частково упорядкована множина $S$ називається \vocab{фільтрівною вправо}, або \vocab{напрямленням за зростанням}, або просто \vocab{напрямленою множиною}, якщо
    \begin{equation*}
        \forall s_1, s_2 \in S \quad \exists s \in S: \quad s \ge s_1, s_2.
    \end{equation*}
\end{definition}

\begin{example}
    Множина натуральних чисел із природним упорядкуванням є напрямленою.
\end{example}

\begin{example}
    Нехай $x$~--- фіксована точка топологічного простору $X$, а $\Omega_x$~--- сукупність усіх околів цієї точки. 

    Введемо в множині $\Omega_x$ відношення упорядкування за оберненим включенням:
    \begin{equation*}
        V \subset U \iff V \ge U.
    \end{equation*}

    Оскільки
    \begin{equation*}
        \forall U_1, U_2 \in \Omega \quad U_1 \cap U_2 \ge U_1, U_2, 
    \end{equation*}
    то множина $\Omega_x$ є \emph{напрямленою} множиною усіх околів точки $x$ в просторі $X$.
\end{example}

Розглянемо довільну множину $X$ і деяку послідовність її елементів $x_n$. Послідовність $x_n$ можна трактувати як відображення
\begin{equation*}
    f: \NN \to X,
\end{equation*}
де $f(n) = x_n$. 

Якщо замінити множину $\NN$ довільною напрямленою множиною $S$, отримаємо означення узагальненої послідовності, або напрямленості.

\begin{definition}
    Будь-яке відображення напрямленої множини називається \vocab{напрямленістю}, або \vocab{узагальненою послідовністю}, або \vocab{сіттю}. До того ж, якщо $f: S \to X$~--- напрямленість, то напрямлена множина $S$ називається \emph{областю визначеності напрямленості} $f$, а множина $f(S)$~--- \emph{областю її значень}.
\end{definition}

\begin{remark}
    Будь-яка послідовність елементів простору $X$ є напрямленістю в $X$ з областю визначення $\NN$. Для зручності значення $f_s$ напрямленості $f: S \to X$ на елементі $s \in S$ часто позначають як $x_s$, а саму напрямленість $f$ подають як множину $\{x_s \mid s \in S\}$.
\end{remark}

\begin{example}
    Нехай $\Omega_x$~--- напрямлена множина усіх околів точки $x$ простору $X$. Вибираючи по одній точці $x_U$ з кожного околу $U \subset \Omega_x$, ми отримуємо напрямленість $\{x_U \mid U \in \Omega_x\}$.
\end{example}

\begin{definition}
    Говорять, що напрямленість $f : S \to X$ починаючи з деякого місця \vocab{належить}, або \vocab{майже вся лежить} в підмножині $A \subset X$, якщо існує $s_0 \in S$, таке що $\forall s \ge s_0$ $x_s \in A$.
\end{definition}

\begin{definition}
    Якщо $\forall s \in A$ $\exists t \ge s$: $f_t \in A$, то говорять, що напрямленість $f: S \to X$ є \vocab{частою} в підмножині $A \subset X$ (\vocab{часто буває} в $A$).
\end{definition}

\begin{remark}
    Якщо напрямленість $f: S \to X$ є частою в $A$, то вона не може майже вся лежати в доповненні $X \setminus A$. І навпаки, якщо напрямленість майже вся лежить в доповненні $X \setminus A$, то вона не може бути частою в $A$.
\end{remark}

\begin{definition}
    Точка $x^\star$ називається \vocab{граничною точкою} напрямленості, якщо ця напрямленість часто буває в будь-якому околі точки $x^\star$.
\end{definition}

\section{Границі напрямленості}

\begin{definition}
    Напрямленість $f: S \to X$ в топологічному просторі $X$ називається \vocab{збіжною} до точки $x_0 \in X$, якщо вона майже вся лежить в будь-якому околі точки $x_0$, тобто якщо для довільного околу $U$ цієї точки знайдеться елементs $s_U \in S$, такий що $\forall s \ge s_U$ $f_s \in U$. Точка $x_0 = \lim_S f_s$ називається \vocab{границею} напрямленості $f: S \to X$.
\end{definition}

\begin{example}
    Кожна збіжна послідовність в просторі $X$ є збіжною напрямленістю в $X$, границя якої є границею послідовності.
\end{example}

\begin{example}
    Нехай $\{x_U \mid U \in \Omega_x\}$~--- напрямленість в просторі $X$. Легко бачити, що ця напрямленість збігається до точки $x$. Дійсно, нехай $U_0$~--- довільний окіл точки $x$. Тоді $\forall U \ge U_0$ $x \in U \subset U_0$, тобто ця напрямленість майже вся лежить в довільному околі точки $x$.
\end{example}

\begin{remark}
    Напрямленість, як і послідовність, в загальних топологічних просторах може мати різні границі. В хаусдорфових просторах вона має одну границю.
\end{remark}

\begin{definition}
    Напрямленість $g: T \to X$ називається \vocab{піднапрямленістю} напрямленості $f: S \to X$, якщо існує відображення $h: T \to S$, таке що $g = f \circ h$ і $\forall s_0 \in S$ $\exists t_0 \in T$: $\forall t \ge t_0$ $h(t) \ge s_0$.
\end{definition}

\begin{remark}
    На відміну від означення звичайної підпослідовності, означення піднапрямленості допускає, щоб область визначення піднапрямленості не була частиною області визначення напрямленості.
\end{remark}

\begin{definition}
    Частково упорядкована множина $X$ є \vocab{конфінальною} своїй підмножині $A$, якщо в $X$ не існує жодного елемента, що є наступним за усіма елементами множини $A$.
\end{definition}

\begin{example}
    Інтервал $(0, 1)$ є конфінальним множині $\left\{ \frac{n}{n + 1} \middle| n \in \NN \right\}$.
\end{example}

\begin{remark}
    Якщо $T \subset S$, а $h$~--- відображення вкладення, то друга умова еквівалентна конфінальності $T$ в $S$. І навпаки, для будь-якої конфінальної частини $T$ з $S$ і будь-якої напрямленості $f: S \to X$ звуження $f$ на $T$ є піднапрямленістю напрямленості $f$.
\end{remark}

\begin{theorem}[Бірхгофа]
    Нехай $A$~--- деяка підмножина довільного топологічного простору $X$. Тоді $x \in \closure A$ тоді і лише тоді, коли існує напрямленість в $A$, що збігається до точки $x$.
\end{theorem}

\begin{proof}
    \textbf{Необхідність.} Нехай $x \in \closure A$ і $\Omega_x$~--- напрямлена множина усіх околів точки $x$. Оскільки
    \begin{equation*}
        \forall U \in \Omega_x \quad A \cap U \ne \emptyset,
    \end{equation*}
    то, вибираючи по одній точці $x_U x \in A \cap U$, отримуємо напрямленість $\{x_U \mid U \in \Omega_x\}$ в $A$, що збігається до точки $x$. 

    \textbf{Достатність.} Нехай $\{x_s \mid s \in S\}$~--- напрямленість в $A$, що збігається в $X$ до точки $x$. Тоді за означенням границі напрямленості
    \begin{equation*}
        \forall U \in \Omega_x \quad \exists s_0 \in S: \quad \forall s \ge s_0 \quad x_s \in U.
    \end{equation*}
    Отже,
    \begin{equation*}
        A \cap U \ne \emptyset \implies x_0 \in \closure A. \qedhere
    \end{equation*}
\end{proof}

\section{Напрямленості та неперервність}

\begin{remark}
    Нагадаємо, що в просторах із першою аксіомою зліченності неперервність відображення $f$ в довільній точці $x_0$ була еквівалентною умові, що з $x_n \to x_0$ випливає $f(x_n) \to f(x_0)$. Перехід від послідовностей до напрямленостей дозволяє відмовитись від цієї умови.
\end{remark}

\begin{theorem}[критерій неперервності]
    Відображення $f: X \to Y$ є неперервним в точці $x_0$ тоді і лише тоді, коли для будь-якої напрямленості $\{x_s \mid s \in S\}$, що збігається до точки $x_0 \in X$ напрямленість $\{f(x_s) \mid s \in S\}$ збігається то точки $f(x_0) \in Y$.
\end{theorem}

\begin{proof}
    \textbf{Необхідність.} Нехай $f: X \to Y$ є неперервною в точці $x_0$ і $\{x_s \mid s \in S\}$~--- деяка напрямленість в $X$, що збігається до точки $x_0$. Нехай також $V_0$~--- довільний окіл точки $f(x_0)$ в $Y$. Тоді достатньо пересвідчитись, що напрямленість $\{f(x_s) \mid s \in S\}$ майже вся лежить в $V_0$. 
    
    Справді, оскільки відображення $f$ є неперервним в точці $x_0$, існує окіл $U_0$ точки $x_0$, такий що $f(U_0) \subset V_0$. Оскільки напрямленість $\{x_s \mid s \in S\}$ збігається до $x_0$, то знайдеться індекс $s_0 \in S$ такий, що при всіх $s \ge s_0$ $x_s \in U_0$. Отже, для всіх $s \ge s_0$ $f(x_s) \in V_0$, а це значить, що майже вся напрямленість $\{f(x_s) \mid s \in S\}$ лежить в $V_0$. 

    \textbf{Достатність.} Припустимо, що умови теореми виконуються, але відображення $f$ не є неперервним в точці $x_0$. Тоді існує такий окіл $V_0$ точки $f(x_0)$, що в будь-якому околі $U$ точки $x_0$ знайдеться точка $x_U$, образ $f(x_U)$ якої належить $Y \setminus V_0$. 
    
    Розглянемо напрямленість $\{x_U \mid U \in \Omega_{x_0}\}$, де $\Omega_{x_0}$~--- напрямлена множина усіх околів точки $x_0$. Очевидно, що ця напрямленість збігається до точки $x_0$. 
    
    Проте напрямленість $\{ f(x_U) \mid U \in \Omega_{x_0}\}$ не може збігатися до точки $f(x_0)$, оскільки в такому випадку вона майже вся лежала б в околі $V_0$. Отримане протиріччя доводить достатність.
\end{proof}

\section{Література}

\begin{enumerate}[label={[\arabic*]}]
\item \textbf{Колмогоров~А.~Н.}
Элементы теории функций и функционального анализа /
А.~Н.~Колмогоров, С.~В.~Фомин~---
М.: Наука, 1981 (стр.~18--21).
\item \textbf{Александрян~Р.~А., }
Общая топология /
Р.~А.~Александрян, Э.~А.~Мирзаханян~---
М.: Высшая школа, 1979 (стр.~91--98).
\item \textbf{Келли~Дж.}
Общая топология /
Дж.~Келли~---
М.: Наука, 1966 (стр.~91--118).
\end{enumerate}
