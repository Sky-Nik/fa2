\newcommand{\AdditionFunction}{+}
\newcommand{\MultiplicationFunction}{\cdot}

\chapter{Основні відомості про топологічні векторні простори}

\section{Простір із неперервними операціями}

\begin{definition}
    Лінійний простір $X$(дійсний чи комплексний) із заданою на ньому топологією $\tau$ називається \vocab{топологічним векторним простором}(ТВП), якщо топологія $\tau$ так погоджена з лінійною структурою, що відображення суми елементів і множення скаляра на елемент є неперервними по сукупності змінних.
\end{definition}

Розпишемо означення докладніше. Нехай $X$~--- топологічний векторний простір. Розглянемо функції $\AdditionFunction: X \times X \to X$ і $\MultiplicationFunction: \RR \times X \to X$. Узгодження топології лінійною структурою означає, что функції $+$ і $\cdot$ є неперервними як функції двох змінних.

\begin{theorem}
    Нехай $U$~--- відкрита множина у просторі $X$. Тоді
    \begin{enumerate}
        \item для будь-якого $x \in X$ множина $U + x$ є відкритою
        \item для будь-якого $\lambda \ne 0$ множина $\lambda U$ є відкритою.
    \end{enumerate}
\end{theorem}

\begin{proof}
    Зафіксуємо $x_2 = -x$ і скористаємося неперервністю функції~$\AdditionFunction(x_1, x_2) = x_1 + x_2$ по першій змінній при фіксованій другій змінній. Отже, функція~$f(x_1) = x_1 - x$ є неперервною по $x_1$, а $U + x$ є прообразом відкритої множини $U$ під дією функції~$f$. Отже, множина $U + x$ є відкритою.

    Друга властивість виводиться так само, але з використанням неперервності функції $g(x) = \lambda\inv x$.
\end{proof}

З теореми випливає, що околи будь-якого елемента $x \in X$ є множинами вигляду~$U + x$, де $U$~--- околи нуля. Відповідно, топологія $\tau$ однозначно визначається системою~$\frak R_0$ околів нуля. Тому інші властивості топології $\tau$ будуть формулюватися через околи нуля. Далі через $S_r$ позначатимемо множину $S_r = \{\lambda \in \RR: |\lambda| \le r\}$.

\section{Поглинаючі та урівноважені множини}

\begin{definition}
    Підмножина $A$ лінійного простору $X$ називається \vocab{поглинаючою}, якщо для будь-якого $x \in X$ існує таке $n \in \NN$, що $x \in t A$ для будь-якого $t > n$.
\end{definition}

\begin{definition}
    Підмножина $A \subset X$ називається \vocab{урівноваженою}, якщо для будь-якого скаляра $\lambda \in S_1$ виконане включення $\lambda A \subset A$.
\end{definition}

\begin{theorem}
    \label{th:tvs-zero-neighbourhood-properties}
    Властивості системи $\frak R_0$ околів нуля топологічного векторного простору~$X$:
    \begin{enumerate}
        \item Будь-який окіл нуля є поглинаючою множиною.
        \item Довільний окіл нуля містить урівноважений окіл нуля.
        \item Для кожного околу $U \in \frak R_0$ існує урівноважений окіл $V \in \frak R_0$ з $V + V \subset U$.
    \end{enumerate}
\end{theorem}

\begin{proof}
    \listhack
    \begin{enumerate}
        \item Зафіксуємо $x \in X$ і скористаємося неперервністю функції~$f(\lambda) = \lambda x$. Оскільки $f(0) = 0$, неперервність у точці $\lambda = 0$ означає, що для будь-якого $U \in \frak R_0$ існує таке $\epsilon > 0$, що $\lambda x \in U$ для будь-якого $\lambda \in S_\epsilon$. Увівши позначення $t = \lambda\inv$, одержимо, що $x \in t U$ для будь-якого $t \ge \epsilon\inv$.

        \item Нехай $U \in \frak R_0$. Через неперервність у точці $(0, 0)$ функції $\MultiplicationFunction\:(\lambda, x) = \lambda x$, існує таке $\epsilon > 0$ і такий окіл $W \in \frak R_0$, що $\lambda x \in U$ для будь-якого $\lambda \in S_\epsilon$ і будь-якого $x \in W$.
        
        Покладемо $V = \bigcup_{\lambda \in S_\epsilon} \lambda W$. Покажемо, що множина $V \supset U$ і є шуканаий урівноваженаий окіл нуля. З одного боку, $V \supset W$, отже, $V \in \frak R_0$. З іншого боку, для будь-якого $\lambda_0 \in S_1$ маємо $\lambda_0 S_\epsilon \subset S_\epsilon$, отже,
        \begin{equation*}
            \lambda_0 V = \bigcup_{\lambda \in S_\epsilon} \lambda_0 \lambda W = \bigcup_{\mu \in \lambda_0 S_\epsilon} \mu W \subset \bigcup_{\mu \in S_\epsilon} \mu W = V,
        \end{equation*}
        чим доведена урівноваженість околу $V$.
    
        \item Через неперервність у точці $(0, 0)$ функції~$\AdditionFunction(x_1, x_2) = x_1 + x_2$, для будь-якого околу $U \in \frak R_0$ існують околи $V_1, V_2 \in \frak R_0$ з $V_1 + V_2 \subset U$. Шуканий урівноважений окіл нуля $V$ виберемо за пунктом 2 так, щоб $V$ містився в околі $V_1 \cap V_2$. \qedhere
    \end{enumerate}
\end{proof}

\section{Узгодженість та віддільність}

\begin{theorem}
    Нехай система $\frak R_0$ околів нуля топології $\tau$ на лінійному просторі $X$ підкоряється умовам \cref{th:tvs-zero-neighbourhood-properties}, і для будь-якої точки $x \in X$ система околів $\frak R_x$ цієї точки отримується паралельним переносом $\frak R_0$ на вектор $x$. Тоді топологія $\tau$ узгоджується з лінійною структурою.
\end{theorem}

\begin{remark}
    Через урівноваженість умову $V + V \subset U$ пункту~3 \cref{th:tvs-zero-neighbourhood-properties} можна записувати у вигляді $V - V \subset U$.
\end{remark}

\begin{theorem}
    Для віддільності за Хаусдорфом топологічного векторного простору $X$ необхідно і достатньо, щоб система $\frak R_0$ околів нуля підкорялася такій умові: для будь-якого $x \ne 0$ існує окіл $U \in \frak R_0$, що не містить точку $x$.
\end{theorem}

\begin{proof}
    Нехай $x \ne y$. Тоді $x - y \ne 0$ та існує окіл $U \in \frak R_0$, який не містить $x - y$. Виберемо такий окіл $V \in \frak R_0$, що $V - V \subset U$. Тоді околи $x + V$ і $y + V$ не перетинаються: якщо існує точка $z$, яка лежить одночасно в $x + V$ і $y + V$, то $z - x \in V$, $z - y \in V$ і $x - y =(z - y) -(z - x) \in V - V \subset U$.
\end{proof}

\section{Література}

\begin{enumerate}[label={[\arabic*]}]
\item \textbf{Кадец~В.~М.}
Курс функционального анализа /
В.~М.~Кадец~---
Х.: ХНУ им.~В.~Н.~Каразина, 2006. (стр.~497--499).
\end{enumerate}
