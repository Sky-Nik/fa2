\newcommand{\CartesianProduct}[1]{\prod_{\gamma\in\Gamma}{#1}_\gamma}
\newcommand{\CartesianNeighbourhood}[1]{U_{K,\{V_\gamma\}_{\gamma\in K}}(#1)}

\chapter{Тихоновський добуток і тихоновська топологія}

\section{Декартів добуток як множина функцій}

Нехай $\Gamma$~--- не обов'язково скінченна індексна множина, кожному елементу $\gamma$ якої поставлено у відповідність деяку множину $X_\gamma$.

\begin{definition}
    \vocab{Декартовим добутком} множин $X_\gamma$ по $\gamma \in \Gamma$ називається множина $\CartesianProduct{X}$, яка складається із усіх таких функцій $x: \Gamma \to \bigcup_{\gamma \in \Gamma} X_\gamma$, що $\forall \gamma \in \Gamma$ $x(\gamma) \in X_\gamma$.
\end{definition}

\begin{remark}
    У частковому випадку, коли $\forall \gamma \in \Gamma$ $X_\gamma = X$, добуток складається з усіх функцій $x: \Gamma \to X$ і називається \vocab{декартовим степенем}~$X^\Gamma$.
\end{remark}

\begin{example}
    Простір Фреше~--- добуток $\prod_{n \in \NN} X_n$, де $X_n = \RR$. Отже, простір Фреше є степенем $\RR^\NN = \RR^{\aleph_0}$, елементами якого є зліченні послідовності $x = \{x_n\}_{n = 1}^\infty$ дійсних чисел~$x_n$.
\end{example}

\begin{example}
    Гільбертів куб~--- добуток $\prod_{n \in \NN} X_n$, де $X_n = I = [0, 1]$, тобто це простір $I^{\aleph_0}$.
\end{example}

\begin{example}
    Тихоновський куб~--- добуток $\CartesianProduct{X}$, де $\# \Gamma = \nu$, а $X_\gamma = I = [0, 1]$, тобто це простір $I^\nu$.
\end{example}

\begin{example}
    Канторів дисконтинуум ваги $\nu$~--- добуток $\CartesianProduct{X}$, де $\#\Gamma = \nu$, а множини $X_\gamma = D = \{0, 1\}$ (проста двокрапка), тобто це простір $D^\nu$.
\end{example}

\section{Проектори, тихоновська топологія і добуток}

\begin{definition}
    Відображення $P_\alpha: \CartesianProduct{X} \to X_\alpha$, що діє за правилом $P_\alpha (x) = x_\alpha$, $\forall \alpha \in \Gamma$, називається \vocab{координатним проектором}.
\end{definition}

\begin{definition}
    Нехай $X_\gamma$, $\gamma \in \Gamma$~--- топологічні простори. \vocab{Тихоновською топологією} на $\CartesianProduct{X}$ називається найслабкіша з топологій, в якій усі координатні проектори $P_\alpha (x)$, $\alpha \in \Gamma$ є неперервними.
\end{definition}

\begin{definition}
    Декартів добуток $\CartesianProduct{X}$, наділений тихоновською топологією, називається \vocab{тихоновським добутком}.
\end{definition}

\begin{remark}
    Очевидно, що координатні проектори розділяють точки добутку, тому за \cref{th:induced-topology-hausdorff-separability-criterion} тихоновський добуток хаусдорфових просторів є віддільним за Хаусдорфом.
\end{remark}

\begin{definition}
    Нехай $K$~--- скінчений набір індексів з $\Gamma$. Добуток~$A = \CartesianProduct{A}$, де $A_\gamma =  X_\gamma$ при $\gamma \notin K$, і $A_\gamma \subset X_\gamma$ при $\gamma \in K$ і $A_\gamma$~--- відкриті множини в топологіях $\tau_\gamma$, називається \vocab{відкритою циліндричною множиною} з основою~$\prod_{\gamma \in K} A_\gamma$.
\end{definition}

Запишемо тихоновську топологію як топологію, що породжена сім'єю відображень. Нехай $x \in \CartesianProduct{X}$, $K \subset \Gamma$~--- скінченна множина індексів, $V_\gamma \subset X_\gamma$, $\gamma \in K$~--- околи точок $x_\gamma$. Введемо позначення
\begin{equation*}
    \CartesianNeighbourhood{x} =
    \left\{ \gamma \in \CartesianProduct{X}: \gamma_\alpha \in V_\alpha, \forall \alpha \in K \right\}.
\end{equation*}

\begin{remark}
    Множина $\CartesianNeighbourhood{x}$ є відкритим циліндричним околом точки $x$ з основою $\prod_{\gamma \in K} V_\gamma$.
\end{remark}

\begin{theorem}[про базу околів точки в тихоновській топології]
    \label{th:tychonoff-topology-base}
    Множини $\CartesianNeighbourhood{x}$ утворюють у тихоновській топології базу околів точки~$x$.
\end{theorem}

\begin{exercise}
    Перевірте властивості бази.
\end{exercise}

\begin{proof}
    \dots
\end{proof}

\section{Тихоновська топологія і фільтри}

\begin{theorem}[критерій збіжності в тихоновському добутку]
    \label{th:tychonoff-topology-convergence-criterion}
    Фільтр $\frak F$ на $\CartesianProduct{X}$ збігається в тихоновській топології до елемента $x = \{x_\gamma\}_{\gamma \in \Gamma}$ тоді і тільки тоді, коли $x_\gamma = \lim_{\frak F} P_\gamma$, $\forall \gamma \in \Gamma$.
\end{theorem}

\begin{proof}
    \textbf{Необхідність.} Оскільки координатні проектори на $\CartesianProduct{X}$ є неперервними і $x = \lim \frak F$, то за \cref{th:filter-continuity} $\lim_{\frak F} P_\gamma = P_\gamma(x) = x_\gamma$.

    \textbf{Достатність.} Покажемо, що будь-який окіл $V$ точки $x$ належить фільтру $\frak F$. З огляду на те, що $\forall A \in \frak F$ $A \subset B \subset X \implies B \in \frak F$, достатньо розглянути відкритий циліндричний окіл точки $x$, який міститься в $V$. Отже, розглянемо відкритий циліндричний окіл $U = \CartesianProduct{V}$ точки $x$ з основою $\prod_{\gamma \in K} V_\gamma$, тобто $\CartesianNeighbourhood{x}$.
    
    Оскільки $\forall \gamma_0 \in K$ множина $V_{\gamma_0}$ є околом точки $x_{\gamma_0}$ в просторі $X_{\gamma_0}$ і $\lim_{\frak F} P_{\gamma_0} = x_{\gamma_0}$, то існує множина $A \in \frak F$ така, що $P_{\gamma_0}(A) \subset V_{\gamma_0}$, отже, $A \subset P_{\gamma_0}\inv (V_{\gamma_0})$, тому $P_{\gamma_0}\inv (V_{\gamma_0}) \in \frak F$. Таким чином, $\forall \gamma \in K$ $P_\gamma\inv (V_\gamma) \in \frak F$. Оскільки множина $K$ є скінченою, то $\bigcap_{\gamma \in K} P_\gamma\inv (V-\gamma) \in \frak F$.
    
    Оскільки
    \begin{equation*}
        \bigcap_{\gamma \in K} P_\gamma\inv (V_\gamma) \subset \CartesianNeighbourhood{x},
    \end{equation*}
    а $\CartesianNeighbourhood{x}$ утворюють в $\CartesianProduct{X}$ базу околів точки $x$ (\cref{th:tychonoff-topology-base}), то
    \begin{equation*}
        \bigcap_{\gamma \in K} P_\gamma\inv (V_\gamma) \subset U, \quad \forall U \in \Omega_x.
    \end{equation*}

    Тому, за четвертою аксіомою фільтра $U \in \frak F$.
\end{proof}

\begin{remark}
    Із \cref{th:tychonoff-topology-base} випливає, що послідовність $x_n = \{x_{n,\gamma}\}_{\gamma \in \Gamma}$ точок добутку $\CartesianProduct{X}$ топологічних просторів збігається до точки $x$ тоді і лише тоді, коли для кожного $\gamma_0 \in \Gamma$ послідовність $\{x_{\gamma_0, n}\}$ збігається в просторі $X_{\gamma_0}$ до точки $x_{\gamma_0}$. 
    
    Інакше кажучи, збіжність в тихоновській топології є покоординатною.
\end{remark}

\begin{theorem}[теорема Тихонова про добуток компактів]
    Тихоновський добуток $\CartesianProduct{X}$ будь-якої сім'ї непорожніх топологічних просторів $X_\gamma$, $\gamma \in \Gamma$ є компактним тоді і лише тоді, коли усі $X_\gamma$ є компактними.
\end{theorem}

\begin{proof}
    \textbf{Необхідність.} Нехай $X_\gamma$, $\gamma \in \Gamma$~--- довільна сім'я непорожніх просторів і їх тихоновський добуток $\CartesianProduct{X}$ є компактним. Оскільки кожна множина $X_\gamma$, $\gamma \in \Gamma$ є образом компактного простору $\CartesianProduct{X}$, отриманим за допомогою неперервного відображення $P_\gamma: X \to X_\gamma$, то простори $X_\gamma$, $\gamma \in \Gamma$ є компактними (неперервний образ компактного простору є компактним простором).

    \textbf{Достатність.} За критерієм компактності в термінах фільтрів, для того щоб простір був компактним, необхідно і достатньо, щоб кожний ультрафільтр на $X$ збігався. Нехай $\frak A$~--- ультрафільтр на $\CartesianProduct{X}$. Оскільки $X_\gamma$, $\gamma \in \Gamma$~--- компактні топологічні простори, то за критерієм компактності в термінах фільтрів $\forall \gamma \in \Gamma$ $\exists y_\gamma = \lim_{\frak A} P_\gamma$. Оскільки $P_\gamma$~--- неперервні відображення, то за \cref{th:tychonoff-topology-convergence-criterion} $y = \{y_\gamma\}_{\gamma \in \Gamma} = \lim \frak A$.
\end{proof}

\section{Література}

\begin{enumerate}[label={[\arabic*]}]
\item \textbf{Кадец~В.~М.}
Курс функционального анализа /
В.~М.~Кадец~---
Х.: ХНУ им.~В.~Н.~Каразина, 2006. (стр.~492--495).
\item \textbf{Александрян~Р.~А., }
Общая топология /
Р.~А.~Александрян, Э.~А.~Мирзаханян~---
М.: Высшая школа, 1979 (стр.~120--126, 230--234).
\end{enumerate}
