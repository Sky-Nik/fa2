\chapter{Ультрафільтри}

\section{Ультрафільтр як мажоранта} % лінійно упорядкованої сім'ї

\begin{lemma}
    \label{lem:linear-filter-set-upper-bound}
    Нехай $\frak M$~--- лінійно упорядкована непорожня сім'я фільтрів, заданих на множині $X$, тобто для довільни $\frak F_1, \frak F_2 \in \frak M$ або $\frak F_1 \subset \frak F_2$, або $\frak F_2 \subset \frak F_1$.  Тоді об'єднання $\frak F$ усіх фільтрів сім'ї $\frak M$ також буде фільтром на $X$.
\end{lemma}
\begin{proof}
    Перевіримо виконання аксіом фільтра для об'єднання сім'ї множин $\frak M$. Перші дві аксіоми є очевидними, а тому перевіримо останні дві. \medskip
    
    Перетин: якщо $A, B \in \frak F$, то знайдуться такі $\frak F_1, \frak F_2 \in \frak M$, що $A \in \frak F_1$, $B \in \frak F_2$. За умовою, один з фільтрів $\frak F_1$ і $\frak F_2$ мажорує інший. Нехай, без обмеження загальності, $\frak F_1 \subset \frak F_2$.  Тоді окрім множини $B$ йому належить і множина $A$, адже $A \in \frak F_1 \subset \frak F_2$. Оскільки $\frak F_2$~--- фільтр, то $A \cap B \in \frak F_2 \subset \frak F$, тобто сім'я $\frak F$ справді замкнена відносно (скінченного) перетину. \medskip
    
    Надмножина: якщо $A \in \frak F$ і $A \subset B \subset X$, то знайдеться такий $\frak F_1 \in \frak M$, що $A \in \frak F_1$, а тому $B \in \frak F_1$, як надмножина елеметну фільтра.  Як наслідок, $B \in \frak F$ і сім'я $\frak F$ виявляється замкненою відносно взяття надмножини.
\end{proof}

\begin{definition}
    \vocab{Ультрафільтром} на $X$ називається максимальний\footnote{для якого не існує більшого, але не обов'язково більший за кожен інший} за включенням фільтр на $X$.  Інакше кажучи, фільтр $\frak A$ на $X$ називається \emph{ультрафільтром}, якщо будь-який фільтр $\frak F$ на $X$, що мажорує $\frak A$, збігається з $\frak A$.
\end{definition}

\begin{theorem}
    \label{th:every-filter-has-supset-ultrafilter}
    Для кожного фільтра $\frak F$ на $X$ існує ультрафільтр, що його мажорує.
\end{theorem}
\begin{proof}
    Випливає з леми Цорна. Більш детально, необхідно розглянути частково упорядковану множину (сім'ю) фільтрів, що мажорують $\frak F$. \Cref{lem:linear-filter-set-upper-bound} показує, що довільний ланцюг (лінійно впорядкована підмножина) має верхню межу (також кажуть \emph{верхню грань} або \emph{мажоранту}). \medskip
    
    Тоді лема Цорна стверджує, що у нашій частково упорядкованій множині є максимальний елемент. З одного боку зрозуміло, що він буде ультрафільтром, адже немає іншого фільтра, що його мажорує, а з іншого~--- зо він буде мажорувати $\frak F$, адже усі елементи нашої частково упорядкованої множини за побудовою мажорують $\frak F$.
\end{proof}

\section{Властивості і критерій ультрафільтра}

\begin{lemma}
    \label{lem:ultrafilter-absorbs-intersections}
    Нехай $\frak A$~--- ультрафільтр, $A \subset X$ і всі елементи ультрафільтр перетинаються з $A$. Тоді $A \in \frak A$.
\end{lemma}
\begin{proof}
    Додавши до сім'ї множин $\frak A$ як елемент множину $A$ ми отримаємо центровану систему множин. Справді, для цього достатньо аби $\frak A$ була просто фільтром, а точніше замкненою відносно скінченного перетину. Тоді додавши у цей перетин, який є елементом $\frak A$, ще й $A$ можна просто скористатися умовою на непорожні перетини $A$ із елементами $\frak A$. Зрозуміло, що ці міркування працюють і для ультрафільтрів, адже кожен ультрафільтр є фільтром. Таким чином уся розширена система множин є центрованою. \medskip
    
    За \cref{th:supset-finite-intersection-property} звідси випливає, що знайдеться фільтр $\frak F$, який містить усі елементи нашої центрованої системи. Але тоді $\frak F \supset \frak A$, звідки випливає, що $\frak F = \frak A$, адже $\frak A$~--- ультрафільтр, і розширюватися уже нікуди. У той же час, за побудовою, $A \in \frak F$, тобто $A \in \frak A$.
\end{proof}

\begin{remark}
    Якщо зняти умову того, що $\frak A$~--- ультрафільтр, і сказати що він просто фільтр, то вийде, що його можна розширити до якогось $\frak A'$ щоб додати якийсь новий елемент $A$, за умови що цей $A$  перетинається із усіма елеменами $\frak A$.
\end{remark}

\begin{theorem}[критерій ультрафільтра]
    \label{th:ultrafilter-criterion}
    Для того, щоб фільтр $\frak A$ на $X$ був ультрафільтром, необхідно і достатньо, щоб для довільної множини $A \subset X$ або сама множина $A$, або її доповнення $X \setminus A$ належало фільтру $\frak A$.
\end{theorem}
\begin{proof}
    \textbf{Необхідність.} Нехай $\frak A$~--- ультрафільтр, і $X \setminus A \notin \frak A$. Тоді жодна множина $B \in \frak A$ не міститься цілком в $X \setminus A$, тобто будя-яка $B \in \frak A$ перетинається з $A$. Отже, за попередньою лемою, $A \in \frak A$. \medskip
    
    \textbf{Достатність.} Припустимо що $\frak A$~--- не ультрафільтр. Тоді існує фільтр $\frak F \supset \frak A$ і множина $A \in \frak F \setminus \frak A$. За побудовою, $A \notin \frak A$. З іншого боку, $X \setminus A$ не перетинається з $A$, $A \in \frak F$, отже $X \setminus A \notin \frak F$, а отже $X \setminus A \notin \frak A \subset \frak F$.
\end{proof}

\begin{corollary}
    \label{crl:ultrafilter-image-is-ultrafilter}
    Образ ультрафільтра є ультрафільтром.
\end{corollary}
\begin{proof}
    Нехай $f: X \to Y$ і $\frak A$~--- ультрафільтр на $X$. Розглянемо довільну множину $A \subset Y$. Тоді або $f\inv (A)$ або $f\inv (Y \setminus A) = Y \setminus f\inv (A)$ належить $\frak A$, отже $A \in f[\frak A]$ або $Y \setminus A \in f[\frak A]$.
\end{proof}

\section{Ультрафільтри, збіжність і компактність}

\begin{lemma}
    \label{lem:ultrafilter-limitpoint-is-its-limit}
    Нехай $\frak A$~--- ультрафільтр на хаусдорфовому топологічному просторі $X$ і $x \in \LIM(\frak A)$. Тоді $x = \lim \frak A$.
\end{lemma}
\begin{proof}
    Нехай $U \in \Omega_x$. Тоді за означенням граничної точки окіл $U$ перетинається зі всіма елементами ультрафільтра $\frak A$. За \cref{lem:ultrafilter-absorbs-intersections} $U \subset \frak A$.
\end{proof}

\begin{theorem}[критерій компактності у термінах ультрафільтрів]
    Для хаусдорфового топологічного простору $X$ нижченаведені умови є еквівалентними:
    \begin{itemize}
        \item $X$ компакт;
        \item кожен ультрафільтр на $X$ має граничну точку;
        \item кожен ультрафільтр на $X$ має границю.
    \end{itemize}
\end{theorem}
\begin{proof}
    $1 \implies 2$. Фільтр $\frak F$~--- центрована сім'я множин. Тим більше, центрованою буде сім'я замикань елементів фільтра. Отже, перетин $\LIM(\frak F)$ цих замикань не є порожнім. \medskip
    
    $2 \implies 1$. Нехай $\frak C$~--- довільна центромана система замкнених підмножин простору $X$. За \cref{th:supset-finite-intersection-property} існує фільтр $\frak F \supset \frak C$. Тоді
    \begin{equation*}
        \bigcap_{A \in \frak C} \closure A \supset \bigcap_{A \in \frak F} \closure A = \LIM(\frak F) \ne \emptyset.
    \end{equation*}
    
    $2 \implies 3$. За умовою кожен ультрафільтр має граничну точку, а за \cref{lem:ultrafilter-limitpoint-is-its-limit} ця точка буде його границею. \medskip
    
    $3 \implies 2$. Розглянемо довільний фільтр $\frak F$ на $X$ і виберемо (\cref{th:every-filter-has-supset-ultrafilter}) ультрафільтр $\frak A \supset \frak F$. За умовою ультрафільтр $\frak A$ має границю $x \in X$. Згідно твердження 3) \cref{th:filters-limits-limitpoitns} точка $x$ є граничною точкою фільтра $\frak F$.
\end{proof}

\begin{corollary}
    Нехай $\frak A$~--- ультрафільтр на $E$, $X$~--- топологічний простір і образ функції $f: E \to X$ лежить в деякому компакті $K \subset X$. Тоді існує $\lim_{\frak A} f$.
\end{corollary}
\begin{proof}
    Розглянемо $f$ як функцію, що діє з $E$ в $K$. Оскільки (\cref{crl:ultrafilter-image-is-ultrafilter}) $f[\frak A]$ є ультрафільтром на компакті $K$, то існує $\lim f[\frak A]$. Отже, за означенням, $\lim_{\frak A} = \lim f[\frak A]$.
\end{proof}

\section{Література}

\begin{enumerate}[label={[\arabic*]}]
\item \textbf{Кадец~В.~М.}
Курс функционального анализа /
В.~М.~Кадец~---
Х.: ХНУ им.~В.~Н.~Каразина, 2006. (стр.~484--490).
\end{enumerate}
