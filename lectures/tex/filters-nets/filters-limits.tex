\chapter{Фільтри і збіжність}

\section{Границі і граничні точки фільтрів}

\begin{definition}
    Нехай на множині $X$ задані фільтри $\frak F_1$ і $\frak F_2$. Говорять, що $\frak F_1$ \vocab{мажорує} $\frak F_2$, якщо $\frak F_2 \subset \frak F_1$, тобто кожний елемент фільтра $\frak F_2$ є водночас і елементом фільтра $\frak F_1$.
\end{definition}

\begin{example}
    Нехай $\{x_n\}_{n \in \NN}$~--- послідовність в $X$, а $\{x_{n_k}\}_{k \in \NN}$~--- її підпослідовність. Тоді фільтр $\frak F_{\{x_{n_k}\}}$ асоційований з підпослідовністю, мажорує фільтр $\frak F_{\{x_n\}}$, асоційований з самою послідовністю. 
    
    Дійсно, нехай $A \in \frak F_{\{x_n\}}$. Тоді існує таке $N \in \NN$, що $\{x_n\}_{n = N}^\infty \subset A$. Але тоді й $\{x_{n_k}\}_{k = N}^\infty \subset A$, тобто $A \in \frak F_{\{x_{n_k}\}}$.
\end{example}

\begin{definition}
    Нехай $X$~--- топологічний простір, $\frak F$~--- фільтр на $X$. Точка $x \in X$ називається \vocab{границею фільтра $\frak F$} (цей факт позначається як $x = \lim \frak F$), якщо $\frak F$ мажорує фільтр околів точки $x$. Іншими словами, $x = \lim \frak F$, якщо кожний окіл точки $x$ належить фільтру $\frak F$.
\end{definition}

\begin{definition}
    Точка $x \in X$ називається \vocab{граничною точкою фільтра $\frak F$}, якщо кожний окіл точки $x$ перетинається з усіма елементами фільтра $\frak F$. Множина усіх граничних точок фільтра називається $\LIM \frak F$.
\end{definition}

\begin{example}
    Нехай $\{x_n\}_{n \in \NN}$~--- послідовність в топологічному просторі $X$. Тоді $x = \lim \frak F_{\{x_n\}} = \lim_{ n \to \infty} x_n$, а $x \in \LIM \frak F_{\{x_n\}}$ збігається з множиною граничних точок послідовності $\{x_n\}_{n \in \NN}$.
\end{example}

\begin{theorem}
    \label{th:filterbase-limits-limitpoitns}
    Нехай $\frak F$~--- фільтр на топологічному просторі $X$, $\frak D$~--- % деяка
    база фільтра $\frak F$. Тоді
    \begin{enumerate}
        \item $x = \lim \frak F \iff \forall U \in \Omega_x$ $\exists A \in \frak D$: $A \subset U$;
        
        \item $x = \lim \frak F \implies x \in \LIM \frak F$. Якщо до того ж $X$~--- хаусдорфів простір, то у фільтра $\frak F$ немає інших граничних точок. Зокрема, якщо у фільтра в хаусдорфовому просторі є границя, то ця границя є єдиною;
        
        \item множина $\LIM \frak F$ збігається з перетином замикань усіх елементів фільтра~$\frak F$.
    \end{enumerate}
\end{theorem}

\begin{proof}
    \listhack
    \begin{enumerate}
        \item $x = \lim \frak F$ $\iff$ $\forall U \in \Omega_x$ $U \in \frak F$ $\iff$ $\forall U \in \frak F$ $\exists A \in \frak D$: $A \subset U$.

        \item $x = \lim \frak F, U \in \Omega_x$ $\implies$ $U \in \frak F$ $\implies$ $\forall A \in \frak F$ $A \cap U \ne \emptyset$ $\implies$ $x \in \LIM \frak F$;
        
        $x \in \LIM \frak F$ $\implies$ $\forall U \in \frak F, V \in \Omega_y$ $U \cap V \ne \emptyset$ $\implies$ $x = y$ (простір хаусдорфів).

        \item $x = \LIM \frak F$ $\iff$ $\forall A \in \frak F, U \in \Omega_x$ $A \cap U \ne \emptyset$ $\iff$ $\forall A \in \frak F$ $x \in \closure A$. \qedhere
    \end{enumerate}
\end{proof}

\begin{theorem}
    \label{th:filters-limits-limitpoitns}
    Нехай $\frak F_1$, $\frak F_2$~--- фільтри на топологічному просторі $X$ і $\frak F_1 \subset \frak F_2$. Тоді:
    \begin{enumerate}
        \item $x = \lim \frak F_1 \implies x = \lim \frak F_2$;
        \item $x \in \LIM \frak F_2 \implies x \in \LIM \frak F_1$;
        \item $x = \lim \frak F_2 \implies x \in \LIM \frak F_1$.
    \end{enumerate}
\end{theorem}

\begin{proof}
    \listhack
    \begin{enumerate}
        \item $\frak F_1$ мажорує фільтр $\frak M_x$ околів точки $x$, $\frak F_1 \subset \frak F_2 \implies \frak M_x \subset \frak F_2$.

        \item Оскільки при збільшенні сім'ї множин її перетин зменшується, то
        \begin{equation*}
            \LIM \frak F_2 = \bigcap_{A \in \frak F_2} \closure A \subset \bigcap_{A \in \frak F_1} \closure A = \LIM \frak F_1.
        \end{equation*}

        \item $x = \lim \frak F_2 \implies x \in \LIM \frak F_2 \implies x \in \LIM \frak F_1$. \qedhere
    \end{enumerate}
\end{proof}

\section{Границя функції по фільтру}

\begin{definition}
    Нехай $X$~--- множина, $Y$~--- топологічний простір, $\frak F$~--- фільтр на $X$. Точка $y \in Y$ називається \vocab{границею функції $f: X \to Y$ по фільтру $\frak F$} (цей факт позначається як $y = \lim_{\frak F} f$, якщо $y = \lim f[\frak F]$. Іншими словами, $y = \lim f [\frak F]$, якщо для довільного околу $U$ точки $y$ існує такий елемент $A \in \frak F$, що $f(A) \subset U$.
\end{definition}

\begin{definition}
    Точка $y \in Y$ називається \vocab{граничною точкою функції $f: X \to Y$ по фільтру $\frak F$}, якщо $y \in \LIM f[\frak F]$, тобто якщо довільний окіл точки $y$ перетинається з образами усіх елементів фільтра $\frak F$.
\end{definition}

\begin{example}
    Нехай $X$~--- топологічний простір, $f: \NN \to X$ і $\frak F$~--- фільтр Фреше на $\NN$. Тоді $\lim_{\frak F} f = \lim_{n \to \infty} f(n)$.
\end{example}

\begin{theorem}
    \label{th:filter-continuity}
    Нехай $X$ і $Y$~--- топологічні простори, $\frak F$~--- фільтр на $X$, $x = \lim \frak F$ і $f: X \to Y$~--- неперервна функція. Тоді $f(x) = \lim_{\frak F} f$.
\end{theorem}

\begin{proof}
    Нехай $U$~--- довільний окіл точки $f(x)$. Тоді існує окіл $V$ точки $X$, для якого $f(V) \subset U$. Умова $x = \lim \frak F$ означає, що $V \in \frak F$. Інакше кажучи, для довільного околу $U$ точки $f(x)$ ми знайшли шуканий елемент $V \in \frak F$: $f(V) \subset U$.
\end{proof}

\section{Література}

\begin{enumerate}[label={[\arabic*]}]
\item \textbf{Кадец~В.~М.}
Курс функционального анализа /
В.~М.~Кадец~---
Х.: ХНУ им.~В.~Н.~Каразина, 2006. (стр.~484--488).
\end{enumerate}
