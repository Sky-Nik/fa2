\chapter{Зв'язок між фільтрами і напрямленностями}

Фільтри і напрямленості в одній множині~$X$ приводять до еквівалентних теорій збіжності. З одного боку, як показано раніше, кожній напрямленості~$\{x_s \mid s \in S\}$ в множині~$X$ відповідає асоційований з нею фільтр в~$X$. З іншого боку, має місце така теорема.

\section{Відповідність між фільтрами і напрямленостями}

\begin{theorem}
    Нехай~$\frak F$~--- довільний фільтр в множині~$X$. Тоді в цій множині існує напрямленість~$\{x_s \mid s \in S\}$ така, що асоційований з нею фільтр збігається з фільтром~$\frak F$.
\end{theorem}

\begin{proof}
    Розглянемо множину усіх можливих пар~$s = (x, M)$, де~$M \in \frak F$, а~$x \in M$. Введемо у множині таких пар~$S$ частковий передпорядок, поклавши~$(x, M) \le (y, N)$, якщо~$M \supset N$. Таким чином, $S$~--- напрямлена множина. 

    Задамо відображення~$f: S \to X$, поклавши
    \begin{equation*}
        f(s) = x, \quad \forall s = (x, M) \in S.
    \end{equation*}

    Нехай~$s = (x, M)$~--- довільний елемент з~$S$, а~$\hat M_s = \{f(t) \mid t \ge s\}$. За означенням фільтра~$\hat{\frak F}$, асоційованого з напрямленістю~$f: S \to X$, система підмножин~$\hat M_s$, де~$s$ пробігає усі значення в множині~$S$, утворює базу~$\hat \beta$ фільтра~$\hat{\frak F}$. 

    Покажемо, що фільтр~$\hat{\frak F}$, асоційований з побудованою напрямленістю~$f: S \to X$, збігається з фільтром~$\frak F$, тобто
    \begin{equation*}
        \hat{\frak F} \le \frak F \quad \text{і} \quad \frak F \le \hat{\frak F}.
    \end{equation*}

    \begin{enumerate}
        \item Для того щоб довести, що~$\hat{\frak F} \le \frak F$, треба показати, що
        \begin{equation*}
            \forall \hat M_s \in \hat \beta \quad
            \exists M \in \frak F: \quad
            M \subset \hat M_s.
        \end{equation*}

        Насправді має місце більш сильний факт:
        \begin{equation*}
            \forall \hat M_s \in \hat \beta \quad
            \exists M \in \frak F: \quad
            M = \hat M_s.
        \end{equation*}

        Дійсно, нехай~$y \in \hat M_s$, тобто
        \begin{equation*}
            \exists t = (z, N) \ge (x, M) = s: \quad y = f(t),
        \end{equation*}
        тоді
        \begin{equation*}
            y = z \in N \subset M \implies \hat M_s \subset M.
        \end{equation*}

        Тепер візьмемо довільну точку~$z \in M$ і покладемо~$t^\star = (z, M)$. Оскільки~$t^\star \ge s = (x, M)$, то~$f(t^\star) = z \in \hat M_s$, тобто~$M \subset \hat M_s$. Таким чином, $M = \hat M_s$.

        \item Покажемо, що має місце і обернене твердження: $\frak F \le \hat{\frak F}$. Для цього пересвідчимось, що
        \begin{equation*}
            \forall M \in \frak F \quad
            \exists \hat M_s \in \hat \beta: \quad
            \hat M_s = M.
        \end{equation*}

        Нехай~$x^\star$~--- довільний елемент з~$M$ і~$s^\star = (x^\star, M)$. Повторимо міркування, наведені вище. 

        Нехай~$s^\star = (x^\star, M)$~--- довільний елемент з~$S$, а~$y^\star \in \hat M_{s^\star}$, тобто
        \begin{equation*}
            \exists t^\star = (z^\star, N) \ge (x^\star, M) = s^\star: \quad y = f(t^\star),
        \end{equation*}
        тоді
        \begin{equation*}
            y^\star = x^\star \in N \subset M \implies \hat M_{s^\star} \subset M.
        \end{equation*}

        Тепер візьмемо довільну точку~$z^\star \in M$ і покладемо~$t^\star = (z^\star, M)$. Оскільки~$t^\star \ge s^\star = (x, M)$, то~$f(t^\star) = z^\star \in \hat M_{s^\star}$, тобто~$M \subset \hat M_{s^\star}$.
    \end{enumerate}
    
    Таким чином, $\frak F = \hat{\frak F}$.
\end{proof}

\section{Границі і граничні точки фільтрів і напрямленостей}

\begin{theorem}
    Нехай~$\xi = \{x_s \mid s \in S\}$~--- напрямленість в топологічному просторі~$X$, а~$\frak F$~--- асоційований з нею фільтр. Тоді кожна границя (відповідно, гранична точка) напрямленості~$\xi$ є границею (відповідно, граничною точкою) фільтра~$\frak F$, і навпаки.
\end{theorem}

\begin{proof}
    \textbf{Необхідність.} Нехай~$x_0 = \lim_S x_s$. Покажемо, що фільтр~$\frak F$ мажорує фільтр~$\frak F_{x_0}$ околів точки~$x_0$, тобто~$x_0 = \lim \frak F$. Нехай~$U_0$~--- довільний елемент~$\frak F_{x_0}$, тобто деякий окіл точки~$x_0$ в просторі~$X$. Тоді
    \begin{equation*}
        x_0 = \lim_S x_s \implies
        \exists s_0 \in S: M_{s_0} = \{ x_s \mid s \ge s_0 \} \subset U_0.
    \end{equation*}

    Оскільки~$M_{s_0}$~--- елемент бази фільтра, асоційованого з напрямленістю~$\xi$, то~$M_{s_0} \subset U_0 \implies U_0 \in \frak F$. Отже, 
    \begin{equation*}
        \frak F \supset \frak F_{x_0} \implies x_0 = \lim \frak F.
    \end{equation*}

    \textbf{Достатність.} Нехай~$x_0 = \lim \frak F$. Отже, будь-який окіл~$U_0$ точки~$x_0$ є елементом фільтра~$\frak F$. За означенням, множини~$M_s = \{x_t \mid t \ge s\}$ утворюють базу фільтра~$\frak F$, тому~$\exists M_{s_0} \subset U_0$. Отже, для будь-якого околу~$U_0$ точки~$x_0$ існує~$s_0 \in S$, такий що усі члени напрямленості~$\xi$ при~$s \ge s_0$ лежать в~$U_0$, тобто~$x_0 = \lim_S x_s$.
\end{proof}

\section{Універсальні напрямленності і ультрафільтри}

\begin{definition}
    Напрямленість~$\{x_s \mid s\in S\}$ в множині~$X$ називається \vocab{універсальною}, якщо для будь-якої підмножини~$M \subset X$ вона або майже вся лежить в~$M$, або майже вся лежить в~$X \setminus M$.
\end{definition}

\begin{theorem}
    \label{th:universal-net-is-ultrafilter}
    Напрямленість в~$X$ є універсальною тоді і лише тоді, коли асоційований з нею фільтр є ультрафільтром.
\end{theorem}

\begin{proof}
    \textbf{Необхідність.} Нехай~$\xi = \{x_s \mid s \in S\}$~--- універсальна напрямленість в~$X$, $\frak F$~--- асоційований з нею фільтр, а~$M$~--- довільна підмножина з~$X$. Покажемо, що або~$M$, або~$X \setminus M$ належать фільтру~$\frak F$, звідки випливає, що~$\frak F$~--- ультрафільтр (\cref{th:ultrafilter-criterion}). 
    
    Оскільки~$\xi = \{x_s \mid s \in S\}$~--- універсальна напрямленість в~$X$, то вона майже вся лежить або в~$M$, або в~$X \setminus M$, тобто існує індекс~$s_0 \in S$, такий що множина~$M_{s_0} = \{x_s \mid s \ge s_0\}$ цілком міститься або в~$M$, або в~$X \setminus M$. Але оскільки~$M_{s_0}$ належить базі фільтра~$\frak F$, то або~$M$, або~$X \setminus M$ містить~$M_{s_0}$, тобто є елементом фільтра~$\frak F$. 
    
    \textbf{Достатність.} Нехай~$\frak F$~--- ультрафільтр, а~$M$~--- довільна підмножина з~$X$. Доведемо, що~$\xi = \{x_s \mid s \in S\}$ майже вся лежить або в~$M$, або в~$X \setminus M$. Оскільки або~$M$, або~$X \setminus M$ є елементом фільтра~$\frak F$, то одна з цих множин повинна цілком містити деяку множину з бази фільтра~$\frak F$ тобто деяку множину~$M_{s_0}$. Це значить, що~$\xi = \{x_s \mid s \in S\}$ майже вся лежить або в~$M$, або в~$X \setminus M$. Отже, $\xi$~--- універсальна напрямленість в~$X$.
\end{proof}

\section{Література}

\begin{enumerate}[label={[\arabic*]}]
\item \textbf{Александрян~Р.~А.}
Общая топология /
Р.~А.~Александрян, Э.~А.~Мирзаханян~---
М.: Высшая школа, 1979 (стр.~101--113).
\end{enumerate}
