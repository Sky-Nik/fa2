\chapter{Фільтри}

Окрім збіжності напрямленостей, існує ще один вид узагальненої збіжності~--- збіжність фільтрів. Ця ідея базується на альтернативному означенні збіжної послідовності: послідовність $x_n$ називається збіжною до точки $x_0$, якщо для будь-якого околу $U$ цієї точки доповнення до прообразу $f\inv (U)$ є скінченною підмножиною з $\NN$, де $f: \NN \to X$~--- відображення, що задає послідовність. Якщо множину $\NN$ замінити абстрактним простором $E$, в якому виділено сім'ю підмножин $F$, що має певні загальні властивості, то можна дати розумне означення узагальненої збіжності.

\section{Фільтри}

\begin{definition}
    Сім'я підмножин $\frak F$ множини $X$ називається \vocab{фільтром} на $X$, якщо:
    \begin{enumerate}
        \item Сім'я $\frak F$ непорожня.
        \item $\emptyset \notin \frak F$.
        \item Якщо $A, B \in \frak F$, то $A \cap B \in \frak F$.
        \item Якщо $A \in \frak F$, $A \subset B \subset X$, то $B \in \frak F$.
    \end{enumerate}
\end{definition}

\begin{corollary}
    $X \in \frak F$.
\end{corollary}

\begin{corollary}
    $A_1, A_2, \dots, A_n \in \frak F \implies \bigcap_{i = 1}^n A_i \in \frak F$.
\end{corollary}

\begin{corollary}
    $A_1, A_2, \dots, A_n \in \frak F \implies \bigcap_{i = 1}^n A_i \ne \emptyset$.
\end{corollary}

\begin{example}
    Система $\Omega_x$ усіх околів точки $x$ у топологічному просторі $X$ є фільтром.
\end{example}

\section{Бази фільтрів}

\begin{definition}
    Непорожня сім'я підмножин $\frak D$ множини $X$ називається \vocab{базою фільтра}, якщо:
    \begin{enumerate}
        \item $\emptyset \notin \frak D$;
        \item $\forall A, B \in \frak D$ $\exists C \in \frak D$: $C \subset A \cap B$.
    \end{enumerate}
\end{definition}

\begin{definition}
    Нехай $\frak D$~--- база фільтра. Фільтром, що \vocab{породжений} базою $\frak D$, називається сім'я $\frak F$ усіх множин $A \subset X$, що містять як підмножину хоча б один елемент бази $\frak D$.
\end{definition}

\begin{exercise}
    Довести, що фільтр, породжений базою, дійсно є фільтром.
\end{exercise}
\begin{proof}
    Перевіримо аксіоми фільтра. Перші дві аксіоми очевидні, адже фільтр містить як підмножину свою непорожню базу, і порожня множина не є надмножиною ніякоїмножини окрім порожньої, а база її не містить. Перевіримо тепер другі дві аксіоми. 
    
    \textbf{Перетин:} якщо $A, B \in \frak F$ то $\exists C, D \in \frak D$ такі, що $C \subset A$ і $D \subset B$, а тоді $\exists E \in \frak D$: $E \subset C \cap D$, і тому $E \subset A \cap B$ і, як наслідок, $A \cap B \in \frak F$. 
    
    \textbf{Надмножина:} якщо $A \in \frak F$ то $\exists B \in \frak D$: $B \subset A$, а тому $B 
    \subset C$ для усіх $C \supset A$ і, як наслідок, $C \in \frak F$.
\end{proof}

\begin{example}
    Якщо $X$~--- топологічний простір, $x_0 \in X$, $\frak D$~--- сукупність усіх відкритих множин, що містять $x_0$, то фільтр, породжений базою $\frak D$, є фільтром $\frak M_{x_0}$, що складається з усіх околів точки $x_0$.
\end{example}

\begin{definition}
    Нехай $\{x_n\}_{n = 1}^\infty$~--- послідовність елементів множини $X$. Тоді сім'я $\frak D_{\{x_n\}}$ ``хвостів'' послідовності $\{x_n\}_{n = N}^\infty$ є базою фільтра. Фільтр $\frak F_{\{x_n\}}$, породжений базою $\frak D_{\{x_n\}}$, називається фільтром, \vocab{асоційованим} з послідовністю $\{x_n\}_{n = 1}^\infty$.
\end{definition}

\section{Образи фільтрів і баз фільтрів}

\begin{theorem}
    \label{th:filterbase-image-filterbase}
    Нехай $X, Y$~--- множини, $f: X \to Y$~--- функція, $\frak D$~--- база фільтра в $X$. Тоді сім'я $f(\frak D)$ усіх множин вигляду $f(A)$, $A \in \frak D$ є базою фільтра в $Y$.
\end{theorem}

\begin{proof}
    Виконання першої аксіоми бази фільтра є очевидним, адже образ непорожньої множини~--- непорожня множина. Нехай $f(A), f(B)$~--- довільні елементи сім'ї $f(\frak D)$, $A, B \in D$. За другою аксіомою існує таке $C \in \frak D$, що $C \subset A \cap B$. Тоді $f(C) \subset f(A) \cap f(B)$. Отже друга аксіома виконується і для сім'ї $f(\frak D)$.
\end{proof}

\begin{corollary}
    \label{th:filter-image-filterbase}
    Якщо $\frak F$~--- фільтр на $X$, то $f(\frak F)$~--- база фільтра в $Y$.
\end{corollary}

\begin{definition}
    \vocab{Образом фільтра $\frak F$} при відображенні $f$ називається фільтр $f[\frak F]$, породжений базою $f(\frak F)$, тобто
    \begin{equation*}
        A \in f[\frak F] \iff f\inv (A) \in \frak F.
    \end{equation*}
\end{definition}

\begin{theorem}
    \label{th:supset-finite-intersection-property}
    Нехай $\frak C \subset 2^X$~--- непорожня сім'я множин. Тоді аби існував фільтр~$\frak F \supset \frak C$ (тобто такий, що усі елементи сім'ї $\frak C$ є елементами фільтра $\frak F$) необхідно і достатньо, щоб $\frak C$ була центрованою.
\end{theorem}

\begin{proof}
    \textbf{Необхідність.} Якщо $\frak F$~--- фільтр і $\frak F \supset \frak C$, то будь-який скінчений набір $A_1, A_2, \dots, A_n$ елементів сім'ї $\frak C$ буде складатися з елементів фільтра $\frak F$. Отже, 
    \begin{equation*}
        \bigcap_{i = 1}^n A_i \ne \emptyset.
    \end{equation*}

    \textbf{Достатність.} Нехай $\frak C$~--- центрована сім'я. Тоді сім'я $\frak D$ усіх множин вигляду
    \begin{equation*}
        \bigcap_{i = 1}^n A_i, \quad n \in \NN, \quad A_1, A_2, \dots, A_n \in \frak C
    \end{equation*}
    буде базою фільтра. Як фільтр $\frak F$ треба взяти фільтр, породжений базою $\frak D$.
\end{proof}

\section{Фільтри, породжені базою}

\begin{definition}
    Нехай $\frak F$~--- фільтр на $X$. Сім'я множин $\frak D$ називається \vocab{базою фільтра $\frak F$}, якщо $\frak D$ база фільтра і фільтр, породжений базою $\frak D$, збігається з $\frak F$.
\end{definition}

\begin{theorem}[критерій бази фільтра $\frak F$]
    \label{th:filterbase-criterion}
    Для того щоб $\frak D$ була базою фільтра $\frak F$, необхідно і достатньо, щоб виконувалися дві умови:
    \begin{enumerate}
        \item $\frak D \subset \frak F$;
        \item $\forall A \in \frak F$ $\exists B \in \frak D$: $B \subset A$.
    \end{enumerate}
\end{theorem}

\begin{exercise}
    Доведіть цю теорему.
\end{exercise}
\begin{proof}
    \textbf{Необхідність.} Без першої з цих умов $\frak F$ замалий щоб бути породженим базою $\frak D$ (не містить якоїсь із множин бази), а без другої~--- завеликий (містить якусь множину $A$, яка не є надмножиною жодної із множин бази). 
    
    \textbf{Достатність.} Зрозуміло, що за таких умов усі множини фільтра $\frak F$ будуть належати фільтру, породженому базою $\frak D$. Відповідно, питання полягає у тому, щоб у породженому фільтрі не опинилося зайвих множин. Розглянемо якусь множинк $A$ з нього. Вона є надмножиною якогось елемента $B$ бази. З першої умови випливає, що фільтр $\frak F$ також містить $B$. Тоді він містить і множину $A$ як надмножину $B$. Отже, породжений базою $\frak D$ фільтр не може бути ані більшим ані меншим від фільтра $\frak F$, і теорема доведена.
\end{proof}

\begin{definition}
    Нехай $F$~--- фільтр на $X$ і $A \subset X$. \vocab{Слідом фільтра $\frak F$ на $A$} називається сім'я підмножин $\frak F_A = \{ A \cap B \mid B \in \frak F\}$.
\end{definition}

\begin{theorem}
    \label{th:filter-restriction-criterion}
    Для того щоб слід $\frak F_A$ фільтра $\frak F$ був фільтром на $A$, необхідно і достатньо, щоб усі перетини $A \cap B$, $B \in \frak F$ були непорожніми.
\end{theorem}

\begin{exercise}
    Доведіть цю теорему.
\end{exercise}
\begin{proof}
    \textbf{Необхідність.} Якщо $A \cap B$ порожня для якогось $B \in \frak F$, то $\frak F_A$ містить $A \cap B = \emptyset$, тобто точно не є фільтром, адже не задовольняє першу аксіому. 
    
    \textbf{Достатність.} Перевіримо аксіоми фільтра. Перші дві аксіоми очевидні. Перевіримо другі дві аксіоми. 
    
    \textbf{Перетин.} Якщо $B, C \in \frak F_A$, то $\exists D, E \in \frak F$: $B = D \cap A$, $C = E \cap A$. Тоді $B \cap C = (D \cap A) \cap (E \cap A) = (D \cap E) \cap A \in \frak F$. 
    
    \textbf{Надмножина.} Якщо $B \in \frak F_A$, $C \supset B$, $C \subset A$, то $\exists D \in \frak F$: $B = D \cap A$. Тоді $C \cup D \supset D$, тобто $C \cup D \in \frak F$, а тому $(C \cup D) \cap A = (C \cap A) \cup (D \cap A) = C \cup B = C \in \frak F_A$.
\end{proof}

\begin{corollary}
    Зокрема, якщо $A \in \frak F$, то $\frak F_A$~--- фільтр.
\end{corollary}

\section{Література}

\begin{enumerate}[label={[\arabic*]}]
\item \textbf{Александрян~Р.~А.}
Общая топология /
Р.~А.~Александрян, Э.~А.~Мирзаханян~---
М.: Высшая школа, 1979 (стр.~99--102).
\item \textbf{Кадец~В.~М.}
Курс функционального анализа /
В.~М.~Кадец~---
Х.: ХНУ им.~В.~Н.~Каразина, 2006. (стр.~481--484).
\end{enumerate}
