% This partially originates from Evan Chen, namely
% https://github.com/vEnhance/dotfiles/blob/master/texmf/tex/latex/evan/evan.sty
% Please, see the license note there.
\usepackage[utf8]{inputenc}
\usepackage[T1,T2A]{fontenc}
\usepackage[english,russian,ukrainian]{babel}
\usepackage{amsfonts}
\usepackage{amsmath}
\usepackage{amssymb}
\usepackage[usenames,svgnames,dvipsnames]{xcolor}
\usepackage[unicode=true]{hyperref}
\hypersetup{
    colorlinks,
    linkcolor={red!50!black},
    citecolor={green!50!black},
    urlcolor={blue!80!black}
}
\usepackage[nameinlink,ukrainian]{cleveref}
\usepackage{graphicx}
\usepackage{enumitem}


\newcommand{\NN}{\mathbb{N}}
\newcommand{\ZZ}{\mathbb{Z}}
\newcommand{\QQ}{\mathbb{Q}}
\newcommand{\RR}{\mathbb{R}}
\newcommand{\CC}{\mathbb{C}}
\newcommand{\LL}{\mathcal{L}}
\newcommand{\inv}{^{-1}}
\newcommand{\conjugate}[1]{#1^\star}
\newcommand{\dconjugate}[1]{#1^{\star\star}}
\newcommand{\transpose}[1]{#1^\intercal}
\newcommand{\weakto}{\rightharpoonup}
\newcommand{\closure}[1]{\overline{#1}}
\renewcommand{\emptyset}{\varnothing}
\renewcommand{\epsilon}{\varepsilon}
\renewcommand{\phi}{\varphi}
\renewcommand{\frak}[1]{\mathfrak{#1}}
\DeclareMathOperator{\sign}{sgn}
\DeclareMathOperator{\spanning}{span}
\DeclareMathOperator{\kernel}{ker}
\DeclareMathOperator{\Closure}{cl}
\DeclareMathOperator{\Interior}{int}
\DeclareMathOperator*{\Bigcup}{\bigcup}
\DeclareMathOperator*{\Bigcap}{\bigcap}
\DeclareMathOperator{\LIM}{LIM}


\usepackage{amsthm}
\usepackage{thmtools}
\usepackage[framemethod=TikZ]{mdframed}
\theoremstyle{definition}

\mdfdefinestyle{mdbluebox}{%
	roundcorner = 10pt,
	linewidth=1pt,
	skipabove=12pt,
	innerbottommargin=9pt,
	skipbelow=2pt,
	nobreak=true,
	linecolor=blue,
	backgroundcolor=TealBlue!5,
}

\declaretheoremstyle[
	headfont=\sffamily\bfseries\color{MidnightBlue},
	mdframed={style=mdbluebox},
	headpunct={\\[3pt]},
	postheadspace={0pt}
]{thmbluebox}

\declaretheorem[style=thmbluebox,name=Теорема,numberwithin=chapter]{theorem}
\declaretheorem[style=thmbluebox,name=Лема,numberwithin=chapter]{lemma}
\declaretheorem[style=thmbluebox,name=Наслідок,numberwithin=chapter]{corollary}
\declaretheorem[style=thmbluebox,name=Твердження,numberwithin=chapter]{proposition}

\mdfdefinestyle{mdredbox}{%
	linewidth=0.5pt,
	skipabove=12pt,
	frametitleaboveskip=5pt,
	frametitlebelowskip=0pt,
	skipbelow=2pt,
	frametitlefont=\bfseries,
	innertopmargin=4pt,
	innerbottommargin=8pt,
	nobreak=true,
	linecolor=RawSienna,
	backgroundcolor=Salmon!5,
}

\declaretheoremstyle[
	headfont=\bfseries\color{RawSienna},
	mdframed={style=mdredbox},
	headpunct={\\[3pt]},
	postheadspace={0pt},
]{thmredbox}

\declaretheorem[style=thmredbox,name=Приклад,numberwithin=chapter]{example}

\mdfdefinestyle{mdgreenbox}{%
	skipabove=8pt,
	linewidth=2pt,
	rightline=false,
	leftline=true,
	topline=false,
	bottomline=false,
	linecolor=ForestGreen,
	backgroundcolor=ForestGreen!5,
}

\declaretheoremstyle[
	headfont=\bfseries\sffamily\color{ForestGreen!70!black},
	bodyfont=\normalfont,
	spaceabove=2pt,
	spacebelow=1pt,
	mdframed={style=mdgreenbox},
	headpunct={ --- },
]{thmgreenbox}

\declaretheorem[style=thmgreenbox,name=Зауваження,numberwithin=chapter]{remark}

\mdfdefinestyle{mdblackbox}{%
	skipabove=8pt,
	linewidth=3pt,
	rightline=false,
	leftline=true,
	topline=false,
	bottomline=false,
	linecolor=black,
	backgroundcolor=RedViolet!5!gray!5,
}

\declaretheoremstyle[
	headfont=\bfseries,
	bodyfont=\normalfont\small,
	spaceabove=0pt,
	spacebelow=0pt,
	mdframed={style=mdblackbox}
]{thmblackbox}

\declaretheorem[style=thmblackbox,name=Вправа,numberwithin=chapter]{exercise}
\newtheorem{definition}{Означення}[chapter]
\newtheorem{abuse}{Зловживання позначеннями}[chapter]

\newcommand{\vocab}[1]{\textbf{\color{blue} #1}}
\newcommand{\mailto}[1]{\href{mailto:#1}{\texttt{#1}}}

\usepackage[skip=\smallskipamount,indent=\parindent]{parskip}
\usepackage[headsepline]{scrlayer-scrpage}
\renewcommand{\headfont}{}
\addtolength{\textheight}{3.14cm}
\setlength{\footskip}{0.5in}
\setlength{\headsep}{10pt}
\automark[chapter]{chapter}
\rohead{\footnotesize\thepage}
\rehead{\footnotesize\textbf{\sffamily Функіональний Аналіз}, \emph{Дмитро Клюшин}}
\lehead{\footnotesize\thepage}
\lohead{\footnotesize\leftmark}
\chead{}
\rofoot{}
\refoot{}
\lefoot{}
\lofoot{}

\makeatletter
\usepackage{etoolbox}
\pretocmd{\tableofcontents}{%
  \if@openright\cleardoublepage\else\clearpage\fi
  \pdfbookmark[0]{\contentsname}{toc}%
}{}{}%
\makeatother
\setcounter{tocdepth}{1}
\usepackage[tocindentauto]{tocstyle}
\usetocstyle{KOMAlike}

\usepackage[tight]{minitoc}
\mtcsetfont{parttoc}{chapter}{\sffamily\bfseries}
\mtcsetfont{parttoc}{section}{\footnotesize\rmfamily\upshape\mdseries}
\mtcsetfont{parttoc}{subsection}{\footnotesize\rmfamily\upshape\mdseries}
\setcounter{parttocdepth}{1}
\renewcommand*{\partheadstartvskip}{\vspace*{20em}}
\renewcommand*{\partheadendvskip}{}
\renewcommand\beforeparttoc{\noindent{\bfseries \Large Частина \thepart: Зміст}}
\doparttoc[n]

\renewcommand*{\sectionformat}{\color{purple}\S\thesection\autodot\enskip}
\renewcommand*{\subsectionformat}{\color{purple}\S\thesubsection\autodot\enskip}
\renewcommand{\thesubsection}{\thesection.\roman{subsection}}

\addtokomafont{chapterprefix}{\raggedleft}
\RedeclareSectionCommand[beforeskip=0.5em]{chapter}
\renewcommand*{\chapterformat}{%
\mbox{\scalebox{1.5}{\chapappifchapterprefix{\nobreakspace}}%
\scalebox{2.718}{\color{purple}\thechapter\autodot}\enskip}}

\addtokomafont{partprefix}{\rmfamily}
\renewcommand*{\partformat}{\color{purple}\scalebox{2.5}{\thepart}}

\newcommand{\listhack}{$\empty$}
\newcommand{\nothing}{$\left.\right.$}
