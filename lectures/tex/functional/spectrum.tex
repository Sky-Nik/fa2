\chapter{Спряжені оператори, спектр і компактні оператори}

\section{Спряжені оператори}

Нехай~$E$ і~$F$~--- лінійні топологічні простори.
Розглянемо неперервний лінійний оператор~$A: E \to F$ і
функціонал~$g \in \conjugate F$.
Застосуємо функціонал~$g$ до елемента
$y = Ax$. Це визначає функціонал
$f \in \conjugate E$, який
визначається формулою~$f(x) = g(Ax)$.

\begin{definition}
Оператор~$\conjugate A: \conjugate F \to \conjugate E$, що визначається
формулою~$f(x) = g(Ax)$ і ставить кожному
функціоналу~$g$ із простору~$\conjugate F$
функціонал~$f$ із простору~$\conjugate E$,
називається \vocab{спряженим} до оператора A.
\end{definition}

\begin{example}
Розглянемо оператор
\begin{equation*}
    A: \RR^n \to \RR^m
\end{equation*}
і функціонал
\begin{equation*}
    y = Ax,
\end{equation*}
який визначається як
\begin{equation*}
    y_i = \sum_{j = 1}^n a_{i,j} x_j, \quad i = 1, 2, \dots, m.
\end{equation*}

Тоді
\begin{equation*}
    f(x) = g(A x) = \sum_{i = 1}^m g_i y_i =
    \sum_{i = 1}^n \sum_{j = 1}^m g_i a_{i,j} x_j =
    \sum_{j = 1}^n x_j \sum_{i = 1}^m g_i a_{i,j}.
\end{equation*}

Отже,
\begin{equation*}
    f_j = \sum_{i = 1}^m g_i a_{i,j}, \quad j = 1, 2, \dots, n
\end{equation*}

З цього випливає, що
\begin{equation*}
    f = \conjugate A g \implies \conjugate A = \transpose A.
\end{equation*}

Це означає, що спряжений оператор визначається
транспонованою матрицею. 
\end{example}

Позначивши значення функціонала~$f$ на елементі~$x$
символом~$(f, x)$, отримаємо, що
\begin{equation*}
    (g, Ax) = (f, x) = (\conjugate A g, x).    
\end{equation*}

\begin{theorem}
Якщо~$A \in \LL(E, F)$, де~$E, F$~--- банахові
простори, то~$\|A\| = \|\conjugate A\|$.
\end{theorem}

\begin{proof}
З одного боку
\begin{equation*}
    |(\conjugate A g, x)| = |(g, A x)| \le \|g\| \cdot \|A\| \cdot \|x\|,
\end{equation*}
звідки
\begin{equation*}
    \|\conjugate A g\| \le \|A\| \cdot \|g\|,
\end{equation*}
тобто
\begin{equation*}
    \|\conjugate A\| \le \|A\|.
\end{equation*}

З іншого боку, для~$x \in E$ і~$Ax \ne 0$ існує елемент
\begin{equation*}
    y_0 = \frac{Ax}{\|Ax\|} \in F \implies \|y_0\| = 1.
\end{equation*}

Отже, за теоремою Хана---Банаха існує функціонал~$g$, такий
що~$\|g\| = 1$, $(g, y_0) = 1$. З цього випливає, що
\begin{equation*}
    (g, y_0) = \left( g, \frac{Ax}{\|Ax\|} \right) = \frac{1}{\|Ax\|} (g, Ax) = 1.
\end{equation*}

Тоді~$(g, Ax) = \|Ax\|$. Таким чином,
\begin{equation*}
    \|Ax\| = (g, Ax) = |(\conjugate A g, x)| \le \|\conjugate A\| \cdot \|g\| \cdot \|x\| = \|\conjugate A\| \cdot \|x\|,
\end{equation*}
тобто
\begin{equation*}
    \|A\| \le \|\conjugate A\|.
\end{equation*}

Поєднуючи дві нерівності отримуємо, що
\begin{equation*}
    \|A\| = \|\conjugate A\|. \qedhere
\end{equation*}
\end{proof}

\section{Спектр оператора}

\begin{definition}
Нехай~$A: E \to E$, де~$E$~--- комплексний банахів простір.
Число~$\lambda$ називається \vocab{регулярним} для
оператора~$A$, якщо оператор
\begin{equation*}
    R_\lambda = (A - \lambda I)\inv
\end{equation*}
визначений на всьому просторі~$E$.
\end{definition}

\begin{definition}
Оператор~$R_\lambda = (A - \lambda I)\inv$ називається
\vocab{резольвентою}.
\end{definition}

\begin{definition}
Сукупність всіх чисел~$\lambda$, які не є регулярними
для оператора~$A$, називається його \vocab{спектром}.
\end{definition}

\begin{definition}
Число~$\lambda$, таке що рівняння
\begin{equation*}
    Ax = \lambda x
\end{equation*}
має ненульові розв’язки,
називається \vocab{власним числом} оператора~$A$.
\end{definition}

\begin{definition}
Всі власні числа оператора~$A$ належать його
спектру і утворюють \vocab{точковий спектр}.
\end{definition}

\begin{definition}
Доповнення до точкового спектру
називається \vocab{неперервним спектром}.
\end{definition}

\begin{example}
Розглянемо простір~$C[a, b]$ і оператор
\begin{equation*}
    A x(t) = t x(t).
\end{equation*}

Тоді
\begin{equation*}
    (A - \lambda I) x(t) = (t - \lambda) x(t).
\end{equation*}

Із умови
\begin{equation*}
    (t - \lambda) x(t) = 0, \quad \forall \lambda \in \RR
\end{equation*}
випливає, що неперервна функція~$x(t)$ тотожно дорівнює
нулю, тому оператор~$(A - \lambda I)\inv$ існує для довільного~$\lambda$.

Проте при~$\lambda \in [a, b]$ обернений оператор, що діє за
формулою
\begin{equation*}
    (A - \lambda I)\inv x(t) = \frac{x(t)}{t - \lambda}
\end{equation*}
визначений не на всьому просторі~$C[a, b]$ і не є
обмеженим. Таким чином, спектром є весь відрізок~$[a, b]$,
власних чисел немає, тобто оператор~$A$ має лише
неперервний спектр.
\end{example}

\begin{remark}
У скінченновимірних просторах
неперервний спектр оператора є порожньою множиною,
спектр збігається із точковим спектром і складається
лише із власних чисел.

У нескінченновимірних просторах
кожне число відносно оператора є регулярним значенням,
власним значенням або елементом неперервного спектру.
\end{remark}

\begin{theorem}
Якщо~$A \in \LL(E, E)$, де~$E$~--- банахів
простір і~$|\lambda| > \|A\|$, то~$\lambda$~--- регулярне значення для
оператора~$A$.
\end{theorem} 

\begin{proof}
Оскільки
\begin{equation*}
    A - \lambda I = - \lambda (I - \tfrac{1}{\lambda} A),
\end{equation*}
то
\begin{equation*}
    R_\lambda = (A - \lambda I)^{-1} =
    -\frac{1}{\lambda} \left( I - \frac{A}{\lambda} \right)^{-1} =
    -\frac{1}{\lambda} \sum_{k = 0}^\infty \frac{A^k}{\lambda^k}.
\end{equation*}

За умови~$|\lambda| > \|A\|$ цей ряд збігається і визначає на~$E$
обмежений оператор (теорема 14.4). 
\end{proof}

\begin{remark}
З \error теореми 15.2 випливає, що спектр
оператора~$A$ міститься в колі радіусу~$\|A\|$ з центром в нулі.
\end{remark}

\section{Компактні оператори}

\begin{definition}
Оператор~$A$, що діє із банахового простору
$E$ в банахів простір~$F$ називається \vocab{компактним}, або
\vocab{цілком неперервним}, якщо кожну обмежену множину він
переводить у відносно компактну множину.
\end{definition}

\begin{example}
Лінійний неперервний оператор~$A$, що
переводить банахів простір~$E$ в його скінченновимірний
підпростір, є компактним.
\end{example}

\begin{theorem}
Якщо послідовність компактних
операторів~$\{A_n\}_{n = 1}^\infty$
в банаховому просторі~$E$ збігається
до оператора~$A$ рівномірно, то оператор~$A$ теж є
компактним.
\end{theorem}

\begin{proof}
Для доведення компактності оператора~$A$
доведемо, що для будь-якої обмеженої послідовності
$\{x_n\}_{n = 1}^\infty \subset E$ із послідовності~$\{Ax_n\}_{n = 1}^\infty$
можна виділити збіжну підпослідовність.

Оператор~$A_1$~--- компактний, тому із послідовності
$\{A_1 x_n\}_{n = 1}^\infty$
можна виділити збіжну підпослідовність. Нехай
$\{x_n^{(1)}\}_{n = 1}^\infty \subset E$~--- послідовність, на якій збігається
послідовність, яку ми виділили із~$\{A_1 x_n\}_{n = 1}^\infty$.

Оператор~$A_2$~--- компактний, тому із послідовності
$\{A_2 x_n^{(1)}\}_{n = 1}^\infty$
можна виділити збіжну підпослідовність. Нехай
$\{x_n^{(2)}\}_{n = 1}^\infty \subset E$~--- послідовність, на якій збігається
послідовність, яку ми виділили із~$\{A_2 x_n^{(1)}\}_{n = 1}^\infty$.

Продовжимо цей процес і виділимо діагональну
послідовність
\begin{equation*}
    x_1^{(1)}, x_2^{(2)}, \dots, x_n^{(n)}, \dots
\end{equation*}

Оператори~$A_1, A_2, \dots, A_n, \dots$ переводять її у збіжну
послідовність. Покажемо, що оператор~$A$ теж переводить її
в збіжну послідовність. Простір~$E$~--- повний, тому
достатньо показати, що~$\{Ax_n^{(n)}\}_{n = 1}^\infty$
є фундаментальною послідовністю.
\begin{multline*}
    \left\|A x_n^{(n)} - A x_m^{(m)}\right\| \le \\
    \left\|A x_n^{(n)} - A_k x_n^{(n)} +
      A_k x_n^{(n)} - A_k x_m^{(m)} +
      A_k x_m^{(m)} - A x_m^{(m)} \right\| \le \\
    \left\|A x_n^{(n)} - A_k x_n^{(n)}\right\| +
    \left\|A_k x_n^{(n)} - A_k x_m^{(m)}\right\| +
    \left\|A_k x_m^{(m)} - A x_m^{(m)} \right\|.
\end{multline*}

Нехай~$\|x_n\| \le C$. Оскільки~$\|A_n - A\| \to 0$ при~$n \to \infty$,
\begin{equation*}
    \exists K \in \NN: \forall k \ge K:
    \left\|A - A_k\right\| < \frac{\epsilon}{3C}.
\end{equation*}

Крім того, оскільки послідовність~$\{A_k x_n^{(n)}\}$ є збіжною,
\begin{equation*}
    \exists N \in \NN: \forall n, m \ge N:
    \left\|A_k x_n^{(n)} - A_k x_m^{(m)} \right\| < \frac{\epsilon}{3}.
    \end{equation*}

Вибравши~$M = \max(K, N)$, отримуємо
\begin{equation*}
    \forall n, m \ge M:
    \left\|A x_n^{(n)} - A x_m^{(m)}\right\| < \epsilon. \qedhere
\end{equation*}
\end{proof}

\begin{theorem}
Якщо~$A$~--- лінійний компактний
оператор, оператор~$B$~--- лінійний обмежений, то
оператори~$AB$ і~$BA$ є компактними.
\end{theorem}

\begin{proof}
Якщо множина~$M \subset E$ є обмеженою,
то~$BM$~--- обмежена множина,
оскільки обмежений оператор
переводить будь-яку обмежену множину
в обмежену множину.
Отже, множина~$ABM$ є відносно компактною.
Це означає, що оператор~$AB$ є компактним.

Аналогічно, якщо множина~$M \subset E$ є обмеженою,
то~$AM$~--- відносно компактна множина,
оскільки компактний оператор
переводить будь-яку обмежену множину
у відносно компактну множину.
Оператор~$B$~--- неперервний,
тому множина~$BAM$ є відносно компактною.
Це означає, що оператор~$BA$ є компактним. 
\end{proof}

\begin{corollary}
В нескінченновимірному просторі~$E$
компактний оператор не може мати
обмеженого оберненого оператору.
\end{corollary}

\begin{theorem}
Оператор, спряжений до компактного, є компактним.
\end{theorem}

Спряжені, самоспряжені і компактні оператори
відіграють особливо важливу роль у гільбертових просторах.
Саме на цих поняттях побудована теорія
розв’язності операторних рівнянь в гільбертових просторах.

\section{Література}

\begin{enumerate}[label={[\arabic*]}]
\item \textbf{Колмогоров~А.~Н.}
Элементы теории функций и функционального анализа. 5-е изд. /
А.~Н.~Колмогоров, С.~В.~Фомин ---
М.: Наука, 1981 (стр.~230--250).
\end{enumerate}
