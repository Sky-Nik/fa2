\chapter{Принцип рівномірної обмеженості}

В цій лекції ми розглянемо види збіжності послідовностей
лінійних неперервних операторів і з’ясуємо, коли простір
$\LL(E, F)$ є банаховим в розумінні тої чи іншої збіжності.

\section{Види збіжності послідовностей операторів}

\begin{definition}
Послідовність операторів $\{A_n\}_{n = 1}^\infty$, що діють із
нормованого простору $E$ в нормований простір $F$,
\vocab{поточково збігається} до оператора $A$ в просторі $\LL(E, F)$
при $n \to \infty$, якщо $\forall x \in E$: $\lim_{n \to \infty} A_n x = A x$.
\end{definition}

\begin{definition}
Послідовність операторів $\{A_n\}_{n = 1}^\infty$, що діють із
нормованого простору $E$ в нормований простір $F$,
\vocab{рівномірно збігається} до оператора $A$ в просторі $\LL(E, F)$
при $n \to \infty$, якщо $\lim_{n \to \infty} \|A_n - A\| = 0$.
\end{definition}

\begin{remark}
Якщо $F = \RR$, то простір $\LL(E, \RR)$ є
спряженим простором, поточкова збіжність є аналогом
слабкої збіжності в спряженому просторі, а рівномірна
збіжність є аналогом сильної збіжності в спряженому
просторі.
\end{remark}

\begin{lemma}
Якщо послідовність лінійних обмежених
операторів $A_n: E \to F$, де $E, F$ --- нормовані простори, є
такою, що послідовність $\{\|A_n\|\}_{n = 1}^\infty$
є необмеженою, то послідовність $\{\|A_n x\|\}_{n = 1}^\infty$
є необмеженою в будь-якій замкненій кулі.
\end{lemma}

\begin{proof}
Припустимо супротивне: послідовність
$\{\|A_n x\|\}_{n = 1}^\infty$
є обмеженою в деякій замкненій кулі $\closure S(x_0, \epsilon)$:
\begin{equation*}
    \exists (\closure S(x_0, \epsilon), C > 0):
    \forall n \in N:
    \forall x \in \closure(x_0, \epsilon):
    \|A_n x\|_F \le C.
\end{equation*}

Кожному елементу $\xi \in E$ поставимо у відповідність елемент
$x = \frac{\epsilon}{\|\xi\|_E} \xi + x_0$, якщо $\xi \ne 0$.
Елементу $\xi = 0$ поставимо у відповідність елемент $x = x_0$.
\begin{equation*}
    \xi \ne 0 \implies
    \|x - x_0\|_E =
    \left\| \frac{\epsilon}{\|\xi\|_E} \xi + x_0 - x_0 \right\|_E =
    \left\| \frac{\epsilon}{\|\xi\|_E} \xi \right\|_E =
    \epsilon.
\end{equation*}

Це означає, що для довільних $\xi \in E$ всі елементи
$x \in \closure S(x_0, \epsilon)$.

Оцінимо наступну величину (використовуючи допоміжну
нерівність $\|x\| - \|y\| \le \|x + y\|$.
\begin{equation}
    \left| \frac{\epsilon}{\|\xi\|_E} \|A_n \xi\|_F - \|A_n x_0\|_F \right| \le
    \left\| \frac{\epsilon}{\|\xi\|_E} A_n \xi + A_n x_0 \right\|_F =
    \left\| A_n \left( \frac{\epsilon}{\|\xi\|_E} \xi + x_0 \right) \right\|_F \le C.
\end{equation}

Отже,
\begin{equation*}
    \frac{\epsilon}{\|\xi\|_E} \|A_n \xi\|_F - \|A_n x_0\|_F \le C.
\end{equation*}

Звідси випливає, що
\begin{equation*}
    \|A_n \xi\|_F \le
    \frac{C + \|A_n x_0\|_F}{\epsilon} \|\xi\|_E \le
    \frac{2 C}{\epsilon} \|\xi\|_E.
\end{equation*}

Отже,
\begin{equation*}
    \exists C_1 = \frac{2 C}{\epsilon} > 0:
    \forall \xi \in E:
    \|A_n \xi\|_E \le C_1 \|\xi\|_E \implies
    \|A_n\| \le C_1.
\end{equation*}

Отримане протиріччя доводить лему.
\end{proof}

\begin{theorem}
[Банаха---Штейнгауза] Нехай послідовність
лінійних обмежених операторів $\{A_n\}_{n = 1}^\infty$, що відображають
банахів простір $E$ в нормований простір $F$, поточково
збігається до оператора $A$ при $n \to \infty$. Тоді послідовність
$\{\|A_n\|\}_{n = 1}^\infty$ є обмеженою, оператор $A$ є лінійним і
неперервним, а $A_n x \to A x$ рівномірно по $n$ на кожному
компакті $K \subset E$ (тобто $n$ не залежить від $x$).
\end{theorem}

\begin{proof}
Припустимо, що послідовність $\{\|A_n\|\}_{n = 1}^\infty$ є необмеженою.
Тоді за лемою 13.1 послідовність $\{\|A_n x\|\}_{n = 1}^\infty$
є необмеженою на довільній замкненій кулі $\closure S(x_0, \epsilon_0)$.

Отже, 
\begin{equation*}
    \exists (n_1 \in \NN, x_1 \in \closure S(x_0, \epsilon_0): \|A_{n_1} x_1\|_F > 1.    
\end{equation*}

Оскільки $A_{n_1}$ --- неперервний оператор,
\begin{equation*}
    \exists \closure S(x_1, \epsilon_1) \subset \closure S(x_0, \epsilon_0):
    \forall x \in \closure S(x_1, \epsilon_1): \|A_{n_1} x\|_F > 1.
\end{equation*}

На кулі $\closure S(x_1, \epsilon_1)$ послідовність $\{\|A_n x\|_F\}_{n = 1}^\infty$
також є необмеженою. Отже,
\begin{equation*}
    \exists \closure S(x_2, \epsilon_2) \subset \closure S(x_1, \epsilon_1):
    \forall x \in \closure S(x_2, \epsilon_2): \|A_{n_2} x\|_F > 2.
\end{equation*}

Нехай $A_{n_1}, A_{n_2}, \dots, A_{n_k}$ і $x_1, x_2, \dots, x_k$:
\begin{equation*}
    n_1 < n_2 < \dots < n_k, \quad
    \closure S(x_0, \epsilon_0) \supset 
    \closure S(x_1, \epsilon_1) \supset
    \dots \supset \closure S(x_k, \epsilon_k).
\end{equation*}

Продовжуючи цей процес при $k \to \infty$, отримуємо
послідовність вкладених замкнених куль, таких що
\begin{equation*}
    \forall x \in \closure S(x_k, \epsilon_k): \|A_{n_k} x\|_F > k, \quad \epsilon_k \to 0.
\end{equation*}

Оскільки $E$ --- повний простір, за принципом вкладених куль
\begin{equation*}
    \exists x^\star \in \Bigcap_{k = 1}^\infty S(x_k, \epsilon_k):
    \|A_{n_k} x^\star\|_F \ge k, \quad \forall k \in \NN.
\end{equation*}

Звідси випливає, що $\exists x^\star \in E$ така, що послідовність $\{A_n x^\star\}$
не збігається. Це суперечить умові теореми, згідно якої
послідовність операторів $\{A_n x\}_{n = 1}^\infty$
поточково збігається в кожній точці простору $E$.

Покажемо, що оператор $A$ --- лінійний. Оскільки
\begin{equation*}
    A_n(x + y) = A_n(x) + A_n(y), \quad A_n(\lambda x) = \lambda A_n(x),
\end{equation*}
маємо
\begin{gather*}
    A(x + y) =
    \lim_{n \to \infty} A_n(x + y) =
    \lim_{n \to \infty} A_n(x) + \lim_{n \to \infty} A_n(y) =
    A x + A y. \\
    A(\lambda x) =
    \lim_{n \to \infty} A_n(\lambda x) =
    \lambda \lim_{n \to \infty} A_n(x) =
    \lambda A x.
\end{gather*}

Крім того,
\begin{equation*}
    \|A_n x\|_F \le C \|x\|_E \implies
    \lim_{n \to \infty} \|A_n x\|_F =
    \left\| \lim_{n \to \infty} A_n x \right\|_F =
    \|A x\|_E \le C \|x\|_E.
\end{equation*}

Отже, $A$ --- лінійний і обмежений, а значить, неперервний.

Нехай $K \subset E$ --- компакт, $\epsilon > 0$. За теоремою Хаусдорфа
існує скінчена $\frac{\epsilon}{3 C}$-сітка $M$:
\begin{equation*}
    \forall x \in K:
    \exists x_\alpha \in M, \alpha \in A:
    \|x - x_\alpha\|_E < \frac{\epsilon}{3 C},
\end{equation*}
де $A$ --- скінчена множина.

Оскільки послідовність $\{A_n x\}_{n = 1}^\infty$
поточково збігається в
кожній точці простору $E$, то вона збігається і в кожній точці
сітки $M$:
\begin{equation*}
    \forall x_\alpha \in M:
    \exists n_\alpha:
    \forall n \ge n_\alpha:
    \|A_n x_\alpha - A x_\alpha\|_F < \frac{\epsilon}{3}.
\end{equation*}

Нехай $n_0 = \max_{\alpha \in A} n_\alpha$ (сітка $M$ є скінченою, тому максимум існує).
Тоді $\forall n \ge n_0$, $\forall x \in S\left(x_\alpha, \frac{\epsilon}{3C}\right)$
\begin{multline*}
    \|A_n x - A x\|_F \le
    \|A_n x - A_n x_\alpha + A_n x_\alpha - A x_\alpha + A x_\alpha - A x\|_F \le \\
    \|A_n x - A_n x_\alpha\|_F + \|A_n x_\alpha - A x_\alpha\|_F + \|A x_\alpha - A x\|_F < \\
    C \|x - x_\alpha\|_F + \frac{\epsilon}{3} + C \|x - x_\alpha\|_F =
    \frac{\epsilon}{3} + \frac{\epsilon}{3} + \frac{\epsilon}{3} = \epsilon.
\end{multline*}

Отже, $\forall n \ge n_0$, $\forall x \in K$: $\|A_n x - A x\|_F < \epsilon$,
до того ж номер $n_0$ не залежить від точки $x$. Це означає, що
$A_n x \to A x$ рівномірно по $n$ на кожному компакті $K \subset E$.
\end{proof}

\section{Повнота простору лінійних неперервних операторів}

З’ясуємо, коли простір $\LL(E, F)$ є повним у розумінні
рівномірної або точкової збіжності.

\begin{theorem}
Якщо нормований простір $F$ --- банахів,
то $\LL(E, F)$ --- банахів у розумінні рівномірної збіжності.
\end{theorem}

\begin{proof}
Нехай $\{A_n\}_{n = 1}^\infty$ --- фундаментальна
послідовність операторів, тобто
\begin{equation*}
    \|A_n - A_m\| \to 0, \quad n, m \to \infty.
\end{equation*}

Тоді $\forall x \in E$
\begin{equation*}
    \|A_n x - A_m x\| \le \|A_n - A_m\| \cdot \|x\| \to 0, \quad n, m \to \infty.
\end{equation*}

Для кожного фіксованого $x \in E$ послідовність $\{A_n x\}$ є
фундаментальною в $F$. Оскільки простір $F$ є повним за
умовою теореми, то послідовність $\{A_n x\}$ збігається до
певного елемента $y \in F$. Позначимо $\lim_{n \to \infty} A_n x$.
Отже, ми визначили відображення $A: E \to F$.
Його лінійність випливає із властивостей границі.
Покажемо його обмеженість:
$\{\|A_n\|\}$ фундаментальна в $\RR$, адже
\begin{equation*}
    |\|A_n\| - \|A_m\|| \le \|A_n - A_m\| \to 0, \quad n, m \to \infty,
\end{equation*}
а отже $\{\|A_n\|\}$ обмежена в $\RR$, тобто
\begin{equation*}
    \exists C: \forall n \in \NN: \|A_n\| \le C.
\end{equation*}

Отже,
\begin{equation*}
    \|A_n x\| \le \|A_n\| \cdot \|x\| \le C \|x\|.
\end{equation*}

Внаслідок неперервності норми, маємо
\begin{equation*}
    \|A x\| \lim_{n \to \infty} \|A_n x\| \le C \|x\|.
\end{equation*}

Покажемо, що $A_n$ рівномірно збігається до $A$ в просторі
$\LL(E, F)$. Задамо $\epsilon > 0$ і виберемо $n_0$ так, щоб
$\|A_{n + p} x - A_n x\| < \epsilon$ для $n \ge n_0$, $p > 0$ і для будь-якого
$x: \|x\| \le 1$.
Нехай $p \to \infty$. Тоді
\begin{equation*}
    \forall n \ge n_0, x: \|x\| \le 1:
    \|A x - A_n x\| < \epsilon,
\end{equation*}
звдки
\begin{equation*}
    \|A_n - A\| = \sup_{\|x\| \le 1} \|(A_n - A) x\| \le \epsilon,
\end{equation*}
а тому $A = \lim_{n \to \infty} A_n$
в розумінні рівномірної збіжності.

Отже, $\LL(E, F)$ є банаховим. 
\end{proof}

\begin{theorem}
Якщо нормовані просторі $E$ і $F$ --- банахові,
то $\LL(E, F)$ --- банахів у розумінні поточкової збіжності.
\end{theorem}

\begin{proof}
Розглянемо точку $x \in E$ і фундаментальну у
розумінні поточкової збіжності послідовність $\{A_n\}_{n = 1}^\infty$.

Оскільки $F$ --- банахів простір, то існує елемент $y = \lim_{n \to \infty} A_n x$.
Таким чином, визначений оператор
$A: E \to F$, такий що $y = Ax$. Лінійність цього оператора
випливає із лінійності границі, а обмеженість --- із теореми
Банаха-Штейнгауза:
\begin{equation*}
    \|A x\| =
    \left\| \lim_{n \to \infty} A_n x\right\| \le
    \lim_{n \to \infty} \|A_n\| \cdot \|x\| =
    C \|x\|. \qedhere
\end{equation*}
\end{proof}

\section{Література}

\begin{enumerate}[label={[\arabic*]}]
\item \textbf{Садовничий~В.~А.}
Теория операторов /
В.~А.~Садовничий ---
М.: Изд-во Моск. ун-та, 1986 (стр.~96--102).
\item \textbf{Ляшко~И.~И.}
Основы классического и современного математического анализа /
И.~И.~Ляшко, В.~Ф~Емельянов, А.~К.~Боярчук. ---
К.: Вища школа, 1988 (стр.~576--578).
\end{enumerate}
