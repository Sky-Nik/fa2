\chapter{Принцип відкритості відображення}

\section{Обмеженість на всюди щільній множині}

\begin{lemma}
Нехай~$E$ і~$F$~--- банахові простори,
$A \in \LL(E, F)$, $E_n$~--- множина тих точок~$x \in E$, для яких
\begin{equation*}
    \|A x\|_F \le n \|x\|_E, \quad n = 1, 2, \dots    
\end{equation*}

Тоді~$E = \Bigcup_{n = 1}^\infty E_n$ і принаймні одна із множин~$E_n$ є всюди
щільною в~$E$.
\end{lemma}

\begin{proof}
Спочатку пересвідчимось в тому, що
\begin{equation*}
    \forall x \in E: \exists n \in \NN: x \in E_n.
\end{equation*}

Очевидно, що~$E_n \ne \emptyset$, оскільки~$\forall n \in \NN: 0 \in E$.
Якщо~$x \ne 0$, позначимо через~$n$ найменше натуральне число, що
задовольняє нерівність
\begin{equation*}
    n \ge \frac{\|A x\|_F}{\|x\|_E}.
\end{equation*}

Тоді
\begin{equation*}
    \forall x \in E: \exists n \in \NN: \|Ax\|_F \le n \|x\|_E.
\end{equation*}

Звідси випливає, що
\begin{equation*}
    E = \Bigcup_{n = 1}^\infty E_n
\end{equation*}

Згідно теореми Бера, банахів простір~$E$ не може бути
поданий у вигляді не більш ніж зліченного об’єднання ніде
не щільних множин. Значить, одна із множин~$E_{n_0}$ не є ніде
не щільною. Отже, існує відкрита куля~$S(x_0, r)$, така що
$S(x_0, r) \subset \closure E_{n_0}$.

Розглянемо замкнену кулю~$\closure S(x_1, r_1)$ з центром
$x_1 \in E_{n_0}$, таку що
\begin{equation*}
    \closure S(x_1, r_1) \subset S(x_0, r).
\end{equation*}

Візьмемо довільний елемент~$x$ з нормою~$\|x\| = r_1$. Оскільки
\begin{equation*}
    \|x_1 + x - x_1\|_E = \|x\|_E = r_1,
\end{equation*}
отримаємо, що~$x_1 + x \in \closure S(x_1, r_1)$. Отже,
$\closure S(x_1, r_1) \subset \closure E_{n_0}$, звідки
\begin{equation*}
    \exists \{y_k\}_{k = 1}^\infty \subset S(x_1, r_1) \cap E_{n_0}:
    y_k \to x_1 + x, \quad k \to \infty.
\end{equation*}

Якщо~$x_1 + x \in E_{n_0}$, ця послідовність може бути стаціонарною.
Таким чином, $\exists \{x_k\}_{k = 1}^\infty = \{y_k - x_1\}_{k = 1}^\infty$, така
що
\begin{equation*}
    \lim_{k \to \infty} x_l = \lim_{k \to \infty} y_k - x_1 = x.
\end{equation*}

Оскільки
\begin{equation*}
    \|x\|_E = r_1, \quad \|x_k\|_E \le r_1,
\end{equation*}
можна вважати, що
\begin{equation}
    \label{eq:14.1}
    \forall k \in \NN: \|x_k\|_E \ge \frac{r_1}{2}
\end{equation}

Із умов~$y_k \in E_{n_0}$, $x_1 \in E_{n_0}$, $y_k = x_k + x_1$
маємо наступні оцінки
\begin{align}
    \label{eq:14.2}
    \|Ax_k\|_F &= \|A y_k - A x_1\|_F \le \|A y_k\|_F + \|A x_1\|_F \le n_0 (\|y_k\|_E + \|x_1\|_E). \\
    \label{eq:14.3}
    \|y_k\|_E &= \|x_k + x_1\|_E \le \|x_k\|_E + \|x_1\|_E \le r_1 + \|x_1\|_E.
\end{align}

Беручи до уваги умову \eqref{eq:14.1} і оцінки \eqref{eq:14.2}, \eqref{eq:14.3}, маємо
\begin{equation*}
    \|A x_k\|_F \le
    n_0 (r_1 + 2 \|x_1\|_E) \le
    \frac{2 n_0|}{r_1} (r_1 + 2 \|x_1\|_E) \|x_k\|_E.
\end{equation*}

Нехай~$n$~--- найменше натуральне число, що задовольняє нерівність
\begin{equation*}
    n \ge \frac{2 n_0}{r_1} (r_1 + 2 \|x_1\|_E).
\end{equation*}

Тоді~$\|Ax_k\|_F \le n \|x_k\|_E$, тобто~$x_k \in E_n$.

Таким чином, довільний елемент~$x$, норма якого дорівнює
$r_1$ можна апроксимувати елементами множини~$E_n$.

Нехай~$x \in E$~--- довільний ненульовий елемент.
Розглянемо точку 
\begin{equation*}
    \xi = r_1 \frac{x}{\|x\|_E}.
\end{equation*}

Вище ми довели, що існує послідовність
\begin{equation*}
    \{\xi_k\}_{k = 1}^\infty: \xi_k \in E_n, \lim_{k \to \infty} \xi_k = \xi.
\end{equation*}

Тоді
\begin{equation*}
    \lim_{k \to \infty} x_k = \lim_{k \to \infty} \xi_k \frac{\|x\|_E}{r_1} = x,
\end{equation*}
звідки
\begin{equation*}
    \|Ax_k\|_F = \frac{\|x\|_E}{r_1} \|A\xi_k\|_F \le \frac{\|x\|_E}{r_1} n \|\xi_k\|_E = n \|x_k\|_E.
\end{equation*}

Отже, $x_k \in E_n$ і~$\lim_{k \to \infty} x_k = x$, $\forall x \in E$.
Таким чином, множина~$E_n$ скрізь щільна в~$E$.
\end{proof}

\section{Лінійний обмежений обернений оператор}

\begin{theorem}[Банаха, про обернений оператор]
Нехай~$E$ і~$F$~--- банахові простори,
$A$~--- лінійний обмежений взаємно-однозначний оператор,
що діє із~$E$ в~$F$.
Тоді існує лінійний обмежений обернений
оператор~$A\inv: F \to E$.
\end{theorem}

\begin{proof}
Покажемо лінійність оберненого оператора.
Покладемо~$\forall x_1, x_2 \in E$: $A x_1 = y_1$, $A x_2 = y_2$.
Внаслідок лінійності оператора~$A$
\begin{equation}
    \label{eq:14.4}
    \forall \alpha, \beta \in \RR:
    A(\alpha x_1 + \beta x_2) =
    \alpha y_1 + \beta y_2.
\end{equation}

Оскільки~$A\inv y_1 = x_1$, $A\inv y_2 = x_2$, помножимо ці рівності на
$\alpha$ і~$\beta$ відповідно і складемо результати:
\begin{equation}
    \label{eq:14.5}
    \alpha A\inv y_1 + \beta A\inv y_2 =
    \alpha x_1 + \beta x_2.
\end{equation}

Із рівності \eqref{eq:14.4} і означення оберненого оператора випливає, що
\begin{equation*}
    \alpha x_1 + \beta x_2 = A\inv (\alpha y_1 + \beta y_2).
\end{equation*}

Беручи до уваги рівність \eqref{eq:14.5}, отримуємо
\begin{equation*}
    A\inv (\alpha y_1 + \beta y_2) = \alpha A\inv y_1 + \beta A\inv y_2.
\end{equation*}

Отже, оператор~$A\inv$ є лінійним. Тепер доведемо його обмеженість.

За лемою 14.1 банахів простір~$F$ можна подати у вигляді
\begin{equation*}
    F = \Bigcup_k F_k
\end{equation*}
де~$F_k$~--- множина таких елементів~$y \in F$, для яких
\begin{equation*}
    \|A\inv y\|_E \le k \|y\|_F,
\end{equation*}
до того ж одна із множин~$F_k$ скрізь щільна в~$F$. Позначимо
цю множину через~$F_n$. Візьмемо довільну точку~$y \in F$, а її
норму позначимо як~$\|y\|_F = a$. Знайдемо таку точку
$y_1 \in F_n$, щоб виконувались нерівності
\begin{equation*}
    \|y - y_1\|_F \le \frac{a}{2}, \quad \|y_1\|_F \le a.
\end{equation*}

Такий вибір можливий, оскільки множина~$\closure S(0, a) \cap F_n$ є
щільною в замкненій кулі~$\closure S(0, a)$ і~$y \in \closure S(0, a)$. Знайдемо
такий елемент~$y_2 \in F_n$, щоб виконувались умови
\begin{equation*}
    \|y - y_1 - y_2\|_F \le \frac{a}{2^2}, \quad \|y_1\|_F \le \frac{a}{2}.
\end{equation*}

Продовжуючи вибір, побудуємо елементи~$y_k \in F_n$, такі що
\begin{equation*}
    \|y - (y_1 + \dots + y_k)\|_F \le \frac{a}{2^k}, \quad \|y_k\|_F \le \frac{a}{2^{k - 1}}.
\end{equation*}

Внаслідок вибору елементів~$y_k$ маємо
\begin{equation*}
    \lim_{m \to \infty} \left\| y - \sum_{k = 1}^m y_k \right\|_F = 0.
\end{equation*}

Це означає, що ряд~$\sum_{k = 1}^\infty y_k$ збігається до елемента~$y$.

Покладемо~$x_k = A\inv y_k$. Тоді отримуємо оцінку
\begin{equation*}
    \|x_k\|_E \le n \|y_k\|_F \le \frac{n a}{2^{k - 1}}.
\end{equation*}

Оскільки
\begin{multline*}
    \|v_{k + p} - v_k\|_E =
    \left\| \sum_{i = k + 1}^{k + p} x_i \right\|_E \le
    \sum_{i = k + 1}^{k + p} \|x_i\|_E \le
    \sum_{i = k + 1}^\infty \|x_i\|_E \le \\
    \sum_{i = k + 1}^\infty \frac{n a}{2^{i - 1}} =
    \sum_{i = 0}^\infty \frac{n a}{2^{i + k}} =
    \frac{n a}{2^k} \sum_{i = 0}^\infty \frac{1}{2^i} =
    \frac{n a}{2^k} \frac{1}{1 - \tfrac{1}{2}} =
    \frac{n a}{2^{k - 1}},
\end{multline*}
а простір~$E$~--- повний, послідовність~$\{v_k\}_{k = 1}^\infty$, де
$v_k = \sum_{i = 1}^k x_i$ збігається до деякої границі~$x \in E$. Отже,
\begin{equation*}
    x = \lim_{k \to \infty} \sum_{i = 1}^k x_i = \sum_{i = 1}^\infty x_i.
\end{equation*}

Внаслідок лінійності і неперервності оператора~$A$, маємо
\begin{equation*}
    A x =
    A \left( \lim_{k \to \infty} \sum_{i = 1}^k x_i \right) =
    \lim_{k \to \infty} \sum_{i = 1}^k A x_i =
    \lim_{k \to \infty} \sum_{i = 1}^k y_i = y.
\end{equation*}

Звідси отримуємо, що
\begin{equation*}
    \|A\inv y\|_E = \|x\|_E =
    \lim_{k \to \infty} \left\| \sum_{i = 1}^k x_i \right\|_E \le
    \lim_{k \to \infty} \sum_{i = 1}^k \|x_i\|_E \le
    \sum_{i = 1}^\infty \frac{n a}{2^{i - 1}} =
    2 n a = 2 n \|y\|_E.
\end{equation*}
 
Оскільки~$y$~--- довільний елемент із простору~$F$,
обмеженість оператора~$A\inv$ доведено. 
\end{proof}

\begin{corollary}
Якщо~$E$ і~$F$~--- банахові простори,
$A \in \LL(E, F)$, то образ будь-якого околу нуля простору~$E$
містить деякий окіл нуля простору~$F$.
\end{corollary}

\section{Обернений до наближеного і резольвента}

Нехай~$E$, $F$~--- банахові простори. Відокремимо в
банаховому просторі~$\LL(E, F)$ множину операторів
$\mathfrak{M}(E, F)$, що мають обернений оператор.

\begin{theorem}
Нехай~$A_0 \in \mathfrak{M}(E, F)$, $\Delta \in \LL(E, F)$ і
$\|\Delta\| \cdot \|A_0\inv\| < 1$.
Тоді~$A = A_0 + \Delta \in \mathfrak{M}(E, F)$.
\end{theorem}

\begin{proof}
Зафіксуємо довільний~$y \in F$ і розглянемо
відображення~$B: E \to E$, таке що~$B x = A_0\inv y - A_0\inv \Delta x$.

Оскільки~$\|\Delta\| \cdot \|A_0\inv\| < 1$, відображення~$B$ є стискаючим.
Простір~$E$~--- банахів, тому існує єдина нерухома точка відображення~$B$
\begin{equation*}
    x = B x = A_0\inv y - A_0\inv \Delta x.
\end{equation*}

Отже,
\begin{equation*}
    A x = A_0 x + \Delta x = y.
\end{equation*}

Якщо існує ще одна точка~$x'$, така що~$A x' = y$,
то~$x'$ також є нерухомою точкою відображення~$B$.
Оскільки це відображення має єдину нерухому точку,
це означає, що~$x = x'$.
Отже, для будь-якого~$y \in F$ рівняння~$A x = y$
має єдиний розв’язок в просторі~$E$.
Значить, оператор~$A$ має обернений оператор~$A\inv$.
За теоремою Банаха про обернений оператор~$A\inv$ є обмеженим. 
\end{proof}

\begin{theorem}
Нехай~$E$~--- банахів простір,
$I$~--- тотожній оператор, що діє в~$E$,
$A \in \LL(E, E)$ і~$\|A\| < 1$.
Тоді оператор~$(I - A)\inv$ існує, обмежений
і може бути поданий у вигляді
\begin{equation*}
    (I - A)\inv = \sum_{k = 0}^\infty A^k.
\end{equation*}
\end{theorem}

\begin{proof}
Спочатку зауважимо, що із~$\|A\| < 1$ випливає
\begin{equation*}
    \sum_{k = 0}^\infty \|A^k\| \le \sum_{k = 0}^\infty \|A\|^k < \infty.
\end{equation*}

Простір~$E$~--- банахів, тому із збіжності ряду~$\sum_{k = 0}^\infty \|A^k\|$
випливає, що~$\sum_{k = 0}^\infty A^k \in \LL(E, E)$. Для довільного~$n \in \NN$:
\begin{equation*}
    (I - A) \sum_{k = 0}^n A^k = \sum_{k = 0}^n A^k (I - A) = I - A^{n + 1}.
\end{equation*}

Перейдемо до границі при~$n \to \infty$ і зважимо на те, що
$\|A^{n + 1}\| \le \|A\|^{n +1} \to 0$. Отже,
\begin{equation*}
    (I - A) \sum_{k = 0}^\infty A^k = \sum_{k = 0}^\infty A^k (I - A) = I.
\end{equation*}

Звідси випливає, що
\begin{equation*}
    (I - A)\inv = \sum_{k = 0}^\infty A^k. \qedhere
\end{equation*}
\end{proof}

\section{Принцип відкритості відображення}

\begin{theorem}[принцип відкритості відображення]
Лінійне сюр’єктивне і неперервне відображення банахова
простору~$E$ на банахів простір~$F$ є відкритим
відображенням.
\end{theorem}

\begin{proof}
Покажемо, що образ будь-якої відкритої
множини простору~$E$ є відкритою множиною простору~$F$.
Нехай~$G \subset E$~--- непорожня відкрита множина, $x \in G$, а
$G_0$~--- окіл нуля в~$E$, такий що~$x + G_0 \in G$. Розглянемо
окіл нуля~$G_1$ в просторі~$F$, такий що~$G_1 \subset A G_0$, який існує
завдяки наслідку 14.1. Мають місце включення
\begin{equation*}
    A x + G_1 \subset A x + A G_0 = A (x + G_0) \subset A G.
\end{equation*}

Оскільки~$A x + G_1$ є околом точки~$A x$, а~$x$~--- довільна
точка із множини~$G$ і~$A x \in A G$, то множина~$A G$ разом із
кожною своєю точкою містить її деякий окіл~$W$. Отже,
множина~$A G$ є відкритою і відображення~$A$ є відкритим.
\end{proof}

\section{Література}

\begin{enumerate}[label={[\arabic*]}]
\item \textbf{Березанский~Ю.~М.}
Функциональный анализ /
Ю.~М.~Березанский, Г.~Ф.~Ус, З.~Г.~Шефтель ---
К.: Выща школа, 1990 (стр.~254--255).
\item \textbf{Ляшко~И.~И.}
Основы классического и современного математического анализа /
И.~И.~Ляшко, В.~Ф~Емельянов, А.~К.~Боярчук. ---
К.: Вища школа, 1988 (стр.~578--581).
\item \textbf{Садовничий~В.~А.}
Теория операторов /
В.~А.~Садовничий ---
М.: Изд-во Моск. ун-та, 1986 (стр.~102--106).
\item \textbf{Колмогоров~А.~Н.}
Элементы теории функций и функционального анализа. 5-е изд. /
А.~Н.~Колмогоров, С.~В.~Фомин ---
М.: Наука, 1981 (стр.~224--233).
\end{enumerate}
