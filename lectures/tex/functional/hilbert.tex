\chapter{Гільбертові простори}

\section{Скалярний добуток і породжена ним норма}

\begin{definition}
Дійсна лінійна система~$H$ називається
\vocab{дійсним передгільбертовим простором} (або \vocab{евклідовим},
або \vocab{унітарним}), якщо кожній парі елементів~$x, y$
поставлено у відповідність дійсне число~$(x, y)$, що
задовольняє умови (\vocab{аксіоми скалярного добутку}):
\begin{enumerate}
\item $(x, x) \ge 0$, до того ж~$(x, x) = 0$ тільки при~$x = \vec 0$;
\item $(x, y) = (y, x)$;
\item $(x_1 + x_2, y) = (x_1, y) + (x_2, y)$;
\item $(\lambda x, y) = \lambda (x, y)$.
\end{enumerate}
\end{definition}

\begin{lemma}[нерівність Коші---Буняковського]
В дійсному передгільбертовому просторі справджується нерівність
\begin{equation*}
|(x, y)| \le \sqrt{(x, x)} \sqrt{(y, y)},
\end{equation*}
для довільних~$x, y \in H$.
\end{lemma}

\begin{proof}
Розглянемо вираз
\begin{equation*}
    0 \le (x + \lambda y, x + \lambda y) = (x, x) + 2 \lambda (x, y) + \lambda^2 (y, y).
\end{equation*}

Це означає, що дискримінант цього квадратного трьохчлена
є недодатним:
\begin{equation*}
    (x, y)^2 - (x, x) (y, y) \le 0.
\end{equation*}

Отже,
\begin{equation*}
|(x, y)| \le \sqrt{(x, x)} \sqrt{(y, y)}. \qedhere
\end{equation*}
\end{proof}

За скалярним добутком в~$H$ можна ввести норму~$\|x\| = \sqrt{(x, x)}$.

\begin{lemma}
Відображення~$\|\cdot\|: x \mapsto \sqrt{(x, x)}$ є нормою.
\end{lemma}

\begin{proof}
Перевіримо аксіоми норми.
\begin{enumerate}
\item $\forall x \in H$: $\|x\| = 0 \iff x = \vec 0$:
\begin{equation*}
    \sqrt{(x, x)} = 0 \iff (x, x) = 0 \iff x = \vec 0.
\end{equation*}

\item $\|\lambda x\| = |\lambda| \cdot \|x\|$, $|\forall x \in H$, $\forall \lambda \in \RR$:
\begin{equation*}
    \|\lambda x\| =
    \sqrt{(\lambda x, \lambda x)} =
    \sqrt{\lambda (x, \lambda x)} =
    \sqrt{\lambda^2 (x, x)} =
    |\lambda| \sqrt{(x, x)} =
    |\lambda| \cdot \|x\|.
\end{equation*}

\item $\|x + y\| \le x + y$, $\forall x, y \in H$:
\begin{multline*}
    \|x + y\|^2 = (x + y, x + y) =
    (x, x) + 2 (x, y) + (y, y) \le \\
    \|x\|^2 + 2 \|x\| \|y\| + \|y\|^2 =
    (\|x\| + \|y\|)^2 \implies
    \|x + y\| \le \|x\| + \|y\|.
\end{multline*}

Що і треба було довести. \qedhere
\end{enumerate}
\end{proof}

\begin{lemma}
Скалярний добуток є неперервним
відображенням, тобто
\begin{equation*}
    \lim_{n \to \infty} x_n = x,
    \lim_{n \to \infty} y_n = y \implies
    \lim_{n \to \infty} (x_n, y_n) = (x, y).
\end{equation*}
\end{lemma}

\begin{proof}
\begin{multline*}
    |(x, y) - (x_n, y_n)| =
    |(x, y) - (x, y_n) + (x, y_n) - (x_n, y_n)| =  \\
    |(x, y - y_n) + (x - x_n, y_n) \le
    |(x, y - y_n)| + |(x - x_n, y_n)| \le \\
    \|x\| \cdot \|y - y_n\| + \|x - x_n\| \cdot \|y_n\|.
\end{multline*}

Враховуючи, що з~$\lim_{n \to \infty} y_n = y$ випливає, що 
$\exists C: \forall n: \|y_n\| \le C$, можемо заключити, що
\begin{equation*}
    \lim_{n \to \infty}  |(x, y) - (x_n, y_n)| \le 0,
\end{equation*}
як сума двох доданків вигляду~$0 \cdot C$, а тому
\begin{equation*}
    \lim_{n \to \infty} (x_n, y_n) = (x, y). \qedhere
\end{equation*}
\end{proof}

\section{Скалярний добуток породжений нормою}

\begin{proposition}
[характеристична властивість передгільбертових просторів]
Для того щоб нормований простір~$E$ був передгільбертовим
необхідно і достатньо, щоб для
довільних елементів x і y виконувалась рівність
\begin{equation}
    \label{eq:16.1}
    \forall x, y \in H: \|x + y\|^2 + \|x - y\|^2 = 2 (\|x\|^2 + \|y\|^2).
\end{equation}
\end{proposition}

\begin{proof}
Необхідність.
\begin{multline*}
    \|x + y\|^2 + \|x - y\|^2 =
    (x + y, x + y) + (x - y, x - y) = \\
    (x, x) + 2 (x, y) + (y, y) + (x, x) - 2 (x, y) + (y, y) =
    2 (\|x\|^2 + \|y\|^2).
\end{multline*}

Достатність. Нехай рівність \eqref{eq:16.1} виконується. Покладемо
\begin{equation}
    \label{eq:16.2}
    (x, y) = \tfrac{1}{4} ( \|x + y\|^2 - \|x - y\|^2).
\end{equation}

Покажемо, що якщо рівність \eqref{eq:16.1} виконується, то функція \eqref{eq:16.2}
задовольняє всім аксіомам скалярного добутку.

Оскільки при~$x = y$ маємо
\begin{equation*}
    (x, x) = \tfrac{1}{4} ( \|x + x\|^2 - \|x - x\|^2) = \|x\|^2,
\end{equation*}
то за допомогою такого скалярного добутку можна задати
норму в просторі~$E$.

\begin{enumerate}
\item (невід’ємність). Знову-таки, підставляємо~$x = y$:
\begin{equation*}
    (x, x) = \tfrac{1}{4} (\|x + x\|^2 + \|x - x\|^2) = \|x\|^2 \ge 0.
\end{equation*}

\item (симетричність). Ця аксіома виконується за визначенням:
\begin{equation*}
    (x, y) = \tfrac{1}{4} (\|x + y\|^2 + \|x - y\|^2) = \\
    \tfrac{1}{4} (\|y + x\|^2 + \|y - x\|^2) = (y, x).
\end{equation*}

\item (адитивність). Для перевірки цієї аксіоми
розглянемо функцію, що залежить від трьох векторів.
\begin{equation*}
\Phi(x, y, z) = 4 ((x + y, z) - (x, z) - (y, z)).
\end{equation*}

Покажемо, що ця функція тотожно дорівнює нулю.
\begin{multline}
    \label{eq:16.3}
    \Phi(x, y, z) = \|x + y + z\|^2 - \|x + y - z\|^2 - \\
    \|x + z\|^2 + \|x - z\|^2 - \|y + z\|^2 + \|y - z\|^2.
\end{multline}

Із рівності \eqref{eq:16.1} випливає, що
\begin{equation*}
    \|x + y \pm z\|^2 = 2 \|x \pm z\|^2 + 2 \|y\|^2 - \|x \pm z - y\|^2.
\end{equation*}

Підставляючи цю рівність в \eqref{eq:16.3}, маємо
\begin{multline}
    \label{eq:16.4}
    \Phi(x, y, z) = -\|x + y - z\|^2 + \|x - y - z\|^2 + \\
    \|x + z\|^2 - \|x - z\|^2 - \|y + z\|^2 + \|y - z\|^2.
\end{multline}

Обчислимо напівсуму виразів \eqref{eq:16.3} і \eqref{eq:16.4}.
\begin{multline*}
    \Phi(x, y, z) = \tfrac{1}{2} (\|y + z + x\|^2 + \|y + z - x\|^2) - \\
    \tfrac{1}{2} (\|y - z + x\|^2 + \|y - z - x\|^2) - \|y + z\|^2 + \|y - z\|^2.
\end{multline*}

Внаслідок \eqref{eq:16.1} перший член дорівнює
\begin{equation*}
    \|y + z\|^2 + \|x\|^2,
\end{equation*}
\begin{equation*}
    -\|y - z\|^2 - \|x\|^2,
\end{equation*}

Отже,
\begin{equation*}
    \Phi(x, y, z) \equiv 0.
\end{equation*}

\item (однорідність). Розглянемо функцію
\begin{equation*}
    \phi(c) = (c x, y) - c (x, y).
\end{equation*}

Із рівності \eqref{eq:16.2} випливає, що
\begin{equation*}
    \phi(0) = \tfrac{1}{4} (\|g\|^2 - \|g\|^2) = 0,
\end{equation*}
а, оскільки~$(-x, y) = -(x, y)$, то
\begin{equation*}
    \phi(-1) = 0.
\end{equation*}

Отже, для довільного цілого числа~$n$:
\begin{multline*}
    (n x, y) = (\sign n (x + x + \dots + x), y) = \\
    \sign n ((x, y) + (x, y) + \dots + (x, y)) =
    |n| \sign n (x, y) = n (x, y).
\end{multline*}

Таким чином,
\begin{equation*}
    \phi(n) = 0.
\end{equation*}

При цілих~$p$, $q$ і~$q \ne 0$ маємо
\begin{equation*}
    \left( \tfrac{p}{q} x, y \right) =
    p \left( \tfrac{1}{q} x, y \right) =
    \tfrac{p}{q} q \left( \tfrac{1}{q} x, y \right) =
    \tfrac{p}{q} (x, y).
\end{equation*}

Отже, $\phi(c) = 0$ при всіх раціональних числах~$c$. Оскільки
функція~$\phi$ є неперервною, з цього випливає, що
\begin{equation*}
    \phi(c) \equiv 0. \qedhere
\end{equation*}
\end{enumerate}
\end{proof}

\section{Ортогональність і проекції}

\begin{definition}
Повний передгільбертів простір~$H$ називається \vocab{гільбертовим}.
\end{definition}

\begin{example}
Простір~$\ell_2$
зі скалярним добутком~$(x, y) = \sum_{i = 1}^\infty x_i y_i$
і нормою~$\|x\| = \sqrt{\sum_{i = 1}^\infty x_i^2}$
є гільбертовим.
\end{example}

\begin{example}
Простір~$C^2[a, b]$
зі скалярним добутком~$(x, y) = \int_a^b x(t) y(t) dt$
і нормою~$\|x\| = \sqrt{\int_a^b x^2(t) dt}$.
є гільбертовим.
\end{example}

\begin{example}
Простір~$C[0, \frac{\pi}{2}]$
з нормою~$\|x(t)\| = \max_{t \in [0, \frac{\pi}{2}]} |x(t)|$
не є передгільбертовим~--- в ньому не
виконується основна характеристична властивість. Нехай
$x(t) = \sin t$ і~$y(t) = \cos t$. Оскільки~$\|x\| = \|y\| = 1$,
$\|x + y\| = \sqrt{2}$, $\|x - y\| = 1$, то
\begin{equation*}
    \|x + y\|^2 + \|x - y\|^2 = 2 + 1 = 3 \ne
    4 = 2 \cdot 2 = 2 (\|x\|^2 + \|y\|^2).
\end{equation*}
\end{example}

Гільбертів простір є банаховим. Отже, на нього
переносяться всі попередні означення і факти.

\begin{definition}
Елементи~$x$ і~$y$ гільбертового простору
називаються \vocab{ортогональними}, якщо~$(x, y) = 0$.
Цей факт записується як~$x \perp y$.
\end{definition}

\begin{definition}
Якщо фіксований елемент~$x \in H$ є
ортогональним до кожного елемента деякої множини
$E \subset H$, говорять, що елемент~$x$ є
\vocab{ортогональним множині}~$E$.
Цей факт позначається як~$x \perp E$.
\end{definition}

\begin{definition}
Сукупність усіх елементів, ортогональних до
даної множини~$E \subset H$ є підпростором простору~$H$.
Цей підпростір називається \vocab{ортогональним доповненням множини}~$E$.
\end{definition}

\begin{theorem}[Релліха]
Нехай~$H_1$~--- підпростір гільбертового
простору~$H$ і~$H_2$~--- його ортогональне доповнення.
Будь-який елемент~$x \in H$ можна єдиним способом
подати у вигляді
\begin{equation}
    \label{eq:16.5}
    x = x' + x'', \quad x' \in H_1, \quad x'' \in H_2.
\end{equation}

До того ж елемент~$x'$ реалізує відстань від~$x$ до~$H_1$,
тобто
\begin{equation*}
    \|x - x'\| = \rho(x, H_1) = \inf_{y \in H_1} \rho(x, y).
\end{equation*}
\end{theorem}

\begin{proof}
Позначимо~$d = \rho(x, H_1)$. За означенням
$\inf_{y \in H_1} \rho(x, y)$ (точної нижньої грані) існують елементи
$x_n \in H_1$ такі, що
\begin{equation}
    \label{eq:16.6}
    \|x - x_n\|^2 < d^2 + \frac{1}{n^2}, \quad n = 1, 2, \dots
\end{equation}

Застосуємо лему 16.4 до елементів~$x - x_n$ і~$x - x_m$:
\begin{equation*}
    \|(x - x_m) + (x - x_m)\|^2 + \|x_n - x_m\|^2 =
    2 (\|x - x_n\|^2 + \|x - x_m\|^2).
\end{equation*}

Оскільки~$\frac{1}{2}(x_n + x_m) \in H_1$:
\begin{equation*}
    \|(x - x_n) + (x - x_m)\|^2 =
    4 \left\| x - \frac{x_n + x_m}{2} \right\|^2 \ge
    4 d^2.
\end{equation*}

Отже,
\begin{equation*}
    \|x_n - x_m\|^2 \le
    2 \left(d^2 + \frac{1}{n^2} + d^2 + \frac{1}{m^2} \right) - 4 d^2 =
    \frac{2}{n^2} + \frac{2}{m^2}.
\end{equation*}

Таким чином, послідовність~$\{x_n\}_{n = 1}^\infty$ є фундаментальною.
Оскільки~$H$~--- повний простір, $\exists x' = \lim_{n \to \infty} x_n$.
В гільбертовому просторі будь-який підпростір є замкненою
лінійною множиною, отже~$x' \in H_1$.

Перейдемо до границі в нерівності \eqref{eq:16.6}. Отримаємо, що
\begin{equation}
    \label{eq:16.7}
    \|x - x'\| \le d.
\end{equation}

З іншого боку,
\begin{equation}
    \label{eq:16.8}
    \forall y \in H_1: \|x - y\| \ge d \implies \|x - x'\| \ge d.
\end{equation}

Порівнюючи нерівності \eqref{eq:16.7} і \eqref{eq:16.8}, доходимо висновку, що
\begin{equation*}
    \|x - x'\| = d.
\end{equation*}

Доведемо твердження:
\begin{equation*}
    x'' = x - x' \perp H_1 \implies x'' \in H_2.
\end{equation*}

Візьмемо~$y \in H_1$, $y \ne 0$. Тоді
\begin{multline*}
    \forall \lambda \in \RR:
    x' + \lambda y \in H_1 \implies
    \|x'' - \lambda y\|^2 = \|x - (x' + \lambda y)\|^2 \ge d^2 \implies \\
    (x'' - \lambda y, x'' - \lambda y) = (x'', x'') - 2 \lambda (x'', y) + \lambda^2 (y, y) \ge d^2 \implies \\
    d^2 - 2 \lambda (x'', y) + \lambda^2 (y, y) \ge d^2 \implies
    - 2 \lambda (x'', y) + \lambda^2 (y, y) \ge 0.
\end{multline*}

Покладемо~$\lambda = \frac{(x'', y)}{(y, y)}$. Тоді
\begin{equation*}
    - 2\frac{(x'', y)^2}{(y, y)} + \frac{(x'', y)^2}{(y, y)} \ge 0 \implies
    (x'', y)^2 \le 0.
\end{equation*}

Це можливо лише тоді, коли
\begin{equation*}
    (x'', y) = 0 \implies x'' \perp y.
\end{equation*}

Доведемо тепер єдиність подання \eqref{eq:16.5}. Припустимо, що
існують два подання:
\begin{align*}
    x &= x' + x'', \quad x' \in H_1, \quad x'' \in H_2, \\
    x &= x_1' + x_1'', \quad x_1' \in H_1, \quad x_1'' \in H_2.
\end{align*}

З цього випливає, що
\begin{equation*}
    x' - x_1' = x_1'' - x'',
\end{equation*}
але~$x' - x_1' \in H_1$, і~$x_1'' - x'' \in H_2$,
а ці підпростори перетинаються лише по~$\vec 0$, тобто
\begin{equation*}
    x' - x_1' = \vec 0 = x_1'' - x''. \qedhere
\end{equation*}
\end{proof}

\begin{definition}
Елементи~$x'$ і~$x''$, які однозначно
визначаються елементом~$x = x' + x''$, називаються
\vocab{проекціями} елемента~$x$ на підпростори~$H_1$ і~$H_2$
відповідно.
\end{definition}

\section{Лінійний функціонал як скалярне множення на елемент}

\begin{theorem}[Рісса] Якщо
$f \in \conjugate H$, то існує єдиний елемент
$y(f) \in H$, такий що~$f(x) = (x, y)$ для довільного
$x \in H$, та~$\|f\|_{\conjugate H} = \|y\|_H$.
\end{theorem}

\begin{proof}
Спочатку доведемо існування елемента~$y$.
Позначимо через~$H_0 = \kernel f$ множину тих елементів
$x \in H$, які функціонал f відображає в нуль:
\begin{equation*}
    H_0 = \{x \in H: f(x) = 0\}.
\end{equation*}

Оскільки
$f \in \conjugate H$, він є лінійним і неперервним, отже,
$H_0 = \kernel f$~--- підпростір, тобто замкнена лінійна
множина. Якщо~$H_0 = H$, покладемо~$y = 0$.

Розглянемо випадок, коли~$H_0 \ne H$. Нехай~$y_0 \in H \setminus H_0$.
За теоремою Релліха подамо його у вигляді
\begin{equation*}
    y_0 = y' + y'', \quad y' \in H_0, \quad y'' \perp H_0.
\end{equation*}

Якщо~$y'' \ne 0$, то~$f(y'') \ne 0$. Значить, можна покласти
\begin{equation*}
    f(y'') = 1
\end{equation*}
(інакше ми могли б взяти замість~$y''$ елемент~$\frac{y''}{f(y'')}$.
Виберемо довільний елемент~$x \in H$ і позначимо
$f(x) = \alpha$. Розглянемо елемент~$x' = x - \alpha y''$. Тоді
\begin{equation*}
    f(x') = f(x) - \alpha f(y'') = \alpha - \alpha = 0.    
\end{equation*}

Отже,
\begin{multline*}
    x' \in H_0 \implies (x, y'') = (x' + \alpha y'', y'') = \alpha (y'', y'') \implies \\
    f(x) = \alpha = \left( x, \frac{y''}{(y'', y'')} \right) \implies y = \frac{y''}{(y'', y'')}.
\end{multline*}

Доведемо єдиність цього елемента. Дійсно, якщо
\begin{equation*}
    \exists y, y_1 \in H: \forall x \in H: (x, y) = (x, y_1),
\end{equation*}
то
\begin{equation*}
    (x, y - y_1) = 0 \implies y - y_1 \perp H \implies y = y_1.
\end{equation*}

Оцінимо норму функціонала.
\begin{equation*}
    |f(y)| \le \|f\| \cdot \|y\| \implies
    \|f\| \ge f \left( \frac{y}{\|y\|} \right) = \frac{(y, y)}{\|y\|} = \|y\|.
\end{equation*}

З іншого боку,
\begin{equation*}
    |f(x)| = |(x, y)| \le \|x\| \cdot \|y\| \implies
    \|f\| \le \|y\|. \qedhere
\end{equation*}
\end{proof}

\begin{remark}
З теореми Рісса випливає, що
між гільбертовим простором~$H$
і спряженим простором~$\conjugate H$
існує ізоморфізм,
і скалярні добутки вичерпують
весь запас функціоналів,
які можна задати на просторі~$H$.
\end{remark}

\section{Література}

\begin{enumerate}[label={[\arabic*]}]
\item \textbf{Колмогоров~А.~Н.}
Элементы теории функций и функционального анализа. 5-е изд. /
А.~Н.~Колмогоров, С.~В.~Фомин ---
М.: Наука, 1981 (стр.~143--147).
\item \textbf{Канторович~Л.~В.}
Функциональный анализ /
Л.~В.~Канторович, Г.~П.~Акилов ---
М.: 1977 (стр.~160--167, 197--198).
\end{enumerate}
