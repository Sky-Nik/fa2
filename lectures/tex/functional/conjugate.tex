\chapter{Спряжений простір}

Ввести топологію в лінійному просторі можна не лише за допомогою норми.

\section{Лінійні топологічні простори і неперевність функціоналів}

\begin{definition}
    Упорядкована четвірка~$(L, +, \cdot, \tau)$ називається \vocab{лінійним топологічним простором}, якщо
    \begin{enumerate}
        \item $(L, +, \cdot)$~--- дійсний лінійний простір;
        \item $(L, \tau)$~--- топологічний простір;
        \item операція додавання і множення на числа в~$L$ є
        неперервними, тобто
        \begin{enumerate}
            \item якщо~$z_0 = x_0 + y_0$, то для кожного околу~$U$ точки~$z_0$
            можна указати такі околи~$V$ і~$W$ точок~$x_0$ і~$y_0$ відповідно,
            що~$\forall x \in V$, $\forall y \in W$: $x + y \in U$;
            \item якщо~$\alpha_0 x_0 = y_0$, то для кожного околу~$U$ точки~$y_0$
            існує окіл~$V$ точки~$x_0$ і таке число~$\epsilon > 0$, що
           ~$\forall \alpha \in \RR: |\alpha - \alpha_0| < \epsilon$, $\forall x \in V$: $\alpha x \in U$.
        \end{enumerate}
    \end{enumerate}
\end{definition}

\begin{example}
    Всі нормовані простори є лінійними топологічними просторами.
\end{example}

\begin{remark}
    Оскільки будь-який окіл будь-якої точки~$x$ в лінійному топологічному просторі можна отримати зсувом околу нуля~$U$ шляхом операції~$U + x$, топологія в лінійному топологічному просторі повністю визначається локальною базою нуля.
\end{remark}

Спочатку доведемо деякі допоміжні факти щодо лінійних функціоналів, заданих на лінійному топологічному просторі~$L$.

\begin{definition}
    Функціонал, визначений на лінійному топологічному просторі~$L$, називається \vocab{неперервним}, якщо для будь-якого~$x_0 \in L$ і будь-якого~$\epsilon > 0$ існує такий окіл~$U$ елемента~$x_0$, що
    \begin{equation*}
        \forall x \in U: |f(x) - f(x_0)| < \epsilon.
    \end{equation*}
\end{definition}

\begin{lemma}
    Якщо лінійний функціонал~$f$ є неперервним в якійсь одній точці~$x_0$ лінійного топологічного простору~$L$, то він є неперервним на усьому просторі~$L$.
\end{lemma}

\begin{proof}
    Дійсно, нехай~$y$~--- довільна точка простору~$L$ і~$\epsilon > 0$. Необхідно знайти такий окіл~$V$ точки~$y$, щоб
    \begin{equation*}
        \forall z \in V: |f(z) - f(y)| < \epsilon.
    \end{equation*}
    
    Виберемо окіл~$U$ точки~$x_0$ так, щоб
    \begin{equation*}
        \forall x \in U: |f(x) - f(x_0)| < \epsilon.
    \end{equation*}
    
    Побудуємо окіл точки~$y$ шляхом зсуву околу~$U$ на елемент~$y - x_0$:
    \begin{equation*}
        V = U + (y - x_0) = \{ z \in L: z = u + y - x_0, u \in U\}.
    \end{equation*}
    
    Із того, що~$z \in V$, випливає, що~$z - y + x_0 \in U$, отже,
    \begin{equation*}
        |f(z) - f(y)| = |f(z - y)| = |f(z - y + x_0 - x_0)| = |f(z - y + x_0) - f(x_0)| < \epsilon.
    \end{equation*}
    
    Що і треба було довести. 
\end{proof}

\begin{corollary}
    Для того щоб перевірити неперервність лінійного функціонала в просторі, достатньо перевірити його неперервність в одній точці, наприклад, в точці~$0$.
\end{corollary}

\begin{remark}
    У скінчено-вимірному лінійному топологічному просторі будь-який лінійний функціонал є неперервним.
\end{remark}

\begin{theorem}
    Для того щоб лінійний функціонал~$f$ був неперервним на лінійному топологічному просторі~$L$, необхідно і достатньо, щоб існував такий окіл нуля в~$L$, на якому значення функціонала~$f$ є обмеженими в сукупності.
\end{theorem}

\begin{proof}
    \textbf{Необхідність.} З того що функціонал~$f$ є неперервним в точці~$0$, випливає що
    \begin{equation*}
        \forall \epsilon > 0: \exists U(0): \forall x \in U(0): |f(x)| < \epsilon.
    \end{equation*}
    Отже, його значення є обмеженими в сукупності на~$U(0)$.
    
    \textbf{Достатність}. Нехай~$U(0)$~--- такий окіл нуля, що
    \begin{equation*}
        \forall U(0): |f(x)| < C.
    \end{equation*}
    
    Крім того, нехай~$\epsilon > 0$. Тоді в околі нуля
    \begin{equation*}
        \tfrac{\epsilon}{C} U(0) = \{ x \in L: x = \tfrac{\epsilon}{C} y, y \in U(0) \}.
    \end{equation*}
    виконується нерівність~$|f(x)| < \epsilon$.
    
    Це означає, що функціонал~$f$ є неперервним в околі нуля, а значить в усьому просторі~$L$.
\end{proof}

Нехай~$E$~--- нормований простір. Нагадаємо, що спряженим простором~$\conjugate E$ називається сукупність усіх лінійних неперервних функціоналів, заданих на просторі~$E$ із нормою
\begin{equation*}
    \|f\| = \sup_{x \ne \vec 0} \frac{|f(x)|}{\|x\|} = \sup_{\|x\| \le 1} |f(x)|.
\end{equation*}

\begin{theorem}
    Для того щоб лінійний функціонал~$f$ був неперервним на нормованому просторі~$E$, необхідно і достатньо, щоб значення функціонала~$f$ були обмеженими в сукупності на одиничній кулі.
\end{theorem}

\begin{proof}
    \textbf{Необхідність.} Нормований простір~$E$ є лінійним топологічним простором. За \error теоремою 11.1 будь-яке значення неперервного лінійного функціонала~$f$ в деякому околі нуля є обмеженими в сукупності.
    \begin{equation*}
        \forall C > 0: \exists U(0): \forall x \in U(0): |f(x)| < C.
    \end{equation*}
    
    В нормованому просторі будь-який окіл нуля містить кулю.
    \begin{equation*}
        \exists S(0, r) \subset U(0).
    \end{equation*}
    
    Отже, значення функціонала~$f$ є обмеженими в сукупності в деякій кулі. Оскільки~$f$~--- лінійний функціонал, це еквівалентно тому, що значення функціонала~$f$ є обмеженими в сукупності в одиничній кулі, оскільки
    \begin{equation*}
        \forall x \in S(0, r): |f(x)| < C \implies
        \forall y = \tfrac{1}{r} x \in S(0, 1): |f(y)| < \tfrac{C}{r}.
    \end{equation*}

    \textbf{Достатність.} Оскільки значення функціонала~$f$ є обмеженими в сукупності в одиничній кулі, а одинична куля є околом точки~$0$, то за теоремою 11.1 він є неперервним в точці~$0$. Отже, лінійний функціонал~$f$ є неперервним в нормованому просторі~$E$. 
\end{proof}

\section{Топологія у спряженому просторі і його повнота}

На спряженому просторі можна ввести різні топології. Найважливішими з них є сильна і слабка топології.

\begin{definition}
    \vocab{Сильною топологією} в просторі~$\conjugate E$ називається топологія, визначена нормою в просторі~$\conjugate E$, тобто локальною базою нуля
    \begin{equation*}
        \{f \in \conjugate E: \|f\| < \epsilon\}.
    \end{equation*}
    де функціонали~$f$ задовольняють умову
    \begin{equation*}
        |f(x)| < \epsilon, \quad \forall x \in E: \|x\| \le 1.
    \end{equation*}
    а~$\epsilon$~--- довільне додатне число.
\end{definition}

\begin{theorem}
    Спряжений простір~$\conjugate E$ є повним.
\end{theorem}

\begin{proof}
    Нехай~$\{f_n\}_{n = 1}^\infty$~--- фундаментальна послідовність лінійних неперервних функціоналів, тобто
    \begin{equation*}
        \forall \epsilon > 0 \quad \exists N \in \NN: \quad \forall n, m \ge N \quad \|f_n - f_m\| < \epsilon.
    \end{equation*}

    Отже,
    \begin{equation}
        \label{eq:11.1}
        \forall x \in E: |f_n(x) - f_m(x)| \le \|f_n - f_m\| \cdot \|x\| < \epsilon \|x\|.
    \end{equation}
    
    Покладемо~$\forall x \in E$:
    \begin{equation*}
        f(x) = \lim_{n \to \infty} f_n(x).
    \end{equation*}
    
    Покажемо, що~$f$~--- лінійний неперервний функціонал.
    \begin{equation*}
        f(\alpha x + \beta y) =
        \lim_{n \to \infty} f_n(\alpha x + \beta y) =
        \lim_{n \to \infty} ( \alpha f_n(x) + \beta f_n(y)) =
        \alpha f(x) + \beta f(y).
    \end{equation*}
    
    Крім того, з нерівності \eqref{eq:11.1} випливає, що
    \begin{equation*}
        \forall x \in E \quad \lim_{m \to \infty} |f_n(x) - f_m(x)| = |f(x) - f(n)| < \epsilon \|x\|.
    \end{equation*}

    Це означає, що функціонал~$f - f_n$ є обмеженим. Оскільки він є лінійним і обмеженим, значить він є неперервним. Таким чином, функціонал~$f = f_n + (f - f_n)$ також є неперервним. Крім того, $\|f - f_n\| \le \epsilon$, $\forall n \ge N$, тобто~$f_n \to f$ при~$n \to \infty$ за нормою простору~$\conjugate E$. 
\end{proof}

\begin{remark}
    Зверніть увагу на те, що простір~$\conjugate E$ повний незалежно від того, чи є повним простір~$E$.
\end{remark}

\begin{example}
   ~$\conjugate c_0 = \ell_1$.
\end{example}

\begin{example}
   ~$\conjugate \ell_1 = m$.
\end{example}

\begin{example}
   ~$\conjugate \ell_p = \ell_q$, де~$\frac{1}{p} + \frac{1}{q} = 1$, $p, q > 1$.
\end{example}

\section{Другий спряжений простір і природне відображення}

\begin{definition}
    \vocab{Другим спряженим простором}~$\dconjugate E$ називається сукупність усіх лінійних неперервних функціоналів, заданих на просторі~$\conjugate E$.
\end{definition}

\begin{lemma}
    Будь-який елемент~$x_0 \in E$ визначає певний лінійний неперервний функціонал, заданий на~$\conjugate E$.
\end{lemma}

\begin{proof}
    Введемо відображення
    \begin{equation*}
        \pi: E \to \dconjugate E,
    \end{equation*}
    поклавши
    \begin{equation*}
        \phi_{x_0}(f) = f(x_0),
    \end{equation*}
    де~$x_0$~--- фіксований елемент із~$E$, а~$f$~--- довільний лінійний неперервний функціонал із~$\conjugate E$. Оскільки остання рівність ставить у відповідність кожному функціоналу~$f$ із~$\conjugate E$ дійсне число~$\phi_{x_0}(f)$, вона визначає функціонал на просторі~$\conjugate E$.
    
    Покажемо, що~$\phi_{x_0}$~--- лінійний неперервний функціонал, тобто він належить~$\dconjugate E$.
    
    Дійсно, функціонал~$\phi_{x_0}$ є лінійним, оскільки
    \begin{equation*}
        \phi_{x_0}(\alpha f_1 + \beta f_2) =
        \alpha f_1(x_0) + \beta f_2(x_0) =
        \alpha \phi_{x_0}(f_1) + \beta \phi_{x_0}(f_2).
    \end{equation*}
    
    Крім того, нехай~$\epsilon > 0$ і~$A$~--- обмежена множина в~$E$, що містить~$x_0$. Розглянемо в~$\conjugate E$ окіл нуля~$U(\epsilon, A)$:
    \begin{equation*}
        U(\epsilon, A) = \{f \in \conjugate E, x_0 \in A: |f(x_0)| \le \epsilon\}.
    \end{equation*}
    тобто
    \begin{equation*}
        U(\epsilon, A) = \{f \in \conjugate E, x_0 \in A: |\phi_{x_0}(f)| \le \epsilon\}.
    \end{equation*}

    З цього випливає, що функціонал~$\phi_{x_0}$ є неперервним в точці~$0$, а значить і на всьому просторі~$\conjugate E$. 
\end{proof}

\begin{definition}
    Відображення~$\pi: E \to \dconjugate E$, побудоване в \error лемі 11.2, називається \vocab{природнім відображенням} простору~$E$ в другий спряжений простір~$\dconjugate E$.
\end{definition}

\section{Рефлексивні простори}

\begin{definition}
    Якщо природне відображення~$\pi: E \to \dconjugate E$ є бієкцією і~$p(E) = \dconjugate E$, то простір~$E$ називається \vocab{напіврефлексивним}.
\end{definition}

\begin{definition}
    Якщо простір~$E$ є напіврефлексивним і відображення~$\pi: E \to \dconjugate E$ є неперервним, то простір~$E$ називається \vocab{рефлексивним}.
\end{definition}

\begin{remark}
    Якщо~$E$~--- рефлексивний простір, то природне відображення~$\pi: E \to \dconjugate E$ є ізоморфізмом.
\end{remark}

\begin{theorem}
    Якщо~$E$~--- нормований простір, то природне відображення~$\pi: E \to \dconjugate E$ є ізометрією.
\end{theorem}

\begin{proof}
    Нехай~$x \in E$. Покажемо, що
    \begin{equation*}
        \|x\|_E = \|\pi(x)\|_{\dconjugate E}.
    \end{equation*}
    
    Нехай~$f$~--- довільний ненульовий елемент простору~$\conjugate E$. Тоді
    \begin{equation*}
        |f(x)| \le \|f\| \cdot \|x\| \implies \|x\| \ge \frac{|f(x)|}{\|f\|}.
    \end{equation*}
    
    Оскільки ліва частина нерівності не залежить від~$f$, маємо
    \begin{equation*}
        \|x\| \ge \sup_{f \in \conjugate E, f \ne 0} \frac{|f(x)|}{\|f\|} =
        \|\pi(x)\|_{\dconjugate E}.
    \end{equation*}
     
    З іншого боку, внаслідок теореми Хана---Банаха, якщо~$x$~--- ненульовий елемент в нормованому просторі~$E$, то існує такий неперервний лінійний функціонал~$f$, визначений на~$E$, що
    \begin{equation*}
        \|f\| = 1, \quad f(x) = \|x\|
    \end{equation*}
    (визначаємо функціонал на одновимірному підпросторі формулою~$f(\alpha x) = \alpha \|x\|$, а потім продовжуємо без збільшення норми на весь простір). Отже, для кожного~$x \in E$ знайдеться такий ненульовий лінійний функціонал~$f$, що
    \begin{equation*}
        |f(x)| = \|f\| \cdot \|x\|
    \end{equation*}
    тому
    \begin{equation*}
        \|\pi(x)\|_{\dconjugate E} = \sup_{f \in \conjugate E, f \ne 0} \frac{|f(x)|}{\|f\|} \ge \|x\|.
    \end{equation*}
    
    Отже, $\|x\|_E = \|\pi(x)\|_{\dconjugate E}$.
\end{proof}

\begin{remark}
    Оскільки природне відображення нормованих просторів~$\pi: E \to \dconjugate E$ є ізометричним, поняття напіврефлексивних і рефлексивних просторів для нормованих просторів є еквівалентними.
\end{remark}

\begin{remark}
    Оскільки простір, спряжений до нормованого, є повним \error (теорема 11.3), будь-який рефлексивний нормований простір є повним.
\end{remark}

\begin{remark}
    Обернене твердження є невірним.
\end{remark}

\begin{example}
    Простір~$c_0$ є повним, але нерефлексивним, тому що спряженим до нього є простір~$\ell_1$, а спряженим до простору~$\ell_1$ є простір~$m$.
\end{example}

\begin{example}
    Простір неперервних функцій~$C[a, b]$ є повним, але нерефлексивним (більше того, немає жодного нормованого простору, для якого простір~$C[a, b]$ був би спряженим).
\end{example}

\begin{example}
    Приклад рефлексивного простору, що не збігається із своїм спряженим:
    \begin{equation*}
        \dconjugate \ell_p = \conjugate \ell_q = \ell_p, \quad p, q > 1, p \ne q, \frac{1}{p} + \frac{1}{q} = 1.
    \end{equation*}
\end{example}

\begin{example}
    Приклад рефлексивного простору, що збігається із своїм спряженим:
    \begin{equation*}
        \dconjugate{\ell_2} = \conjugate{\ell_2} = \ell_2.
    \end{equation*}
\end{example}

\section{Література}

\begin{enumerate}[label={[\arabic*]}]
\item \textbf{Садовничий~В.~А.}
Теория операторов /
В.~А.~Садовничий ---
М.: Изд-во Моск. ун-та, 1986 (стр.~112--123).
\item \textbf{Колмогоров~А.~Н.}
Элементы теории функций и функционального анализа. 5-е изд. /
А.~Н.~Колмогоров, С.~В.~Фомин ---
М.: Наука, 1981 (стр.~175--178, 182--192).
\end{enumerate}
