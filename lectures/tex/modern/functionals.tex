\chapter{Лінійні оператори і функціонали}

Нехай~$X$ і~$E$~--- топологічні векторні простори.

\section{Лінійні оператори, обмеженість і неперервність}

\begin{theorem}
    \label{th:linear-continuous-functional-is-continuous-at-zero}
    Лінійний оператор~$T: X \to E$ є неперервним тоді і лише тоді, коли він є неперервним в точці~$x = 0$.
\end{theorem}

\begin{proof}
    \textbf{Необхідність.} Неперервний оператор є неперервним у будь-якій точці простору, зокрема у нулі.

    \textbf{Достатність.} Припустимо, що оператор~$T$ є неперервним в нулі. Доведемо, що він є неперервним у довільній точці~$x_0 \in X$. Нехай~$V$~--- довільний окіл точки~$T x_0$ у просторі~$E$. Тоді~$V - T x_0$~--- окіл нуля в~$E$. За умовою теореми,~$T\inv(V - T x_0)$~--- окіл нуля в~$X$. Оскільки оператор~$T$ є лінійним, маємо
    \begin{equation*}
        T\inv(V) = T\inv(V - T x_0) + x_0,
    \end{equation*}
    отже~$T\inv(V)$~--- окіл точки~$x_0$.
\end{proof}

\begin{definition}
    Лінійний оператор~$T: X \to E$ називається \vocab{обмеженим}, якщо образ будь-якої обмеженої множини під дією~$T$ в~$X$ є обмеженою множиною в~$E$.
\end{definition}

\begin{theorem}
    \label{th:linear-continuous-functional-is-bounded}
    Кожний неперервний лінійний оператор~$T: X \to E$ є обмеженим.
\end{theorem}

\begin{proof}
    Нехай~$A$~--- обмежена множина в~$X$. Доведемо обмеженість множини~$T(A)$. Нехай~$V$~--- довільний окіл нуля в~$E$ і~$U$~--- такий окіл нуля в~$X$, що~$T(U) \subset V$. Оскільки~$A$~--- обмежена множина, то існує таке число~$N > 0$, що~$\forall t > N$~$A \subset t U$. Тоді
    \begin{equation*}
        \forall t > N \quad T(A) \subset t T(U) \subset t V. \qedhere
    \end{equation*}
\end{proof}

\begin{theorem}
    \label{th:boundness-on-any-neighbourhood-implies-continuity}
    Нехай оператор~$T: X \to E$ переводить деякий окіл~$U$ простору~$X$ в обмежену множину. Тоді оператор~$T$ є неперервним.
\end{theorem}

\begin{proof}
    Нехай~$T(U)$~--- обмежена множина. Для довільного околу~$V$ нуля в~$E$ існує число~$t > 0$, що~$T(U) \subset t U$. Тоді~$t\inv U \subset T\inv(V)$, тобто~$T\inv(V)$ є околом нуля у просторі~$X$.
\end{proof}

\section{Лінійні функціонали і їхні ядра}

\begin{theorem}
    \label{th:linear-functional-continuity-equivalent-conditions}
    Для ненульового лінійного функціонала~$f$, заданого на топологічному просторі~$X$, наступні умови є еквівалентними.
    \begin{enumerate}
        \item Функціонал~$f$ є неперервним.
        \item Ядро функціонала~$f$ є замкненим.
        \item Ядро функціонала~$f$ не є щільним в~$X$.
        \item Існує окіл нуля~$U$: $f(U)$~--- обмежена множина.
    \end{enumerate}
\end{theorem}

\begin{proof}
   ~$1 \implies 2$.~$\kernel f = f\inv(0)$. Оскільки~$\{0\}$~--- замкнена множина, а~$f$~--- неперервний функціонал, то, оскільки прообраз замкненої множини під дією неперервного функціонала є замкненим,~$\kernel f$ є замкненою множиною.

   ~$2 \implies 3$. (Від супротивного.) Якщо ядро функціонала є замкненим і щільним в~$X$, то~$\kernel f = X$, тобто~$f \equiv 0$, але за умовою теореми~$f$~--- ненульовий функціонал.

   ~$3 \implies 4$. Нехай ядро не є щільним. Тоді існує точка~$x \in X$ і врівноважений окіл нуля~$U$, такі що~$(U + x) \cap \kernel f = \emptyset$. Це значить, функціонал~$f$ в жодній точці~$y \in U$ не може набувати значення~$-f(x)$. Отже,~$f(U)$~--- врівноважена множина чисел, що відрізняється від числової прямої (точніше, відрізок, симетричний відносно нуля).

   ~$4 \implies 1$. Випливає з \cref{th:boundness-on-any-neighbourhood-implies-continuity}
\end{proof}


Позначимо через~$\conjugate X$ множину усіх неперервних лінійних функціоналів на~$X$.

\section{Скінченновимірні простори і координатні функціонали}

\begin{definition}
    Нехай~$\{x_k\}_{k = 1}^\infty$~--- базис банахового простору~$X$ і~$x \in X$. Коефіцієнти розкладу~$f_n(x)$ елемента~$x$ по базису~$\{f_k\}_{k = 1}^\infty$ називаються \vocab{координатними функціоналами}, що визначені на просторі~$X$: $x = \sum_{k = 1}^\infty f_n(x) x_k$.
\end{definition}

\begin{theorem}
    \label{th:finite-dimensional-hausdorff-tvs-properties}
    Нехай~$X$~--- хаусдорфовий ТВП із~$\dim X = n$. Тоді:
    \begin{enumerate}
        \item Будь-який лінійний функціонал на~$X$ є неперервним.
        \item Для будь-якого топологічного векторного простору~$E$ будь-який лінійний оператор~$T: X \to E$ є неперервним.
        \item Простір~$X$ є ізоморфним~$n$-вимірному гільбертовому простору~$\ell_2^n$.
        \item Простір~$X$ є повним.
    \end{enumerate}
\end{theorem}

\begin{proof}
    % Зазначимо, що при фіксованому~$n$ мають місце імплікації~$1 \implies 2 \implies 3 \implies 4$.
   ~$1 \implies 2$. Обираючи в~$X$ базис~$\{x_k\}_{k = 1}^n$ з координатними функціоналами~$\{f_k\}_{k = 1}^n$, оператор~$T$ можна подати у вигляді
    \begin{equation*}
        T(x) = T \left( \sum_{k = 1}^n f_k(x) x_k \right) = \sum_{k = 1}^n f_k(x) T x_k. 
    \end{equation*}
    
    Отже, обчислення~$T(x)$ зводиться до обчислення скалярів~$f_k(x)$, де~$f$~--- неперервний функціонал, множенню їх на фіксовані вектори~$Tx_k$ і додаванню добутків. В результаті отримуємо неперервний оператор~$T$.
    
   ~$2 \implies 3$. Оскільки обидва простори~$X$ і~$\ell_2^n$ мають однакову розмірність~$n$, то існує лінійна бієкція~$T: X \to \ell_2^n$. За умовою оператори~$T$ і~$T\inv$ є неперервними, отже, існує ізоморфізм~$T: X \to \ell_2^n$.
    
   ~$3 \implies 4$. Випливає з повноти простору~$\ell_2^n$.

   ~$4 \implies 1$. Скористаємось математичною індукцією по~$n$. При~$n = 0$ простір~$X$ містить лише нульовий елемент, тому твердження є тривіальним. Доведемо тепер крок індукції: нехай~$\dim X = n + 1$ і~$f$~--- ненульовий функціонал на~$X$. Тоді~$\dim \kernel f = n$. За імплікаціями~$1 \implies 2 \implies 3 \implies 4$ отримуємо, що~$\kernel f$~--- повний простір. Отже,~$\kernel f$ є замкнений в~$X$ і за \cref{th:linear-functional-continuity-equivalent-conditions} функціонал~$f$ є неперервним.
\end{proof}

\section{Література}

\begin{enumerate}[label={[\arabic*]}]
\item \textbf{Кадец~В.~М.}
Курс функционального анализа /
В.~М.~Кадец~---
Х.: ХНУ им.~В.~Н.~Каразина, 2006. (стр.~507--510).
\end{enumerate}
