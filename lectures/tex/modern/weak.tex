\chapter{Слабка топологія}

\section{Слабка топологія: означення і властивості}

\begin{definition}
    Нехай~$X$~--- лінійний простір, $X'$~--- алгебраїчно спряжений до нього простір (тобто простір усіх лінійних функціоналів, заданих на~$X$), $E \subset X'$~--- деяка підмножина. 
    
    \vocab{Слабкою топологією} на~$X$, породженою множиною функціоналів~$E$, називається найслабкіша топологія, в якій функціонали з~$E$ є неперервними.
\end{definition}

\begin{remark}
    Ця топологія є частковим випадком топології, породженою сім'єю відображень, тому для неї використовується те ж позначення~$\sigma(X, E)$.
\end{remark}

Для будь-якого скінченого набору функціоналів~$G = (g_1, g_2, \dots, g_n)$ і будь-якого~$\epsilon > 0$ введемо позначення
\begin{equation*}
    U_{G, \epsilon} = \bigcap_{g \in G} \{x \in X: |g(x)| < \epsilon\} = \{x \in X: \max_{g \in G} |g(x)| < \epsilon\}.
\end{equation*}

Сім'я множин вигляду~$U_{G,\epsilon}$, де~$G = (g_1, g_2, \dots, g_n) \subset E$ і~$\epsilon > 0$, утворює базу околів нуля топології~$\sigma(X, E)$. Базу околів будь-якого елемента~$x_0 \in X$ утворюють множини вигляду 
\begin{equation*}
    \bigcap_{g \in G} \{x \in X: |g(x - x_0)| < \epsilon\} = x_0 + U_{G, \epsilon}.
\end{equation*}

Звідси випливає, що топологія~$\sigma(X, E)$~--- це локально-опукла топологія, що породжена сім'єю напівнорм~$p_G(x) = \max_{g \in G} |g(x)|$, де~$G$ пробігає усі скінчені підмножини множини~$E$. Для того щоб ця топологія була віддільною, необхідно і достатньо, щоб сім'я функціоналів~$E$ розділяла точки простору~$X$.

Як зазначалося в попередніх лекціях, фільтр~$\frak F$ на~$X$ збігається в топології~$\sigma(X, E)$ до елемента~$x$ тоді і лише тоді, коли~$\lim_{\frak F} f = f(x)$ для всіх~$f \in E$. Зокрема, цей критерій збіжності є слушним і для послідовностей: $x_n \to x$ в топології~$\sigma(X, E)$, якщо~$f(x_n) \to f(x)$ для всіх~$f \in E$.

\section{Леми про перетин ядер і обмеженість на підпросторі}

\begin{lemma}
    \label{lem:kernel-intersection-supset-linear-combination}
    Нехай~$f, \{f_k\}_{k = 1}^n$~--- лінійні функціонали на~$X$ і~$\kernel f \supset \bigcap_{k = 1}^n \kernel f_k$. Тоді~$f \in \linear(f_1, f_2, \dots, f_n)$.
\end{lemma}

\begin{proof}
    Застосуємо індукцію по~$n$, поклавши як базу~$n = 1$.

    Якщо~$f_1 = 0$, то~$\kernel f \supset \kernel f_1 = X$, тобто~$f = 0$.

    Якщо~$f_1 \ne 0$, то~$Y = \kernel f_1$~--- це гіперплощина в~$X$. Отже, існує вектор~$e \in X \setminus Y$ такий, що~$\linear(e, Y) = X$. Позначимо~$a = f(e)$ і~$b = f_1(e)$. Функціонал~$f - ab\inv f_1$ дорівнює нулю як на~$Y$, так і точці~$e$. Отже, функціонал~$f - ab\inv f_1$ дорівнює нулю на всьому просторі~$X = \linear(e, Y)$, тобто~$f \in \linear(f_1)$.

    Індукційний перехід~$n \to n + 1$. Розглянемо підпростір~$Y = \bigcap_{k = 1}^n \kernel f_k$. Умова~$\kernel f \supset \bigcap_{k = 1}^{n + 1} \kernel f_k$ означає, що ядро звуження функціонала~$f$ на~$Y$ містить ядро звуження функціонала~$f_{n + 1}$ на~$Y$. Отже (випадок~$n = 1$), існує такий скаляр~$\alpha$, що~$f - \alpha f_{n + 1}$ дорівнює нулю на всьому~$Y = \bigcap_{k = 1}^n \kernel f_k$. Отже,
    \begin{equation*}
        \kernel (f - \alpha f_{n + 1}) \supset Y = \bigcap_{k = 1}^n \kernel f_k.
    \end{equation*}

    За припущенням індукції, $f - \alpha f_{n + 1} \in \linear(f_1, \dots, f_n)$, тобто~$f \in \linear(f_1, \dots, f_{n + 1})$.
\end{proof}

\begin{lemma}
    \label{lem:linear-subspace-bounded-implies-zero}
    Нехай~$Y$~--- підпростір лінійного простору~$X$, $f \in X'$ і існує таке~$a > 0$, що~$|f(y)| \le a$ на всьому підпросторі~$Y$. Тоді~$f(y) = 0$ для всіх~$y \in Y$.
\end{lemma}

\begin{proof}
    Нехай існує~$y_0 \in Y$ такий що~$f(y_0) \ne 0$.
    
    Тоді на елементі~$y = 2 a f(y_0)\inv y_0 \in Y$ маємо~$|f(y)| = 2 a > a$.
\end{proof}

\section{Неперервність функціоналів у слабкій топології}

\begin{theorem}
    Функціонал~$f \in X'$ є неперервним в топології~$\sigma(X, E)$, тоді і лише тоді, коли~$f \in \linear(E)$.
\end{theorem}

\begin{remark}
    Зокрема, якщо~$E \subset X'$~--- лінійний підпростір, множина~$\conjugate{(X, \sigma(X, E))}$ усіх функціоналів, неперервних в топології~$\sigma(X, E)$ на~$X$, збігається з~$E$.
\end{remark}

\begin{proof}
    \textbf{Необхідність.} За означенням топології~$\sigma(X, E)$, усі елементи множини~$E$ є функціоналами, неперервними в топології~$\sigma(X, E)$. Отже, неперервними будуть і їх лінійні комбінації.

    \textbf{Достатність.} Нехай функціонал~$f \in X'$ є неперервним в~$\sigma(X, E)$. Тоді існує скінчена множина функціоналів~$G = \{g_1, g_2, \dots, g_n\} \subset E$ і таке~$\epsilon > 0$, що в околі
    \begin{equation*}
        U_{G, \epsilon} = \{x \in X: \max_{g \in G} |g(x)| < \epsilon\}.
    \end{equation*}
    усі значення функціонала~$f$ є обмеженими за модулем деяким числом~$a > 0$. Цим же число будуть обмежені значення функціонала на підпросторі
    \begin{equation*}
        Y = \bigcap_{k = 1}^n \kernel f_k \subset U_{G, \epsilon}.
    \end{equation*}

    За \cref{lem:linear-subspace-bounded-implies-zero} функціонал~$f$ обертається на нуль на просторі~$Y$, що за \cref{lem:kernel-intersection-supset-linear-combination} значить, що~$f \in \linear(g_1, g_2, \dots, g_n) \subset \linear(E)$.
\end{proof}

\section{Література}

\begin{enumerate}[label={[\arabic*]}]
\item \textbf{Кадец~В.~М.}
Курс функционального анализа /
В.~М.~Кадец~---
Х.: ХНУ им.~В.~Н.~Каразина, 2006. (стр.~516--518).
\end{enumerate}
