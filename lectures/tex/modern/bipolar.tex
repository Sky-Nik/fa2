\chapter{Біполяра}

\section{Абсолютна опуклість і біполяра}

\begin{definition}
    Нехай~$X$~--- лінійний простір.
    
    \vocab{Абсолютно опуклою комбінацією} набору елементів~$\{x_k\}_{k = 1}^n \subset X$ називається будь-яка сума вигляду~$\sum_{k = 1}^n \lambda_k x_k$, де~$\sum_{k = 1}^n |\lambda_k| \le 1$.
\end{definition}

\begin{definition}
    \vocab{Абсолютно опуклою оболонкою} множини~$A$ в лінійному просторі~$X$ називається множина усіх абсолютно опуклих комбінацій скінченнлшл числа елементів множини~$A$. Позначається абсолютно опукла оболонка як~$\aconvex A$.
\end{definition}

Нехай~$(X, Y)$~--- дуальна пара, $A \subset X$. Тоді~$A\polar \subset Y$ і у цієї множини теж можна розглянути поляру.

\begin{definition}
    Множина~$(A\polar)\polar \subset X$ називається \vocab{біполярою} множини~$A \subset X$ і позначається як~$A\bipolar$.
\end{definition}

\begin{theorem}
    Біполяра~$A\bipolar$ множини~$A \subset X$ збігається з~$\sigma(X, Y)$-замиканням абсолютно опуклої оболонки множини~$A$.
\end{theorem}

\begin{proof}
    Зауважимо, що~$A\bipolar \supset A$. Дійсно, якщо~$x \in A$, то за означенням множини~$A\polar$:
    \begin{equation*}
        \forall y \in A\polar \quad |\langle x, y \rangle| \le 1.
    \end{equation*}

    Це означає, що~$X \in A\bipolar$.

    Далі, біполяра~--- частковий приклад поляри. Отже, відповідно до пункту 6) теореми 13.2~$A\bipolar$~---опукла врівноважена~$\sigma(X, Y)$-замкнена множина. Відповідно, $A\bipolar \supset \claconvex A$.
    
    Для доведення оберненого включання візьмемо довільну точку~$x_0 \in X \setminus \claconvex A$ і переконаємося, що~$x_0 \notin A\bipolar$. Дійсно, оскільки~$x_0 \in \claconvex A$ і~$\claconvex A$~--- це опукла врівноважена~$\sigma(X, Y)$-замкнена множина, тому за теоремою Хана---Банаха (\cref{th:locally-convex-hahn-banach-theorem}) існує такий~$\sigma(X, Y)$-неперервний лінійний функціонал~$y$ на~$X$, що
    \begin{enumerate}
        \item $|y(x)| \le 1$~$\forall x \in \claconvex A$;
        \item $|y(x_0)| > 1$.
    \end{enumerate}

    Будь-який~$\sigma(X, Y)$-неперервний лінійний функціонал~--- це елемент простору~$Y$. Умова 1 означає, що~$y \in (\claconvex A)\polar \subset A\polar$. Тоді друга умова означає, що~$x_0 \notin A\bipolar$.
\end{proof}

\begin{corollary}
    Якщо~$A \subset X$~---~$\sigma(X, Y)$-замкнена врівноважена множина, то~$A\bipolar = A$. Зокрема, $B\polar{}\bipolar = B\polar$~$\forall B \subset Y$.
\end{corollary}

\begin{corollary}
   ~$A^{\perp\perp} = \cllinear A$~$\forall A \subset X$. Якщо~$A$~--- лінійний підпростір, то~$A^{\perp\perp} = \closure A$. Нарешті, $B^{\perp\perp\perp} = B^\perp$~$\forall B \subset Y$.
\end{corollary}

\begin{proof}
    \begin{equation*}
        A^{\perp\perp} = (A^\perp)^\perp = ((\linear A)^\perp)^\perp) = (\linear A)\bipolar = \cllinear A. \qedhere
    \end{equation*}
\end{proof}

\begin{corollary}
    Якщо~$A_1, A_2 \subset X$~---~$\sigma(X, Y)$-замкнені врівноважені множини, то~$A_1 = A_2 \iff A_1\polar = A_2\polar$. Якщо до того ж~$A_1 = A_2$~--- підпростори, то~$A_1 = A_2 \iff A_1^\perp = A_2^\perp$
\end{corollary}

\begin{proof}
    Очевидно, що~$A_1 = A_2 \implies A_1\polar = A_2\polar$.
    
    З іншого боку, якщо~$A_1\polar = A_2\polar$, то~$A_1\bipolar = A_2\bipolar$ і можна застосувати теорему про біполяру.
\end{proof}

\begin{theorem}
    Нехай~$(X, Y)$~--- дуальна пара і~$A \subset X$.
    Тоді наступні умови є еквівалентними:
    \begin{enumerate}
        \item Множина функціоналів~$A \subset X$ розділяє точки простору~$X$.
        \item $A^\perp = \{0\}$;
        \item $A^{\perp\perp} = X$;
        \item Лінійна оболонка множини~$A$ є~$\sigma(Y, X)$-щільною в~$Y$.
    \end{enumerate}
\end{theorem}

\begin{proof}
   ~$1 \implies 2$. Включення~$A^\perp \supset \{0\}$ виконано завжди. Якщо ж~$x \in X \setminus \{0\}$, то за умовою існує~$y \in A$ такий, що~$\langle x, y \rangle \ne 0$. У цьому випадку~$x \notin A^\perp$.

   ~$2 \implies 1$. Нехай~$x \in X \setminus \{0\}$. Тоді~$x \notin A^\perp$, отже існує~$y \in A$, такий що~$\langle x, y \rangle \ne 0$.

   ~$2 \iff 3$. Оскільки~$A^\perp$ і~$\{0\}$~--- це~$\sigma(X, Y)$-замкнені підпростори, можна скористатися наслідком 11.3.

   ~$3 \iff 4$. За наслідком 11.2~$A^{\perp\perp} = \cllinear A$.
\end{proof}

\section{Література}

\begin{enumerate}[label={[\arabic*]}]
\item \textbf{Кадец~В.~М.}
Курс функционального анализа /
В.~М.~Кадец~---
Х.: ХНУ им.~В.~Н.~Каразина, 2006. (стр.~533--535).
\end{enumerate}
