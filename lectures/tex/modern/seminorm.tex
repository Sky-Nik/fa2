\chapter{Напівнорми і топології}

Нехай~$X$~--- топологічний векторний простір.

\section{Локальна опуклість, опуклі комбінація і оболонка}

\begin{definition}
    ТВП~$X$ називається \vocab{локально опуклим}, якщо для будь-якого околу нуля~$U$ існує опуклий окіл нуля~$V$, що міститься в~$U$.
\end{definition}

\begin{remark}
    Інакше кажучи, топологічний векторний простір~$X$ є локально опуклим, якщо система околів нуля~$\frak R_0$ містить базу, що складається з опуклих множин.
\end{remark}

\begin{definition}
    Нехай~$\{x_k\}_{k = 1}^n$~--- довільний скінчений набір елементів лінійного простору~$X$. Елемент вигляду~$x = \sum_{k = 1}^n \lambda_k x_k$ називається \vocab{опуклою комбінацією} елементів~$x_k$, якщо~$\lambda_k > 0$,~$\forall k = 1, \dots, n$ і~$\sum_{k = 1}^n \lambda_k = 1$.
\end{definition}

\begin{definition}
    Нехай~$A$~--- довільна підмножина лінійного простору~$X$. Множина усіх опуклих комбінацій елементів з~$A$ називається \vocab{опуклою оболонкою} множини~$A$ і позначається як~$\convex A$.
\end{definition}

\begin{definition}
    Нагадаємо, що підмножина~$A \subset X$ називається \vocab{урівноваженою}, якщо для будь-якого скаляра~$\lambda$ із~$|\lambda| \le 1$ виконане включення~$\lambda A \subset A$.
\end{definition}

\begin{theorem}
    Кожний опуклий окіл нуля~$U$ містить опуклий врівноважений відкритий окіл нуля~$V$.
\end{theorem}

\begin{proof}
    За \cref{th:tvs-zero-neighbourhood-properties} у кожному відкритому околі нуля~$U$ міститься відкритий врівноважений окіл нуля~$V$.
    
    \begin{enumerate}
        \item Покажемо, що~$\convex V \subset U$. Опуклість цієї множини є очевидною (за означенням опуклої оболонки).

        \item Покажемо, що~$\convex V \subset \frak R_0$.~$V \in \frak R_0, V \subset \convex V \implies \convex V \in \frak R_0$.

        \item Покажемо, що~$\convex V$ є врівноваженим околом. Нехай~$|\lambda| \le 1$.~$V$~--- врівноважений окіл нуля~$\implies$~$\lambda V \subset V$~$\implies$~$\lambda \convex V = \convex (\lambda V) \subset \convex V$

        \item Покажемо, що~$\convex V$ є відкритою множиною.~$V \in \tau$, операції множення на скаляр і суми множин замкнені відносно відкритих множин~$\implies$~$\sum_{k = 1}^n \lambda_k V \in \tau$, де~$n \in \NN$,~$\lambda_k > 0$ і~$\sum_{k = 1}^n \lambda_k = 1$~$\implies$~$\convex V = \bigcup_{n = 1}^\infty \sum_{k = 1}^n \lambda_k V \in \tau$. \qedhere
    \end{enumerate}
\end{proof}

\section{Напівнорми, одиничні кулі і функіонал Мінковського}

\begin{definition}
    Функція~$p: X \to \RR$ називається \vocab{напівнормою}, якщо
    \begin{enumerate}
        \item $p(x) \ge 0$,~$\forall x \in X$;
        \item $p(\lambda x| = |\lambda| p(x)$,~$\forall x \in X, \lambda \in \RR$;
        \item $p(x + y) \le p(x) + p(y)$,~$\forall x, y \in X$.
    \end{enumerate}
\end{definition}

\begin{remark}
    Напівнорма відрізняється від норми тим, що вона напівнорма маже дорівнювати нулю на деяких ненульових елементах~$x \in X$.
\end{remark}

\begin{definition}
    \vocab{Одиничною кулею} напівнорми~$p$ називається множина~$B_p = \{x \in X: p(x) < 1\}$.
\end{definition}

\begin{remark}
    Множина~$B_p$ є опуклою врівноваженою множиною.
\end{remark}

\begin{definition}
    \vocab{Функціоналом Мінковського} опуклої поглинаючої множини в лінійному просторі~$X$ називається дійсна функція, задана на~$X$ формулою
    \begin{equation*}
        \phi_A(x) = \inf \{ t > 0: t\inv x \in A\}.
    \end{equation*}
\end{definition}

\begin{remark}
    Функціонал~$\phi_A$ пов'язаний з множиною~$A$ такими співвідношеннями:
    \begin{enumerate}
        \item $x \in A \implies \phi_A(x) \le 1$;
        \item $\phi_A(x) < 1 \implies x \in A$.
    \end{enumerate}
\end{remark}

\begin{remark}
    Якщо~$A$~--- опукла поглинаюча множина в лінійному просторі~$X$, то~$\phi_A$~--- опуклий функціонал, що набуває невід'ємні значення.
\end{remark}

\begin{theorem}
    Напівнорма~$p$ на топологічному векторному просторі~$X$ є неперервною тоді і лише тоді, коли~$B_p$~--- окіл нуля.
\end{theorem}

\begin{proof}
    \textbf{Необхідність.}~$B_p = p\inv (-1,1)$~--- прообраз відкритої множини. Якщо~$p$~--- неперервна функція, то прообраз відкритої множини є відкритим.

    \textbf{Достатність.} Нехай~$B_p$~--- окіл нуля. Доведено неперервність напівнорми. Для будь-якого~$x \in X$ і будь-якого~$\epsilon > 0$ треба знайти такий окіл~$U$ точки~$x$, що
    \begin{equation*}
        p(U) \subset (p9x) - \epsilon, p(x) + \epsilon).
    \end{equation*}

    Таким околом є~$U = x + \epsilon B_p$. Дійсно,
    \begin{equation*}
        \forall y \in U \quad y = x + \epsilon z, \quad p(z) < 1.
    \end{equation*}

    Отже, за нерівністю трикутника
    \begin{equation*}
        p(x) - \epsilon < p(y) < p(x) + \epsilon.
    \end{equation*}
\end{proof}

\section{Лінійно-опукла топологія породжена сім'єю напівнорм}

\begin{definition}
    Нехай~$G$~--- сім'я напівнорм на лінійному просторі~$X$. Позначимо через~$\frak D_G$ систему усіх скінчених перетинів множин вигляду~$r B_p$, де~$p \in G$ і~$r > 0$.
    
    \vocab{Лінійно-опуклою топологією}, породженою сім'єю напівнорм~$G$, називається топологія~$\tau_G$ на~$X$, у якій базою околів точки~$x \in X$ є сім'я множин вигляду~$x + U$, де~$U \in \frak D_G$.
\end{definition}

\begin{remark}
    Тобто,~$\frak D_G$ є базою околів нуля топології~$\tau_G$.
\end{remark}

\begin{definition}
    Сім'я напівнорм~$G$ називається \vocab{невиродженою}, якщо для будь-якого~$x \in X \setminus \{0\}$ існує~$p \in G$ з~$p(x) \ne 0$.
\end{definition}

\begin{theorem}
    Нехай~$G$~--- сім'я напівнорм на лінійному просторі~$X$. Тоді мають місце такі твердження:
    \begin{enumerate}
        \item топологія~$\tau_G$, породжена сім'єю~$G$, узгоджується з лінійною структурою і є локально опуклою;
        \item топологія~$\tau_G$ є віддільною тоді і лише тоді, коли сім'я напівнорм~$G$ є невиродженою;
        \item топологічний векторний простір~$X$ є локально опуклим тоді і лише тоді, коли його топологія породжується деякою сім'єю напівнорм.
    \end{enumerate}
\end{theorem}

\begin{theorem}
    Нехай~$X$~--- топологічний векторний простір, а~$f$~--- лінійний функціонал на~$X$. Для неперервності функціонала~$f$ необхідно і достатньо, щоб існувала така неперервна напівнорма~$p$ на~$X$, що~$|f(x)| \le p(x)$~$\forall x \in X$.
\end{theorem}

\begin{proof}
    \textbf{Необхідність.} Нехай~$f$~--- неперервний. Тоді шукана напівнорма задається формулою~$p(x) = |f(x)|$.

    \textbf{Достатність.} Нехай~$|f(x)| \le p(x)$~$\forall x \in X$ і~$p$~--- неперервна напівнорма. Тоді функціонал~$f$ є обмеженим в околі нуля~$B_p$. 
\end{proof}

\begin{theorem}[теорема Хана-Банаха в локально опуклих просторах]
    \label{th:locally-convex-hahn-banach-theorem}
    Нехай~$f$~--- лінійний непепервний функціонал, заданий на підпросторі~$Y$ локально опуклого простору~$X$. Тоді функціонал~$f$ можна продовжити на весь простір~$X$ зі збереженням його лінійності і неперервності.
\end{theorem}

\section{Література}

\begin{enumerate}[label={[\arabic*]}]
\item \textbf{Кадец~В.~М.}
Курс функционального анализа /
В.~М.~Кадец~---
Х.: ХНУ им.~В.~Н.~Каразина, 2006. (стр.~512--515).
\end{enumerate}
