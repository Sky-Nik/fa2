\chapter{Лінійні простори}

Лінійна система є алгебраїчною структурою, яка абстрагує
властивості, пов’язані із додаванням та множенням векторів
евклідова простору на скаляр.

\section{Лінійні простори і функціонали}

\begin{definition}
\vocab{Дійсним лінійним (векторним) простором}
називається упорядкована трійка $(E, +, \cdot)$, що складається з
множини $E$, елементи якого називаються \vocab{векторами},
операції додавання і операції множення на дійсні числа, якщо
для кожних двох її елементів $x$ та $y$ визначено їх суму
$x + y \in E$, і для будь-якого $x$ та дійсного числа $\lambda$ визначено
добуток $\lambda x \in E$, які задовольняють аксіоми лінійного
простору:
\begin{enumerate}
\item $\exists \vec 0 \in E$, що $x + \vec 0 = x$ для довільного $x \in E$;
\item $\forall x \in E$ $\exists (-x) \in E$: $x + (-x) = 0$;
\item $(x + y) + z = x + (y + z)$ (асоціативність додавання);
\item $x + y = y + x$ (комутативність додавання);
\item $(\lambda + \mu) x = \lambda x + \mu x$ (дистрибутивність);
\item $\lambda (x + y) = \lambda x + \lambda y$ (дистрибутивність);
\item $(\lambda \mu) x = \lambda (\mu x)$ (асоціативність множення);
\item $1 \cdot x = x$.
\end{enumerate}
\end{definition}

\begin{remark}
Властивості 1--4 означають, що лінійний простір є
\vocab{абелевою (комутативною) групою}.
\end{remark}

\begin{example}
Сукупність дійсних чисел $\RR$ із звичайними
арифметичними операціями додавання та множення є
лінійним простором.
\end{example}

\begin{example}
Евклідів простір $\RR^n$ --- сукупність векторів
$(x_1, x_2, \dots, x_n)$, що складаються с дійсних чисел, є лінійним.
\end{example}

\begin{definition}
Лінійні простори $E$ і $F$ називаються
\vocab{ізоморфними}, якщо між їхніми елементами можна
установити взаємно-однозначну відповідність, яка
узгоджена із операціями в цих просторах, тобто $x \leftrightarrow x'$,
$y \leftrightarrow y'$, $x, y \in E$, $x', y' \in F$: $x + y \leftrightarrow x' + y'$, $\lambda x \leftrightarrow \lambda x'$.
\end{definition}

\begin{remark}
Ізоморфні простори можна вважати різними реалізаціями
одного простору.
\end{remark}

\begin{example}
Простір $\RR^n$ і простір поліномів, степінь яких
не перевищує $n - 1$ є ізоморфними.
\end{example}

\begin{definition}
Числова функція $f$, визначена на лінійному
просторі $E$, називається \vocab{функціоналом}.
\end{definition}

\begin{definition}
Функціонал $f$ називається \vocab{адитивним}, якщо
\begin{equation*}
    \forall x, y \in E: f(x + y) = f(x) + f (y).
\end{equation*}
\end{definition}

\begin{definition}
Функціонал називається \vocab{однорідним}, якщо
\begin{equation*}
    \forall \lambda \in \RR \forall x \in E: f(\lambda x) = \lambda f(x).
\end{equation*}
\end{definition}

\begin{definition}
Адитивний однорідний функціонал називається \vocab{лінійним}.
\end{definition}

\begin{definition}
Функціонал називається \vocab{неперервним у точці $x_0$},
якщо з того що послідовність $x_n$ прямує до $x_0$
випливає, що послідовність $f(x_n)$ прямує до $f(x_0)$.
\end{definition}

\begin{definition}
Сукупність усіх лінійних неперервних
функціоналів, заданих на лінійному топологічному
просторі $E$, називається \vocab{спряженим простором}, і
позначається як $\conjugate E$.
\end{definition}

\begin{example}
$I(f) = \int_a^b f(t) dt$ є лінійним функціоналом в $C[a, b]$.
\end{example}

\begin{definition}
Нехай $E$ --- лінійний простір. Визначений на
просторі $E$ функціонал $p(x)$ називається \vocab{опуклим}, якщо
\begin{equation*}
    \forall x, y \in E, 0 \le a \le 1:
    p(\lambda x + (1 - \lambda) y) \le \lambda p(x) + (1 - \lambda) p(y).
\end{equation*}
\end{definition}

\begin{definition}
Функціонал $p(x)$ називається \vocab{додатно-однорідним}, якщо
\begin{equation*}
    \forall x \in E, \lambda > 0: p(\lambda x) = \lambda p(x).
\end{equation*}
\end{definition}

\begin{example}
Будь-який лінійний функціонал є додатно-однорідним.
\end{example}

\begin{definition}
Непорожня підмножина $L'$ лінійного простору
$L$ називається \vocab{лінійним підпростором}, якщо вона сама
утворює лінійний простір відносно операцій додавання і
множення на число, уведених в просторі $L$.
\end{definition}

\section{Продовження функціоналів}

\begin{definition}
Нехай $E$ --- дійсний лінійний простір, а $E_0$ ---
його підпростір. До того ж на підпросторі $E_0$ заданий
деякий лінійний функціонал $f_0$. Лінійний функціонал $f$,
визначений на всьому просторі $E$, називається
\vocab{продовженням} функціонала $f_0$, якщо 
\begin{equation*}
    \forall x \in E_0: f_0(x) = f(x).
\end{equation*}
\end{definition}

\begin{theorem}[Хана---Банаха]
Нехай $p(x)$ ---додатно-однорідний
і опуклий функціонал, визначений на дійсному лінійному
просторі $L$, а $L_0$ --- лінійний підпростір в $L$. Якщо $f_0$ ---
лінійний функціонал, заданий на $L_0$ і підпорядкований на
цьому підпросторі функціоналу $p$, тобто
\begin{equation}
    \label{eq:9.1}
    f_0(x) \le p(x), \forall x \in L_0
\end{equation}
то функціонал $f_0$ може бути продовжений до лінійного
функціонала $f$, заданого на просторі $L$ і підпорядкованого
функціоналу $p$ на всьому просторі $L$:
\begin{equation}
    \label{eq:9.2}
    f(x) \le p(x), \forall x \in L.
\end{equation}
\end{theorem}

\begin{proof}
Покажемо, що якщо $L_0 \ne L$, то $f_0$ можна
продовжити на $L' \supset L_0$, зберігаючи умову підпорядкованості.
Нехай $z \in L' \setminus L_0$, а $L'$--- елементарне розширення $L_0$:
\begin{equation*}
    L' = \{x': x' = \lambda z + x, x \in L_0, \lambda \in \RR\} = \{L_0; z\}.
\end{equation*}

Якщо $f'$ --- шукане продовження $f_0$ на $L'$, то
\begin{equation*}
    f'(\lambda z + x) = \lambda f'(z) + f(x) = \lambda f'(z) + f_0(x).
\end{equation*}

Покладемо $f'(z) = c$. Тоді $f'(\lambda z + x) = \lambda c + f_0(x)$. Виберемо
$c$ так, щоб виконувалась умова підпорядкованості:
\begin{equation}
    \label{eq:9.3}
    \forall x \in L_0: f_0(x) + \lambda c \le p(x + \lambda z).
\end{equation}

Якщо $\lambda > 0$, поділимо \eqref{eq:9.3} на $\lambda$ і отримаємо еквівалентну
умову
\begin{equation}
    \label{eq:9.4}
    \forall x \in L_0: f_0(\tfrac{x}{\lambda}) + c \le p(\tfrac{x}{\lambda} + z) \implies
    c \le p(\tfrac{x}{\lambda} + z) - f_0(\tfrac{x}{\lambda}).
\end{equation}

Якщо $\lambda < 0$, поділимо \eqref{eq:9.3} на $-\lambda$. Тоді
\begin{equation}
    \label{eq:9.5}
    \forall x \in L_0: -f_0(\tfrac{x}{\lambda}) - c \le p(-\tfrac{x}{\lambda} - z) \implies
    c \ge -p(-\tfrac{x}{\lambda} - z) - f_0(\tfrac{x}{\lambda}).
\end{equation}

Покажемо, що число $c$, що задовольняє умови \eqref{eq:9.4} і \eqref{eq:9.5} існує.
Нехай $y'$ і $y'' \in L_0$, а $z \in L' \setminus L_0$. Тоді
\begin{equation*}
    f_0(y'' - y') = f_0(y'') - f_0(y') \le p(y'' - y') =
    p(y'' + z - y - z) \le p(y'' + z) + p(-y' - z).
\end{equation*}

З цього випливає, що
\begin{equation*}
    -f_0(y'') + p(y'' + z) \ge -f_0(y') - p(-y' - z).
\end{equation*}

Покладемо
\begin{equation*}
    c'' = \inf_{y''} (-f_0(y'') + p(y'' + z)), \quad c' = \sup_{y'} (-f_0(y') + p(-y' - z)). 
\end{equation*}

Оскільки $y'$ і $y''$ --- довільні, то з умови підпорядкованості
випливає, що $c'' \ge c'$. Отже, $\exists c: c'' \ge c \ge c'$.

Визначимо функціонал $f'$ на $L'$:
\begin{equation*}
    f'(\lambda z + x) = \lambda c + f_0(x).
\end{equation*}

За побудовою цей функціонал задовольняє умову \eqref{eq:9.1}. Отже,
якщо $f_0$ задано на $L_0 \subset L$ і задовольняє на $L_0$ умову \eqref{eq:9.1}, то
його можна продовжити на $L' \supset L$ із збереженням цієї умови \eqref{eq:9.2}.

Якщо в просторі $L$ існує злічена система елементів
$x_1, x_2, \dots, x_n, \dots$ така, що будь-який елемент простору $L$ можна
подати як (скінченну) лінійну комбінацію елементів $x_1, x_2, \dots, x_n, \dots$, то
продовження функціонала $f_0$ на $L$ можна побудувати за
індукцією, розглядаючи зростаючий ланцюжок підпросторів
\begin{equation*}
    L^{(1)} = \{L_0, x_1\}, \quad L^{(2)} = \{L^{(1)}; x_2\}, \quad \dots, \quad L^{(n)} = \{L^{(n - 1)}; x_n\}, \quad \dots,
\end{equation*}
де $L^{(k)} = \{L^{(k - 1)}; x_k\}$ --- мінімальний лінійний підпростір, що
містить $L^{(k - 1)}$ і $x_k$. Тоді кожний елемент $x \in L$ увійде в
деякий $L^{(k)}$ і функціонал $f_0$ буде продовжений на весь
простір $L$. 
\end{proof}

В загальному випадку використовується схема,
яка базується на лемі Цорна. Уведемо в розгляд потрібні
означення.

\section{Ланцюги і мажоранти}

\begin{definition}
Говорять, що на множині $X$ задано \vocab{відношення
часткового порядку} $\le$, якщо виділено деяку сукупність пар
$P = \{(x, y) \in X \times X\}$, для яких
\begin{enumerate}
    \item $x \le x$;
    \item $x \le y, y \le z \implies x \le z$.
\end{enumerate}
При цьому не вимагається, щоб усі елементи були порівняними.
\end{definition}

\begin{example}
Площина $\RR^2$, на якій між точками
$x = (x_1, x_2)$ і $y = (y_1, y_2)$
встановлено відношення $x \le y$, якщо
$x_1 \le y_1$ і $x_2 \le y_2$.
\end{example}

\begin{definition}
Якщо всі елементи $X$ є попарно порівняними, то
множина $X$ називається \vocab{лінійно упорядкованою}.
\end{definition}

\begin{definition}
Лінійно упорядкована підмножина частково
упорядкованої множини називається \vocab{ланцюгом}.
\end{definition}

\begin{example}
Пряма $\RR$ із покоординатним порядком, що
розглядається як підмножина площини $\RR^2$, є ланцюгом.
\end{example}

\begin{definition}
Якщо $X$ --- частково упорядкована множина і
$M \subset X$, то елемент $m^\star \in X$ називається \vocab{мажорантою}
множини $M$, якщо
\begin{equation*}
    m \le m^\star, \forall m \in M.
\end{equation*}
\end{definition}

\begin{definition}
Якщо $m_\star$ --- така мажоранта $M \subset X$, що $m_\star \le m'$
для будь-якої іншої мажоранти $m'$ множини $M$, то $m_\star$
називається \vocab{точною верхньою гранню} множини $M$.
\end{definition}

\begin{definition}
Елемент $m \in X$ називається \vocab{максимальним},
якщо немає такого елемента $m' \in X$, що $m \le m'$.
\end{definition}

\begin{lemma}[Цорна] Якщо будь-який ланцюг в частково
упорядкованій множині $X$ має мажоранту, то в $X$ існує
максимальний елемент.
\end{lemma}

\begin{proof}(теореми Хана---Банаха)
Позначимо через $\mathfrak{M}$ сукупність усіх можливих
продовжень функціоналу $f_0$ на більш широкі підпростори з
умовою підпорядкованості $p$. Кожне таке продовження $f'$
має лінійну область визначення $L'$, на якій $f' \le p$ і
$f'|_{X_0} = f$. Будемо вважати продовження $f'$ підпорядкованим
продовженню $f''$, якщо для відповідних областей визначення
маємо $L' \subset L''$ і $f''|_{L'} = f'$. Таким чином, маємо частковий
порядок. Умова щодо ланцюгів виконана: якщо дано ланцюг
продовжень $f_\alpha$ з областями визначення $L_\alpha$, то мажоранта
$f \in \mathfrak{M}$ будується так.
Розглянемо множину $L = \Bigcup_\alpha L_\alpha$, яка є
лінійним простором, оскільки $\forall x, y \in L$ $\exists L_\alpha, L_\beta$, такі що
$x \in L_\alpha$ і $y \in L_\beta$.
Але за означенням ланцюга або $L_\alpha \subset L_\beta$, або
$L_\beta \subset L_\alpha$, тобто $x + y \in L$.
Ясно, що $\lambda x \in L$, $\forall \lambda \in \RR$. З тих же
причин функціонал $f(x) = f_\alpha(x_\alpha)$ для $x = x_\alpha$ коректно
заданий на $L$, тобто $f_\alpha(x_\alpha) = f_\beta(x_\beta)$,
якщо $x_\alpha = x_\beta$. До того ж
$f \le p$ на $L$. Отже, $f \in \mathfrak{M}$ --- мажоранта для всіх $f_\alpha$. За
лемою Цорна в $\mathfrak{M}$ є максимальний елемент $f$. Отже,
область визначення функціонала $f$ збігається із $X$, інакше
функціонал $f$ можна було б лінійно продовжити на більш
широкий простір із умовою підпорядкованості $p$, що
суперечить максимальності $p$.
\end{proof}


\section{Література}

\begin{enumerate}[label={[\arabic*]}]
\item \textbf{Садовничий~В.~А.}
Теория операторов /
В.~А.~Садовничий ---
М.: Изд-во Моск. ун-та, 1986 (стр.~91--96, 106--109).
\item \textbf{Колмогоров~А.~Н.}
Элементы теории функций и функционального анализа. 5-е изд. /
А.~Н.~Колмогоров, С.~В.~Фомин ---
М.: Наука, 1981 (cтр.~119--138).
\item \textbf{Богачев~В.~И.}
Действительный и функциональный анализ. Университетский курс /
В.~И.~Богачев, О.~Г.~Смолянов ---
М.: Ижевск: НИЦ ``Регулярная и хаотическая динамика'', 2009 (стр.~14--16, 258--264).
\end{enumerate}
