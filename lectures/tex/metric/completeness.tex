\chapter{Повні метричні простори}

\section{Повнота, ізометрія і поповнення}

\begin{definition}
Метричний простір називається \vocab{повним}, якщо
в ньому будь-яка фундаментальна послідовність має
границю.
\end{definition}

\begin{example}
$\left( \RR^n, \sqrt{\sum_{i = 1}^n (x_i - y_i)^2} \right)$.
\end{example}

\begin{example}
$\left( C[a, b], \max_{t \in [a, b]} |x(t) - y(t)| \right)$.
\end{example}

\begin{definition}
Бієктивне відображення~$\phi$ одного метричного
простору~$(E_1, \rho_1)$ на інший~$(E_2, \rho_2)$ називається
\vocab{ізометрією}, якщо
\begin{equation*}
    \forall x_1, x_2 \in E_1: \rho_1(x_1, x_2) = \rho_2(\phi(x_1), \phi(x_2)).
\end{equation*}
\end{definition}

\begin{definition}
Метричні простори, між якими існує
ізометрія, називаються \vocab{ізометричними}.
\end{definition}

\begin{definition}
Повний метричний простір~$(\tilde E, \tilde \rho)$ називається
\vocab{поповненням} метричного простору~$(E, \rho)$, якщо
\begin{enumerate}
    \item $E \subset \tilde E$;
    \item $\closure E = \tilde{E}$.
\end{enumerate}
\end{definition}

\begin{theorem}
[про поповнення метричного простору, Хаусдорф]
Будь-який метричний простір має поповнення,
єдине з точністю до ізометрії, що залишає точки простору
нерухомими.
\end{theorem}

\begin{lemma}
Якщо фундаментальна послідовність містить
збіжну підпослідовність, то сама послідовність збігається
до тієї ж границі.
\end{lemma}

\begin{proof}
Припустимо, що~$\lim_{n_k \to \infty} \rho(x_{n_k}, x_0) = 0$, тобто
\begin{equation*}
    \forall \epsilon > 0 \exists N_1(\epsilon) > 0: \forall n \ge N_1: \rho(x_{n_k}, x_0) < \epsilon.
\end{equation*}

За нерівністю трикутника
\begin{equation*}
    \rho(x_n, x) \le \rho(x_n, x_{n_k}) + \rho(x_{n_k}, x).
\end{equation*}

Оскільки послідовність~$\{x_n\}_{n \in \NN}$ є фундаментальною,
\begin{equation*}
    \forall \epsilon > 0 \exists N_2(\epsilon) > 0: \forall n, m \ge N_2: \rho(x_n, x_m) < \epsilon.
\end{equation*}

Таким чином,
\begin{equation*}
    \forall \epsilon > 0 \forall n, n_k \ge \max(N_1, N_2):
    \rho(x_n, x_0) \le \rho(x_n, x_{n_k}) + \rho(x_{n_k}, x_0) <
    \epsilon + \epsilon = 2 \epsilon. \qedhere
\end{equation*}
\end{proof}

\begin{lemma}
Будь-яка підпослідовність фундаментальної
послідовності є фундаментальною.
\end{lemma}

\begin{proof}
За нерівністю трикутника
\begin{equation*}
    \rho(x_{n_k}, x_{n_l}) \le \rho(x_{n_k}, x_n) + \rho(x_n, x_{n_l}).
\end{equation*}

Оскільки послідовність~$\{x_n\}_{n \in \NN}$ є фундаментальною,
\begin{equation*}
    \forall \epsilon > 0 \exists N(\epsilon) > 0: \forall n, m \ge N: \rho(x_n, x_m) < \epsilon.
\end{equation*}

Отже,
\begin{equation*}
    \forall \epsilon > 0 \forall n, n_k, n_l \ge N:
    \rho(x_{n_k}, x_{n_l}) \le \rho(x_{n_k}, x_n) + \rho(x_n, x_{n_l})
    < \epsilon + \epsilon = 2 \epsilon. \qedhere
\end{equation*}
\end{proof}

\section{Вкладені кулі і повнота}

\begin{theorem}
[принцип вкладених куль] Для того щоб
метричний простір був повним, необхідно і достатньо, щоб
у ньому будь-яка послідовність замкнених вкладених одна в
одну куль, радіуси яких прямують до нуля, мала непорожній
перетин.
\end{theorem}

\begin{proof}
Необхідність. Нехай~$(X, \rho)$~--- повний
метричний простір, а~$S_1^\star(x_1, r_1) \supset S_2^\star(x_2, r_2) \supset \dots$~--- вкладені
одна в одну замкнені кулі.

Послідовність їх центрів є фундаментальною, оскільки
\begin{equation*}
    \rho(x_n, x_m) < r_n \text{ при } m > n, \text{ а } r_n \to 0 \text{ при } n \to \infty.
\end{equation*}

Оскільки~$(X, \rho)$~--- повний метричний простір, існує
елемент~$x = \lim_{n \to \infty} x_n$,~$x \in X$.

Покажемо, що~$x$ належить всім кулям~$S_n^\star(x_n, r_n)$,~$n \in \NN$,
тобто~$x \in \Bigcap_{n = 1}^\infty S_n^\star(x_n, r_n)$. Дійсно, оскільки
$x = \lim_{n \to \infty} x_n$, то
\begin{equation*}
    \forall \epsilon > 0 \exists N > 0: \forall n \ge N: \rho(x_n, x) < \epsilon.    
\end{equation*}

Значить, в довільному околі точки~$x$ знайдеться
нескінченна кількість точок із послідовності~$\{x_n\}$,
починаючи з деякого номера~$N$. Оскільки кулі вкладені
одна в одну, ці точки належать всім попереднім кулям
$S_1^\star, S_2^\star, \dots, S_{N - 1}^\star$.
Отже, для довільного~$n$ точка~$x$ є точкою
дотику множини~$S_n^\star$, тобто належить його замиканню.
Оскільки кожна куля є замкненою, точка~$x$ належить всім
$S_n^\star$. Це означає, що
\begin{equation*}
    x \in \Bigcap_{n = 1}^\infty S_n^\star.
\end{equation*}

Достатність. Покажемо, що якщо~$\{x_n\}_{n \in \NN}$ ---
фундаментальна послідовність, то вона має границю~$x \in X$.

\begin{enumerate}
\item Оскільки послідовність~$\{x_n\}_{n = 1}^\infty$ є фундаментальною, то
$\forall \epsilon > 0 \exists n_1 > 0$:
$\forall n \ge n_1$~$\rho(x_n, x_{n_1}) < \epsilon$.
Поклавши~$\epsilon = \frac{1}{2}$, ми можемо вибрати точку
$x_{n_1}$ так, що~$\rho(x_n, x_{n_1}) < \frac{1}{2}$ для
довільного~$n > n_1$. Зробимо точку
$x_{n_1}$ центром замкненої кулі радіуса~$1$: 
$S_1^\star(x_{n_1}, 1)$.

\item Оскільки підпослідовність~$\{x_n\}_{n = n_1}^\infty$
є фундаментальною (за лемою 7.2), то поклавши~$\epsilon = \frac{1}{2^2}$,
можна вибрати точку~$x_{n_2}$ x таку, що 
$\rho(x_n, x_{n_2}) < \frac{1}{2^2}$ для довільного
$n > n_2 > n_1$. Зробимо точку
$x_{n_2}$ центром замкненої кулі
радіуса~$\frac{1}{2}$: $S_2^\star(x_{n_2}, \frac{1}{2})$.

\dots

\item[$k$.] Нехай~$x_{n_1}, x_{n_2}, \dots, x_{n_{k - 1}}$,
де~$n_1 < n_2 < \dots < n_{k - 1}$ уже вибрані.
Тоді, оскільки підпослідовність~$\{x_n\}_{n = n_{k - 1}}^\infty$
є фундаментальною, покладемо~$\epsilon = \frac{1}{2^k}$ і виберемо
точку~$x_{n_k}$ так, щоб виконувалися умови 
$\rho(x_n, x_{n_k}) < \frac{1}{2^k}$
для довільного~$n \ge n_k > n_{k - 1}$. Як і раніше, будемо
вважати точку~$x_{n_k}$ центром замкненої кулі радіуса
$\frac{1}{2^{k - 1}}$: $S_k^\star(x_{n_k}, \frac{1}{2^{k - 1}})$.

\dots
\end{enumerate}

Продовжуючи цей процес, ми отримаємо послідовність
замкнених куль, радіуси яких прямують до нуля. Покажемо,
що ці кулі вкладаються одна в одну, тобто
\begin{equation*}
    S_{k + 1}^\star ( x_{n_{k + 1}}, \tfrac{1}{2^k} ) \subset
    S_k^\star ( x_{n_k}, \tfrac{1}{2^{k - 1}} ).
\end{equation*}

Нехай точка~$y \in S_{k + 1}^\star ( x_{n_{k + 1}}, \tfrac{1}{2^k} )$.
Значить,~$\rho(y, x_{n_{k + 1}}) \le \frac{1}{2^k}$.
За нерівністю трикутника
\begin{equation*}
    \rho(y, x_{n_k}) \le \rho(y, x_{n_{k + 1}}) + \rho(x_{n_{k + 1}}, x_{n_k}).
\end{equation*}

Оскільки~$n_{k + 1} > n_k$, то~$\rho(x_{n_{k + 1}}, x_{n_k}) < \frac{1}{2^k}$. Значить,
\begin{equation*}
    \rho(y, x_{n_k}) \le \frac{1}{2^k} + \frac{1}{2^k} =
    \frac{2}{2^k} = \frac{1}{2^{k - 1}}.
\end{equation*}

Інакше кажучи,
\begin{equation*}
    y \in S_k^\star ( x_{n_k}, \tfrac{1}{2^{k - 1}} )
\end{equation*}

Таким чином, ми побудували послідовність вкладених одна
в одну замкнених куль, радіуси яких прямують до нуля.
За припущенням, в просторі~$(X, \rho)$ існує точка~$x$,
спільна для всіх таких куль:
$x \in \Bigcap_{k = 1}^\infty S_k^\star(x_{n_k}, \frac{1}{2^{k - 1}})$. 
Крім того, за побудовою,~$\rho(x_n, x) = \frac{1}{2^{k - 1}} \to 0$,
коли~$k \to \infty$. Таким чином,
фундаментальна послідовність~$\{x_n\}$ містить
підпослідовність~$\{x_{n_k}\}$, що збігається до деякої точки в
просторі~$(X, \rho)$. Із леми 7.1 випливає, що і вся
послідовність~$\{x_n\}$ прямує то тієї ж точки. Таким чином,
простір~$(X, \rho)$ є повним.
\end{proof}

\begin{remark}
Покажемо, що умову~$r_n \to 0$ зняти не
можна. Розглянемо метричний простір~$(\NN, \rho)$, де
\begin{equation*}
    \rho(n, m) = \begin{cases}
        1 + \frac{1}{n + m}, & n \ne m, \\
        0, & \text{інакше}.
    \end{cases}
\end{equation*}

Визначимо послідовність замкнених куль з центрами в
точках~$n$ і радіусом~$1 + \frac{1}{2 n}$:
\begin{equation*}
    \closure S (n, 1 + \tfrac{1}{2n}) =
    \{ m: \rho(n, m) \le 1 + \tfrac{1}{2n} \} =
    \{n, n + 1, \dots\}, \quad n = 1, 2, \dots.
\end{equation*}

Ці кулі є вкладеними одна в одну і замкненими, простір є
повним, але перетин куль є порожнім (яке б число ми не
взяли, знайдеться нескінченна кількість куль, які лежать
правіше цієї точки). Отже, необхідні умови в принципі
вкладених куль не виконуються.
\end{remark}

\section{Категорії множин}

\begin{definition}
Підмножина~$M$ метричного простору~$(X, \rho)$
називається \vocab{множиною першої категорії}, якщо її
можна подати у вигляді об’єднання не більш ніж зліченої
кількості ніде не щільних множин.
\end{definition}

\begin{definition}
Підмножина~$M$ метричного простору~$(X, \rho)$
називається \vocab{множиною другої категорії}, якщо вона не є
множиною першої категорії.
\end{definition}

\begin{theorem}
[теорема Бера про категорії] Нехай
$(X, \rho)$--- непорожній повний метричний простір, тоді~$X$
є множиною другої категорії.
\end{theorem}

\begin{proof}
Припустимо супротивне, тобто
\begin{equation*}
    X = \Bigcup_{n = 1}^\infty E_n,
\end{equation*}
і кожна множина~$E_n$,~$n = 1, 2, \dots$ є ніде не щільною в~$X$.
Нехай~$S_0$~--- деяка замкнена куля радіуса~$1$.

Оскільки множина~$E_1$ є ніде не щільною, існує замкнена
куля~$S_1$, радіус якої менше~$\frac{1}{2}$, така що
\begin{equation*}
    S_1 \subset S_0 \text{ і } S_1 \cap E_1 = \emptyset.
\end{equation*}

(Якщо існує куля радіуса більше~$\frac{1}{2}$, що задовольняє таким
умовам, то ми виберемо в ній кулю, радіуса менше~$\frac{1}{2}$.)

Оскільки множина~$E_2$ є ніде не щільною, існує замкнена
куля~$S_2$, радіус якої менше~$\frac{1}{2^2}$, така що
\begin{equation*}
    S_2 \subset S_1 \text{ і } S_2 \cap E_2 = \emptyset.
\end{equation*}

Продовжуючи цей процес, ми отримаємо послідовність
вкладених одна в одну замкнених куль~$\{S_n\}_{n \in \NN}$, радіуси яких
прямують до нуля. За принципом вкладених куль існує 
точка~$x \in \Bigcap_{n = 1}^\infty S_n \cap X$.
Оскільки за побудовою~$S_n \cap E_n = \emptyset$, то
$x \not\in E_n$,~$\forall n = 1, 2, \dots$. Значить,
$x \not\in \Bigcup_{n = 1}^\infty E_n$.. Це суперечить
припущенню, що~$X = \Bigcup_{n = 1}^\infty E_n$. 
\end{proof}

\section{Стискаючі відображення}

\begin{definition}
Відображення~$g: (X, \rho) \to (X, \rho)$ називається
\vocab{стискаючим}, якщо існує таке число~$0 < a < 1$, що
$\rho(g(x), g(y)) \le a \rho(x, y)$ для довільних~$x, y \in X$ .
\end{definition}

\begin{theorem}
Будь-яке стискаюче відображення є неперервним.
\end{theorem}

\begin{proof}
Нехай~$x_n \to x$, а~$g: X \to X$ є стискаючим відображенням. Тоді
\begin{equation*}
    0 \le \rho(g(x_n), g(x)) \le \alpha \rho(x_n, x) \to 0 \text{ при } n \to \infty.
\end{equation*}

Отже,
\begin{equation*}
    g(x_n) \to g(x), \text{ коли } x_n \to x. \qedhere
\end{equation*}
\end{proof}

\begin{theorem}
[принцип стискаючих відображень Банаха]
Будь-яке стискаюче відображення повного
метричного простору~$(X, \rho)$ в себе має лише одну
нерухому точку, тобто~$\exists! x \in X$: $g(x) = x$.
\end{theorem}

\begin{proof}
Нехай~$x_0$~--- деяка точка із~$X$. Визначимо
послідовність точок~$\{x_n\}_{n \in \NN}$ за таким правилом:
\begin{equation*}
    x_1 = g(x_0), \dots, x_n = g(x_{n - 1}).
\end{equation*}

Покажемо, що ця послідовність є фундаментальною.
Дійсно, якщо~$m > n$, то
\begin{multline*}
    \rho(x_n, x_m) = \rho(g(x_{n - 1}), g(x_{m - 1})) \le
    \alpha \rho(x_{n - 1}, x_{m - 1}) \le \dots \le \\
    \alpha^n \rho(x_0, x_{m - n}) \le
    \alpha^n ( \rho(x_0, x_1) + \rho(x_1, x_2) + \dots + \rho(x_{m - n - 1}, x_{m - n}) \le \\
    \alpha^n \rho(x_0, x_1) (1 + \alpha + \alpha^2 + \dots + \alpha^{m - n - 1}) \le
    \alpha^n \rho(x_0, x_1) \tfrac{1}{1 - \alpha}.
\end{multline*}

Таким чином, оскільки~$0 < \alpha < 1$,
\begin{equation*}
    \rho(x_n, x_m) \to 0, n \to \infty, m \to \infty, m > n.
\end{equation*}

Внаслідок повноти простору~$(X, \rho)$ в ньому існує границя
послідовності~$\{x_n\}$. Позначимо її через~$x = \lim_{n \to \infty} x_n$.

Із теореми 7.3 випливає, що
\begin{equation*}
    g(x) = \lim_{n \to \infty} g(x_n) = \lim_{n \to \infty} x_{n + 1} = x.
\end{equation*}
Отже, нерухома точка існує.

Доведемо її єдиність. Якщо~$g(x) = x$ і~$g(y) = y$, то
$\rho(x, y) \le a \rho(x, y)$, тобто~$\rho(x, y) = 0$.
За аксіомою тотожності (невиродженості) це означає, що~$x = y$. 
\end{proof}

\begin{corollary}
Умову~$a < 1$ не можна замінити на~$a \le 1$.
\end{corollary}

\begin{proof}
Якщо відображення~$g: (X, \rho) \to (X, \rho)$
має властивість~$\rho(g(x), g(y)) \le \rho(x, y)$,
$\forall x, y \in X$,~$x \ne y$,
то нерухомої точки може не бути.
Дійсно, розглянемо простір~$([1, \infty), |x - y|)$
і визначимо відображення~$g(x) = x + \frac{1}{x}$.
Тоді~$\rho(g(x), g(y)) = |x + \frac{1}{x} - y - \frac{1}{y}| < |x - y|$. 
Оскільки для жодного~$x \in [1, \infty)$
$g(x) = x + \frac{1}{x} \ne x$,
нерухомої точки немає.
\end{proof}

\section{Література}

\begin{enumerate}[label={[\arabic*]}]
\item \textbf{Садовничий~В.~А.}
Теория операторов. /
В.~А.~Садовничий ---
М.: Изд-во Моск. ун-та, 1986 (стр.~41--47).
\item \textbf{Колмогоров~А.~Н.}
Элементы теории функций и функционального анализа. 5-е изд. /
А.~Н.~Колмогоров, С.~В.~Фомин ---
М.: Наука, 1981 (стр. 66--75).
\end{enumerate}
