\chapter{Слабка топологія і слабка збіжність}

Ми розглянули поняття сильної топології і сильної
збіжності в нормованому просторі $E$, а також сильної
топології і сильної збіжності в спряженому просторі $\conjugate E$.
Ці топології та поняття збіжності спиралися на поняття норми.

Розглянемо відповідні поняття слабкої топології і слабкої
збіжності в нормованих просторах $E$ і $\conjugate E$.

\section{Слабка топологія}

\begin{definition}
\vocab{Слабкою топологією} в просторі $\conjugate E$
називається топологія, визначена локальною базою нуля,
тобто сукупністю множин
\begin{equation*}
    U_{f_1, f_2, \dots, f_n; \epsilon} =
    \{ x \in L: |f_i(x)| < \epsilon, i = 1, 2, \dots, n\},
\end{equation*}
де $f_1, f_2, \dots, f_n$ --- скінченна сукупність неперервних
функціоналів, а $\epsilon$ --- довільне додатне число.
\end{definition}

\begin{lemma}
Слабка топологія слабкіша за вихідну топологію простору $L$.
\end{lemma}

\begin{proof}
Розглянемо скінчену сукупність неперервних
функціоналів $f_1, f_2, \dots, f_n$ і довільне додатне число $\epsilon$.

Тоді внаслідок неперервності функціоналів $f_1, f_2, \dots, f_n$
множина $U_{f_1, f_2, \dots, f_n; \epsilon}$
є відкритою в вихідній топології простору $L$,
оскільки прообразом відкритої множини
при неперервному відображенні є відкрита множина,
і містить нуль, тобто є околом нуля,
оскільки ці функціонали є лінійними.

Перетин двох таких околів сам містить множину точок,
в яких скінченна кількість функціоналів за модулем менше $\epsilon$,
отже, виконується критерій локальної бази.

Оскільки нова топологія є лише частиною локальною бази нуля
в вихідній топології, вона є слабкішою. 
\end{proof}

\begin{remark}
Слабка топологія є найменшою з усіх
топологій, в яких є неперервними всі лінійні функціонали,
неперервні у природній топології простору.
\end{remark}

\begin{remark}
У нормованому просторі слабка
топологія задовольняє аксіому $T_2$, але може не задовольняти
першу аксіому зліченності, отже, вона не описується за
допомогою збіжних послідовностей.
\end{remark}

\section{Слабка збіжність}

\begin{definition}
Послідовність 
називається \vocab{слабко збіжною},
якщо вона є збіжною в слабкій топології.
\end{definition}

\begin{lemma}
Послідовність $\{x_n\}_{n = 1}^\infty$
елементів лінійного топологічного простору $L$ є слабко збіжною до $x_0 \in L$ тоді
і лише тоді, коли для будь-якого неперервного лінійного
функціонала $f$ на $L$ числова послідовність $f(x_n)$
збігається до $f(x_0)$.
\end{lemma}

\begin{proof}
Необхідність. Без обмеження загальності,
розглянемо випадок $x_0 = 0$. Якщо для будь-якого околу
$U_{f_1, \dots, f_k; \epsilon}$ в слабкій топології існує таке число $N$, що
$x_n \in U_{f_1, \dots, f_k; \epsilon}$ для всіх $n \ge N$, то ця умова виконується і для
околу $U_{f;\epsilon}$, де $f \in \conjugate L$ --- довільний фіксований функціонал, а
це означає, що $f(x_n) \to 0$ при $n \to \infty$.

Достатність. Припустимо, що $f(x_n) \to 0$ для будь-якого
$f \in \conjugate L$. Тоді ця умова виконується і для всіх функціоналів
$f_i \in \conjugate L$, $i = 1, 2, \dots, k$, що визначають довільний окіл в слабкій
топології:
\begin{equation*}
    U_{f_1, f_2, \dots, f_k; \epsilon} =
    \{ x \in L: |f_i(x)| < \epsilon, i = 1, 2, \dots, k \}.
\end{equation*}

Виберемо числа $N_i$ так, щоб $|f_i(x_n)| < \epsilon$ при $n \ge N_i$ і
покладемо $N = \max_{i = 1, \dots, k} N_i$. Отже, при всіх $n \ge N$ виконується
умова $x_n \in U$. Це означає, що послідовність $\{x_n\}_{n = 1}^\infty$
збігається в слабкій топології. 
\end{proof}

\begin{lemma}
Будь-яка сильно збіжна послідовність є слабко
збіжною, але не навпаки.
\end{lemma}

\begin{proof}
Відповідно до леми 12.1, слабка топологія
слабкіша за вихідну топологію лінійного топологічного
простору, тому будь-яка послідовність, що збігається в
сильній топології, буде збігатися і в слабкій.

Обернене твердження є невірним, тому що, наприклад, в
просторі $\ell_2$ послідовність ортів $e_n = (0, 0, \dots, 0, 1, 0, \dots)$ слабко
збігається до нуля, але не збігається до нуля сильно. 
\end{proof}

Розглянемо поняття слабкої збіжності в нормованому
просторі $E$.

\begin{theorem}
Якщо послідовність $\{x_n\}_{n = 1}^\infty$
слабко збігається в нормованому просторі $E$, то існує така
константа $C$, що
\begin{equation*}
    \|x_n\| \le C
\end{equation*}
тобто будь-яка слабко збіжна послідовність в нормованому
просторі є обмеженою.
\end{theorem}

\begin{proof}
Розглянемо в просторі $\conjugate E$ множини
\begin{equation*}
    A_{k,n} = \{f \in \conjugate E: |f(x_n)| \le k\}, \quad
    k, n = 1, 2, \dots
\end{equation*}

Оскільки при фіксованому $x_n$ функціонали $\phi_{x_n}(f) = f(x_n)$ є
неперервними (лема 11.2), множини $A_{k,n}$ є замкненими.

Дійсно,
\begin{equation*}
    f_m \to f, f_m \in A_{k,n} \implies
    \phi_{x_n}(f_m) = f_m(x_n) \le k \implies
    f(x_n) \le k.
\end{equation*}

Отже, множина
\begin{equation*}
    A_k = \Bigcap_{n = 1}^\infty A_{k,n}
\end{equation*}
є замкненою.

Оскільки послідовність $\{x_n\}_{n = 1}^\infty$
збігається слабко, послідовність $\phi_{x_n}(f)$ є обмеженою для кожного
$f \in \conjugate E$.

Дійсно,
\begin{equation*}
    x_n \weakto x \implies
    \phi_{x_n}(f) = f(x_n) \to f(x) \implies
    \exists k > 0: |f(x_n)| \le k.
\end{equation*}

Отже, будь-який функціонал $f \in \conjugate E$ належить деякій
множині $A_k$, тобто
\begin{equation*}
    \conjugate E = \Bigcup_{k = 1}^\infty A_k.
\end{equation*}

Оскільки простір $\conjugate E$ є повним (теорема 11.3), то за теоремою
Бера хоча б одна з множин $A_k$, наприклад,
$A_{k_0}$ повинна буди
щільною в деякій кулі $S(f_0, \epsilon)$. Оскільки множина $A_{k_0}$
замкненою, це означає, що
\begin{equation*}
    S(f_0, \epsilon) \subset \closure A_{k_0} = A_{k_0}.
\end{equation*}

Звідси випливає, що послідовність $\{\phi_{x_n}(f)\}_{n = 1}^\infty$
є обмеженою на кулі $S(f_0, \epsilon)$, а значить, на будь-якій кулі в просторі $\conjugate E$,
оскільки $\conjugate E$ є лінійним топологічним простором. Зокрема, це
стосується одиничної кулі. Таким чином, послідовність
$\{x_n\}_{n = 1}^\infty$
є обмеженою як послідовність елементів з $\dconjugate{E}$
Оскільки природне відображення $\pi: E \to \dconjugate{E}$ є ізометричним,
це означає обмеженість послідовності $\{x_n\}_{n = 1}^\infty$ в просторі $E$.
\end{proof}

\begin{theorem}
Послідовність $\{x_n\}_{n = 1}^\infty$елементів
нормованого простору $E$ слабко збігається до $x \in E$, якщо
\begin{enumerate}
\item значення $\|x_n\|$ є обмеженими в сукупності деякою
константою $M$;
\item $f(x_n) \to f(x)$ для будь-яких функціоналів $f$, що
належать множині, лінійні комбінації елементів якого скрізь
щільними в $\conjugate E$.
\end{enumerate}
\end{theorem}

\begin{proof}
Із умови 2) і властивостей операцій над
лінійними функціоналами випливає, що якщо $\phi$ --- лінійна
комбінація функціоналів $f$, то
\begin{equation*}
    \phi(x_n) \to \phi(x).
\end{equation*}

Нехай $\phi$ --- довільний елемент з $\conjugate E$ і $\{\phi_k\}_{k = 1}^\infty$
--- сильно збіжна до $\phi$ послідовність лінійних комбінацій із
функціоналів $f$, тобто $\|\phi_k - \phi\| \to 0$ (вона завжди існує
внаслідок щільності). Покажемо, що $\phi(x_n) \to \phi(x)$.

Нехай $M$ задовольняє умову
\begin{equation*}
    \|x_n\| \le M, \quad n = 1, 2, \dots, n, \dots, \quad \|x\| \le M.
\end{equation*}

Оскільки $\phi_k \to \phi$, то
\begin{equation*}
    \forall \epsilon > 0 \exists K \in \NN:
    \forall k \ge K: \|\phi - \phi_k\| < \epsilon.
\end{equation*}

З цього випливає, що
\begin{multline*}
    |\phi(x_n) - \phi(x)| \le
    |\phi(x_n) - \phi_k(x_n) +
    |\phi_k(x_n) - \phi_k(x)| +
    |\phi_k(x) - \phi(x)| \le \\
    \|\phi - \phi_k\| M +
    |\phi_k(x_n) - \phi_k(x)| +
    \|\phi - \phi_k\| M \le 
    \epsilon M + 
    |\phi_k(x_n) - \phi_k(x)| +
    \epsilon M.
\end{multline*}

За умовою теореми, $\phi_k(x_n) \to \phi_k(x)$ при $n \to \infty$. Отже,
\begin{equation*}
    \phi(x_n) - \phi(x) \to 0, \quad n \to \infty, \quad \forall \phi \in \conjugate E. \qedhere
\end{equation*}
\end{proof}

\section{Види топології у спряженому просторі}

Розглянемо поняття слабкої топології в спряженому
просторі $\conjugate E$. Спочатку згадаємо, що із означення 11.3 сильної
топології в спряженому просторі випливає, що цю топологію
можна задати за допомогою локальної бази нуля. Наведемо її
еквівалентне формулювання.

\begin{definition}
\vocab{Сильною топологією} в спряженому просторі
$\conjugate E$ називається топологія, визначена локальною базою нуля,
тобто сукупністю множин
\begin{equation*}
    B_{\epsilon, A} = \{ f \in \conjugate E: |f(x)| < \epsilon, x \in A \subset E\},
\end{equation*}
де $A$ --- довільна обмежена множина в $E$, а $\epsilon$ --- довільне
додатне число.
\end{definition}

\begin{remark}
Оскільки будь-яка скінченна множина є
обмеженою, то слабка топологія в $\conjugate E$ є слабкішою, ніж
сильна топологія цього простору.
\end{remark}

\begin{definition}
Послідовність $\{f_n\}_{n = 1}^\infty$
називається \vocab{слабко збіжною},
якщо вона є збіжною в слабкій топології $\conjugate E$,
інакше кажучи, $f_n(x) \to f(x)$ для кожного $x \in E$.
\end{definition}

\begin{remark}
В спряженому просторі сильно збіжна
послідовність є одночасно слабко збіжною, але не навпаки.
\end{remark}

В спряженому просторі мають місце теореми, аналогічні
теоремам 12.1 і 12.2.

\begin{theorem}
Якщо послідовність лінійних функціоналів
$\{f_n\}_{n = 1}^\infty$
слабко збігається на банаховому просторі $E$, то
існує така константа $C$, що
\begin{equation*}
    \|f_n\| \le C,
\end{equation*}
тобто будь-яка слабко збіжна послідовність простору,
спряженого до банахова простору, є обмеженою.
\end{theorem}

\begin{theorem}
Послідовність лінійних функціоналів $\{f_n\}_{n = 1}^\infty$
елементів спряженого простору $\conjugate E$ слабко збігається до
$f \in E$, якщо
\begin{enumerate}
\item послідовність $\|f_n\|$ є обмеженою, тобто
\begin{equation*}
\exists C \in \RR: \|f_n\| \le C, \quad n = 1, 2, \dots;
\end{equation*}
\item $\phi_x(f_n) \to \phi_x(f)$ для будь-яких елементів $x$, що
належать множині, лінійні комбінації елементів якої скрізь
щільними в $E$.
\end{enumerate}
\end{theorem}

\begin{remark}
Простір $\conjugate E$ лінійних неперервних функціоналів, заданих
на просторі $E$, можна тлумачити і як простір, спряжений до
простору $E$, і як основний простір, спряженим до якого є
простір $\dconjugate{E}$. Відповідно, слабку топологію в просторі $\conjugate E$
можна ввести або за означенням 12.4 (через скінченні
множини елементів простору $E$), або як в основному
просторі відповідно до означення 12.1 (через функціонали із
простору $\dconjugate{E}$). Для рефлексивних просторів це одне й теж, а
для нерефлексивних просторів ми таким чином отримуємо
різні слабкі топології.
\end{remark}

\begin{definition}
Топологія в спряженому просторі $\conjugate E$, що
вводиться за допомогою простору $\dconjugate{E}$ (як в означенні 12.1),
називається \vocab{слабкою} і позначається як $\sigma(\conjugate E, \dconjugate{E})$.
\end{definition}

\begin{definition}
Топологія в спряженому просторі $\conjugate E$, що
вводиться за допомогою простору $E$ (як в означенні 12.4),
називається \vocab{$^\star$-слабкою} і позначається як $\sigma(\conjugate E, E)$.
\end{definition}

\begin{remark}
Очевидно, що $\star$-слабка топологія в $\conjugate E$ є
більш слабкою, ніж слабка топологія простору $E$, тобто в
слабкій топології не менше відкритих множин, ніж в
$\star$-слабкій топології.
\end{remark}

\section{Література}

\begin{enumerate}[label={[\arabic*]}]
\item \textbf{Садовничий~В.~А.}
Теория операторов /
В.~А.~Садовничий ---
М.: Изд-во Моск. ун-та, 1986 (стр.~114--117).
\item \textbf{Колмогоров~А.~Н.}
Элементы теории функций и функционального анализа. 5-е изд. /
А.~Н.~Колмогоров, С.~В.~Фомин ---
М.: Наука, 1981 (стр.~192--202).
\end{enumerate}
